% --------------------------------------------------------------------------------
% Reference Manual: About the documentation
% --------------------------------------------------------------------------------

% ================================================================================
\chapter{About the documentation}
% ================================================================================

% --------------------------------------------------------------------------------
\hypertarget{FuncRef:DocumentationStructure}{
\section{Documentation structure}}
% --------------------------------------------------------------------------------

The description for each function is always structured in the
following way:

\section*{Function's name}

\textbf{Usage:}
\begin{verbatim}  This shows the syntax for using the function: what and how many
  input arguments and outputs are there, and what is their order ?
\end{verbatim}
\textbf{Description:}
\begin{verbatim}  This shows the semantics of the function:
  what is it intended for and/or how does it work ?
\end{verbatim}
\textbf{Input arguments:}
\begin{verbatim}  A description of the input arguments together with a specification
  of whether they are optional or not, and what their defaults are.
\end{verbatim}
\textbf{Output:}
\begin{verbatim}  A description of the outputs of the function.
\end{verbatim}
\textbf{Remarks:}
\begin{verbatim}  Any additional remarks that are not in the other sections.
\end{verbatim}
\textbf{Example:}
\begin{verbatim}  A 'copy-and-paste' example of how to use the function.
\end{verbatim}
\textbf{Authors:}
\begin{verbatim}  A list of the people who wrote (parts of) the code of the function,
  or have made some changes to it. Accompanied by the date of their most
  recent changes. As always, dates are written in the following format:
  YYYYMMDD (year,month,day)
\end{verbatim}


% --------------------------------------------------------------------------------
\hypertarget{FuncRef:ReferenceTerminology}{
\section{Used terminology}}
% --------------------------------------------------------------------------------

\subsection{Prefixes}

In the function declarations, input arguments are always starting
with the prefix 'in', while output arguments are always starting
with the prefix 'out'. In some cases 'io' is used for arguments
that are used for both input and output.


\subsection{Signal and SampleFreq}

Signal is a general name for any matrix representing a
time-varying entity in possibly more than one dimensions.

Single-channel signals are represented by a row-vector.
Multi-channel signals are represented by a matrix in which each
row represents one channel. Each column thus represents a moment
in time.

A signal is always accompanied by a sample frequency, called
something like SampleFreq and is specified in Hertz (Hz).


\subsection{ANI, ANIFreq and ANIFilterFreqs}

ANI is an abbreviation of Auditory Nerve Image and is used for a
(multi-channel) signal coming out of the auditory model. ANIFreq
specifies the sample frequency of such an ANI, and ANIFilterFreqs
specifies the center frequencies of the band pass filters that
were used for calculating the ANI.


\subsection{FileName and FilePath}

These are used for functions that take input from / generate
output to a file.

FileName is a Matlab string representing the name of the file,
while FilePath is a Matlab string representing the complete
directory path to the file.

For example: \newline
\begin{IPEMCodeEnvironment}
FileName = 'music.wav'\newline FilePath =
'E:$\backslash$Tests$\backslash$Sounds'
\end{IPEMCodeEnvironment}


\subsection{Empty and not specified input arguments}

\subsubsection*{Empty input arguments}
Sometimes, input arguments can be left empty, which means you can
specify the empty matrix [] for them. If possible, a default value
will be used in such a case.

\subsubsection*{Optional input arguments}
Arguments at the end of the input arguments list do not always
have to be specified, which means you do not have to type a value
for them at all. This again is mostly used for allowing the
caller to use default values.


\subsection{PlotFlag}
Always used as inPlotFlag, this argument specifies whether or not
plots should be generated by the function.

The following convention is used: if it is zero, no plots are
generated, if it is non-zero then plots \textsl{are} generated.
