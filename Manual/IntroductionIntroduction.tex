% --------------------------------------------------------------------------------
% Introduction: Introduction
% --------------------------------------------------------------------------------

\chapter*{\IPEMFullTitle}
\hypertarget{Chapter:IntroductionIntroduction}{}

\section*{Our motivation}
The motivation for developing a toolbox for perception-based
music analysis starts from the observation that much of the world
music production (seen over different cultures and time periods)
has no score representation, and that if a score representation is
available, it still represents but a small percentage of what
musical communication really is about. A sonological analysis of
musical sounds may provide steps towards a better understanding
of the components that contribute to musical information
processing but is also insufficient in view of how humans deal
with musical signals.

Up to the 19th century, the score was indeed the main carrier for
musical representation. And as a consequence musicology had a
focus on pitch and duration analysis, the main musical parameters
to encode in scores. But in the 20th century, thanks to the
breakthrough of electronic means for music production and
analysis, more powerful devices allowed for the control of
physical aspects of music. New methods for musical sound analysis
have been developed and the focus on pitch has been extended
towards timbre and sound effects. By the end of the 20th century,
digital technology laid the foundation for computer modeling of
perception and cognition, thereby offering a global human
information processing approach to music analysis. In this
approach aspects of musical communication that are based on pitch,
timbre, texture, expression, spatialization etc... can be studied
in close connection with empirical based scientific disciplines.

Technology has revolutionized our conception of what kind of music
research will be possible in the 21st century. But it is the task
of musicologists, in collaboration with other scientists, such as
engineers, psychologists, and brain scientists, to make these
dreams more concrete. Our basic motivation for developing this
toolbox is a \emph{pragmatic} one: if the musicology aims at
understanding music as a social and cultural phenomenon embedded
in the physical world and mediated through human faculties of
perception, cognition, and processing of expressive communication,
then new tools must be developed that allow a fully integrated
approach.

\section*{Our aims}
Our aim is to provide a foundation of music analysis in terms of
human perception. We start from sound and take human perception as
the basis for musical feature extraction and higher-level
conceptualization and representation. To achieve our goals we use
computer modeling of the human auditory system. These models are
quite complex, and there is indeed a thorough need for
higher-level working tools and companion manuals that focus on
music. Well, this is exactly what we offer: a toolbox for
perception-based music analysis within the modern laboratory
environment of MATLAB.

\section*{Our approach}
The approach advocated here is both ecological and naturalistic.
\emph{Ecological} means that the human information processing
system \emph{and} the musical environment are considered as a
global unity in which musical content is an emerging outcome of an
interactive process. \emph{Naturalistic} means that methods from
the natural sciences are used to study phenomena of the musical
mind. In particular, we refer to computer modeling and
experimenting (see \citeA{LemanSchneider:97}, and
\citeA{incollection:Semiotics:Leman:1999c} for more details about
the historical, epistemological and methodological foundations).

\section*{The content of this document}
The toolbox provides a set of functions that allow researchers to
deal with different aspects of feature extraction in perception
(i.e. chord and tonality, pitch, roughness, onset detection, beat
and meter extraction, ...) but we also offer a global concept, a
\emph{philosophy}, say, of how to deal with perception-based music
analysis. In order to make clear what we mean by this we present
the main concepts and offer some elaborate demonstrations of how
the toolbox has been used in our research. The reference manual
gives an overview of all available functions, which are in fact
the tools you can work with.

\section*{Our audience}
The IPEM Toolbox is useful for researchers interested in content
analysis and automatic extraction and description of musical
features. Students may use the toolbox for acquiring practical
knowledge in perception-based auditory modeling. We encourage
cooperation and solicit feedback.

\section*{Our policy}
Users are kindly requested to send their comments to
\IPEMMailSupport. Users can also send us a request to add their
own modules, scripts or demos. We will only consider fully
documented and working code, however (see the
\hyperlink{Chapter:FuncRef}{Function Reference} for more details
on what we see as "good coding behaviour").\\

\section*{Acknowledgement}
The toolbox for perception-based music analysis (\IPEMVersion) has
been developed at IPEM (Institute for Psychoacoustics and
Electronic Music) research center of the Department of Musicology
at the Ghent University, with support from the Research Council of
the University (BOF) and the Fund for Scientific Research of
Flanders (FWO). The promoter of this ongoing project is Marc
Leman.

\section*{Thanks}
The authors would like to thank Dirk Moelants and Francesco
Carreras for testing this toolbox and giving valuable comments
during the course of its development.
