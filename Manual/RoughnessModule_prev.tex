% --------------------------------------------------------------------------------
\newpage
\section{Roughness Module}
% --------------------------------------------------------------------------------

% Make general target
\hypertarget{Concepts:RoughnessModule}{}

% Make target for following functions:
\hypertarget{Concepts:IPEMRoughnessFFT}{}

\subsection{Introductory description}
% --------------------------------------------------------------------------------

The Roughness Module (RM) calculates the roughness (or
equivalently: the sensory dissonance) of a sound. Roughness is
considered to be a sensory process highly related to texture
perception. Hence, in our global chart of image transformation
modules it is localized in the top left (texture and sensory
based) section of figure \ref{Fig:ModulesRM}.
\begin{figure}[h]
    \centering
    \includegraphics[width=\textwidth]{Graphics/ModulesRM}
    \caption{Chart of image transformation modules, with RM highlighted}
    \label{Fig:ModulesRM}
\end{figure}

The module takes sound as input (fig. \ref{Fig:RMModule}) and
produces a roughness estimation. The estimation should be
considered an inference.
\begin{figure}[h]
    \centering
    \includegraphics[width=\IPEMDefaultFigureWidth]{Graphics/RMModule}
    \caption{Image Transformation Process: Roughness Module}
    \label{Fig:RMModule}
\end{figure}
However, the module offers more than just an inference. The
visualization and calculation method of this module is based on
\citeA{inproceedings:Leman:DAFx:2000} where roughness is
defined as the energy of the relevant beating frequencies in the
auditory channels. The model is called the synchronization index
model of roughness and it is based on phase-locking to frequencies
that are present in the neural patterns. The synchronization index
model allows a straightforward visualization of the energy
components underlying roughness, in particular (i) with respect to
auditory channels, and (ii) with respect to phase-locking
synchronization (=the synchronization index for the relevant
beating frequencies on a frequency scale). That's why it is more
than just a inference. It assumes that neurons somehow extract the
energy of the beating frequencies and form internal images on
which the inference is based.

But what are beating frequencies? Take an amplitude modulated
sound with carrier frequency $f_c$ and modulation frequency
$f_m$. Such a signal has only three frequencies in the spectrum,
in particular $f_c$, $f_c-f_m$, and $f_c+f_m$. The beating
frequency is $f_m$. Now, in the auditory system, these frequencies
are introduced as effective beating frequencies into the spectrum
of the neural rate-code patterns. This is due to wave
rectification in the cochlea where the lower part of the modulated
signal is cut off, and as a result new frequencies are introduced
of which the most important ones correspond with the beating
frequency $f_m$ and its multiples. As a matter of fact, neurons
may synchronize with these frequencies provided that they fall in
the frequency range where synchronization is physiologically
possible. This mechanism forms a physiological basis for the
detection of beats and hence, the sensation of roughness.

According to \citeA[p.~520-521]{Javel:88}, the synchronization
index represents the degree of phase-locking to a particular
frequency that is present in the neural pattern. The index can be
mathematically expressed as the (normalized) Fourier Series
coefficient at the frequency of interest. Using this concept, the
degree of roughness can be defined as the sum of the normalized
magnitudes or 'energies' of the relevant beating frequencies in
the Fourier spectrum.


\subsection{Functional-logical description}
% --------------------------------------------------------------------------------

Strictly speaking the roughness module realizes a mapping from
auditory nerve images onto roughness values. In a broad sense the
module starts from sound involving:
\begin{eqnarray}
    APM: s(t) \rightarrow \tilde{e}(t)\\
    RM: \tilde{e}(t) \rightarrow r(t)
\end{eqnarray}
where APM is the Auditory Peripheral Module which is used as the
first step, and RM is the mapping from the primary images
$\tilde{e}(t)$ onto the roughness values $r(t)$ (our Roughness
Module in the strict sense). The latter are scalar values which we
can compare with human behavioral outputs.

\subsection{Signal processing description}
% --------------------------------------------------------------------------------

The synchronization index model calculates roughness in terms of
the 'energy' of neural synchronization to the beating frequencies.
The 'energy' refers to a quantity which we derive from the
magnitude spectrum. Since the beating frequencies are contained
in the lower spectral area of the neuronal pattern $d_c$, the
spectral part we are interested in is defined as:
\begin{equation}
  B(t,f,c) ~=~ F(f,c)D(t,f,c)
\end{equation}
where $F(f,c)$ is a filter whose spectrum is depending on the
channel $c$. The short-term spectrum of the neural
synchronization in channel $c$ is:
\begin{equation}\label{E1}
  D(t,f,c)~=~\int_{t'=-\infty}^{+\infty} e(t,c)w(t'-t)e^{-j2\pi
  ft'}~dt'
\end{equation}
where $e(t,c)$ is the neural pattern in channel $c$, and $w(t'-t)$
is a (hamming) window. This formula calculates the amount of
synchronization for each frequency.  The \emph{magnitude}
spectrum is then defined as $\left|~D(t,f,c)~\right|$, and the
\emph{phase} spectrum as $<D(t,f,c)$.

In order to be able to reproduce the psychoacoustical data with
our inferences, the filters $F(f,c)$ should be more narrow at
auditory channels where the center frequency is below 800 Hz. And
the filters should be attenuated for high center frequencies as
well. In general, however, we can say that $B(t,f,c)$ represents
the \emph{spectrum} of the neural synchronization to the beating
frequencies in channel $c$. The synchronization index of the
beating frequencies is given by:
\begin{equation}
  I(t,f,c) ~=~ \left |~ \frac{B(t,f,c)}{D(t,0,c)}~\right |
\end{equation}
where $D(t,0,c)$ is the DC-component of the whole signal. We now
assume that roughness is related to the 'energy' of this
normalized Fourier transform.

In particular, roughness is calculated in the individual channels
and the total roughness is the sum of the channel roughnesses.
Given the above discussion, it will be possible to visualize the
contribution of the synchronization energy along the axis of the
auditory channels, as well as along the axis of the beating
frequencies. The short-term \emph{'energy' spectrum} of the neural
synchronization to beating frequencies in a particular auditory
channel $c$ is defined as:
\begin{equation}\label{Rspectrum}
\textbf{B}(t,f,c) ~=~ I(t,f,c) ~^\alpha
\end{equation}
where $\alpha$ is a parameter which can be related to the power
law mentioned in Sect.~\ref{ExperimentalEvidence} ($1<\alpha <2$).
%\item
% the energy spectrum of the neural synchronization to
% beating frequencies over all auditory
% channels is:
%\begin{equation}\label{R2}
%\textbf{B}(t,f) ~=~ \sum_{c=1}^{C} \textbf{B}(t,f,c)~ w_c
%\end{equation}
%where $w_c$ is the weighting factor defined for each channel.
%\item
% the overall energy of the neural synchronization to beating frequencies in one channel is:
% \begin{equation}\label{R3}
%E_\textbf{B}(t,c) ~=~ \int_{f=-\infty}^{+\infty}
%\textbf{B}(t,f,c)~df
%\end{equation}
%\end{itemize}
We then obtain the following relationships for the calculation of
roughness:
\begin{equation}\label{R5}
\begin{array}{lll}
  R_\textbf{B}(t)&~=~\int \textbf{B}(t,f)~df
                 &~=~\int \sum_{c=1}^{C}
                 \textbf{B}(t,f,c)~df\\\\
                 &~=~\sum_{c=1}^{C} E_\textbf{B}(t,c)
                 &~=~\sum_{c=1}^{C} \int  \textbf{B}(t,f,c)~df\\
\end{array}
\end{equation}
This expression implies a proper vector-based visualization, one
along the axis of auditory channels and one along the axis of the
(beating) frequencies.

Given the general concept as described above, the implementation
requires a specification of the filters $F(f,c)$. In the present
modeling, we choose elegance as an important criterium to
describe the filters mathematically.
%The form of the filter
%$F'(f,c)$ has first been defined according to the following
%formula's:
%\begin{equation}
%F'(f,c)~=~e^{-8 \frac{f}{f_B}} \left [~1 - cos(2 \pi
%\frac{f}{10f_B}) ~\right ] w(c)
%\end{equation}
%where $f_B$ is the frequency range of the filter and $f$ is
%running from 1 to $f_B$, $w(c)$ is a weight depending on the
%channel $c$. The weight takes into account a linear decrease of
%the impact of the filter depending on the auditory channel number:
%\begin{equation}
%w(c)~=~1-\frac{0.55c}{C})
%\end{equation}
%with c running from 1 to C (=40). The frequency range $f_B$ is
%defined such that it is narrow for the lower auditory channels
%below 800 Hz, broad at about 1000 Hz and again slightly more
%narrow at frequencies higher than 1500 Hz.
%
%A sigmoid function has been defined according to
%%(fig.~\ref{Sigmoid}):
%\begin{equation}\label{Sigmoid}
%S~=~ \frac{(\frac{c}{C})^{.2}} {0.04 + (\frac{c}{C})^{2.45} -
%0.007c}
%\end{equation}
%The sigmoid function is used to define the frequency range as:
%\begin{equation}
%f_B(c) = 10 + (300 * \frac{S}{S_{max}} );
%\end{equation}
%It means that the frequency range of the filter in the lowest
%auditory channel is 10 Hz, and that this range is at most 310 Hz
%(around 1000 Hz). The values of S are normalized so that the
%maximum value of $\frac{S}{S_{max}}$ is equal to one.
%
%%\begin{figure}
%%  \centering
%%  \includegraphics{figures/Sigmoid}
%%  \caption{Sigmoid function}\label{Sigmoid}
%%\end{figure}
%Given the set of filters $F(f,c)$, the sigmoid function
%(\ref{Sigmoid}) has also been used to define where the maximum of
%the filter should be located. Data indicate that the maximum
%around 1000 Hz is at about 70 Hz. It is lower for low auditory
%channels.
%\begin{equation}
%f_M(c) = 20 + (52 * \frac{S}{S_{max}} );
%\end{equation}
%This function starts at 20 Hz and the maximum is obtained at 72
%Hz.
%
%The precise location of the filter shape $F'(f,c)$ on the
%frequency axis is then determined by placing its maximum at
%$f_M(c)$.
The proposed set of filters $F(f,c)$ are shown in figure
\ref{Fig:Filters}. Small changes in parameters do not have a
dramatic effect on the performance of the model.

\begin{figure}[ht]
    \centering
    \includegraphics[width=\IPEMDefaultFigureWidth]{Graphics/Filters}
    \caption{Filters become more narrow in the lower auditory channels.}
    \label{Fig:Filters}
\end{figure}


%Figure \ref{Filters} shows an array of such filters.
%\begin{figure}
%  \centering
%  \includegraphics{figures/Filters}
%  \caption{Filters become more narrow in the lower auditory channels}\label{Filters}
%\end{figure}


\subsection{Implementation}
% --------------------------------------------------------------------------------

\begin{tabularx}{\linewidth}{llX}
\hyperlink{FuncRef:IPEMRoughnessFFT}{IPEMRoughnessFFT} & - & Calculates roughness using synchronization index model\\
\end{tabularx}


\subsection{Examples}
% --------------------------------------------------------------------------------

The function
\hyperlink{FuncRef:IPEMRoughnessFFT}{IPEMRoughnessFFT}
has a number of input parameters. To calculate the roughness of
the wav-file \emph{SchumannKurioseGeschichte.wav} do:\\

\IPEMCodeExtract{[ANI,ANIFreq,ANIFilterFreqs] = IPEMCalcANIFromFile('SchumannKurioseGeschichte.wav');}\\

\IPEMCodeExtract{IPEMRoughnessFFT(ANI,ANIFreq,ANIFilterFreqs,5,300,0.20,0.02);}\\

Notice that the toolbox functions are linked through the variables
\IPEMCodeExtract{ANI,ANIFreq,ANIFilterFreqs}. These variables are
the output of the Auditory Peripheral Module IPEMCalcANIFromFile,
and (part of) the input to the Roughness Module IPEMRoughnessFFT.
Your resulting figure should be similar
to the one shown in figure \ref{Fig:RMRoughness}.\\

\begin{figure}[h]
    \centering
    \includegraphics[width=\IPEMDefaultFigureWidth]{Graphics/RMRoughness}
    \caption{The top panel shows the energy as distributed over the
    auditory channels, the middle panel shows the energy as
    distributed over the beating frequencies, the lower panel
    shows the roughness (which is the sum of either the top or
    middle panel)}
    \label{Fig:RMRoughness}
\end{figure}

See also \hyperlink{Chapter:ConceptsApplications}{Applications}
for more elaborate examples.\\
