% --------------------------------------------------------------------------------
% Introduction: How to read
% --------------------------------------------------------------------------------

\chapter*{How to read this text?}
\hypertarget{Chapter:HowToRead}{}

Part II introduces the reader to the concepts and worked out
applications of the toolbox. Part III contains the reference
manual. We strongly recommend the lecture of the concepts part, in
particular, the introductory descriptions of the modules, as well
as the associated examples. They will give you a flavor of what
the toolbox can offer in terms of music analysis. The signal
processing sections can be left out for a second reading.

The concepts part also contains an examples section that shows how
to use the very basic toolbox commands. Please take into account
that we don't present an exhaustive tutorial of all functions
described in the reference manual!

If you want to have an idea of how to apply the toolbox for
somewhat larger projects in music analysis, you can have a look at
the applications. For the list of all functions, see the reference
manual.

It's true, you can read the first part (concepts and applications)
without using the toolbox. But we strongly recommend to use the
toolbox from the very beginning. Please, go to
\hyperlink{ReferenceManual:GeneralInformation}{Reference Manual,
General Information}, and install the toolbox... and don't forget
to read the conditions/disclaimer section!
