% --------------------------------------------------------------------------------
\newpage
\section{Stimulus-Driven Inference Module}
% --------------------------------------------------------------------------------

The Stimulus-Driven Inference Module (SDIM) realizes a mapping of two images onto a
scale:
\begin{equation}
SDIM: \tilde{p}_{T1} \bigotimes \tilde{p}_{T2} \rightarrow [-1 ... 1]
\end{equation}
 where $T1$ and $T2$ are standing for the respective echos and $\bigotimes$
 represents the
calculation of a correlation coefficient. Two types of stimulus-driven inference are
considered, called {\sl Method I} and {\sl Method II}. They introduce two inference
strategies connected to two different short-term memory models (see discussion):
\begin{itemize}
\item
    In Method I, a local image at the end of the probe tone is fixed, and this image is
    compared with running images over the whole stimulus. This can be formally expressed
    as:
    \begin{displaymath}
    SDIM_{Method I}: \tilde{p}_{T1}(\tau) \bigotimes \tilde{p}_{T2}(t) \rightarrow [-1 ...
    1]
    \end{displaymath}
    where $\tau$ represents a fixed time instance, typically situated at the end of the
    probe tone, and $t$ represents the running time.
\item
    In Method II, the local images are compared with the global images at run-time. The
    formal expression is:
    \begin{displaymath}
    SDIM_{Method II}: \tilde{p}_{T1}(t) \bigotimes \tilde{p}_{T2}(t) \rightarrow [-1 ...
    1]
    \end{displaymath}
    where $t$ represents the running time for both the local and global images.
\end{itemize}

Method I and Method II can be illustrated with the raising
diatonic C major scale. Given this context-defining sequence, and
12 different probe tones, 12 inferences will be inferred. Each
context-defining sequence plus probe tone is called a {\sl trial}.
The sound of each trial has been processed, first, with APM, PCM,
and EMM. The echo for local and global images is set to $T=0.1$,
and $T=1.5$, respectively. In Fig.~\ref{IM}, a trial is shown
using probe tone $c$). For each trial, the stimulus-driven
inference is calculated as follows:

\begin{itemize}
\item
    Using Method I, a fixed local image is taken at the end of the stimulus and this image
    is correlated with the running global images. Figure \ref{scaleMethodI} shows the
    stimulus-driven inferences for the 12 trials. At each time step, this figure shows a
    correlation coefficient for twelve probes. What is seen, therefore, is the profile of
    the stimulus while it develops through time.

    During the development of the profile, different stages can be distinguished. They
    reflect the way a profile is built up during the temporal deployment of the stimulus.
    Each time a tone of the scale is introduced, its corresponding rating pops up. It
    decays when the pitch image corresponding to the note is no longer present.

\item
    Using Method II, the local images are correlated with the global images at run-time.
    Figure \ref{scaleMethodII} shows the stimulus-driven inference for the twelve
    different trials. As a matter of fact, the part of the stimulus that defines the tonal
    context shows the same stimulus-driven inference for all twelve trials. The situation
    changes at the time when the probes are introduced. At that time the local images
    start to get different from the global image in a why that depends on the probe. Since
    all probes are different, one obtains different correlation coefficients for each
    probe.
\end{itemize}

Starting from the time instance where the local image of Method I is fixed (typically
at the end of the trial), the profiles of Method I and Method II are exactly the same
and they remain the same until the complete decay of the local image (not shown on the
figure). The profile at that time is shown in Fig.~\ref{ProfileScale}. The time
shortly after the stimulus presentation corresponds with the time at which the
listener has to make a judgement. Method I has been introduced basically because it
provides interesting information of the change in stimulus with respect to the probes.
The probes ``inspect" so to speak the stimulus during its temporal deployment. Method
II gives the instantaneous ``tension" of a local image with respect to the global
image. The tension is due to the differences in the echo of both images. The longer
echo of the global images smear out the tonal events of the stimulus over a longer
past, hence they contain more residues of past tonal events. Method I and Method II
actually assume a different memory model: an short-term episodic (echoic) memory in
Method I, and a short-time run-time (echoic) memory in Method II, see
Sect.~\ref{Results}.
