\newpage
\sectionmark{General functions}
~\vfill
\begin{center}
   \section*{General functions}
   \subsection*{}
\end{center}
\vfill
\addcontentsline{toc}{section}{General functions}

\newpage
\section*{\hyperlink{Concepts:IPEMAMTone}{IPEMAMTone}\sectionmark{IPEMAMTone}}
\addcontentsline{toc}{subsection}{IPEMAMTone}
\hypertarget{FuncRef:IPEMAMTone}{}

\textbf{Usage:}
\begin{verbatim}  [outSignal] = IPEMAMTone(inCarrierFreq,inModulationFreq,inModulationDepth,...
                           inDuration,indBLevel,inSampleFreq,inFadeInOut)

\end{verbatim}
\textbf{Description:}
\begin{verbatim}  Generates an amplitude modulated tone according to:
    s(t) = (1 + inModulationDepth*sin(2*pi*inModulationFreq*t))
           *sin(2*pi*inCarrierFreq*t)

\end{verbatim}
\textbf{Input arguments:}
\begin{verbatim}  inCarrierFreq = carrier frequency (in Hz)
  inModulationFreq = modulation frequency (in Hz)
  inModulationDepth = modulation depth
  inDuration = wanted duration (in s)
  indBLevel = wanted dB level (in dB)
              if empty or not specified, -20 dB is used by default
  inSampleFreq = wanted sample frequency (in Hz)
                 if empty or not specified, 22050 is used by default
  inFadeInOut = duration of linear fade in and out of the sound signal (in s)
                given as [FadeIn FadeOut] or as a scalar (FadeIn == FadeOut)
                if empty or not specified, 0.010 is used by default

\end{verbatim}
\textbf{Output:}
\begin{verbatim}  outSignal = generated sound signal

\end{verbatim}
\textbf{Example:}
\begin{verbatim}  s = IPEMAMTone(1000,70,0.65,0.200,-20,44100,0.010);

\end{verbatim}
\textbf{Authors:}
\begin{verbatim}  Koen Tanghe - 20040323
\end{verbatim}


\newpage
\section*{\hyperlink{Concepts:IPEMAdaptLevel}{IPEMAdaptLevel}\sectionmark{IPEMAdaptLevel}}
\addcontentsline{toc}{subsection}{IPEMAdaptLevel}
\hypertarget{FuncRef:IPEMAdaptLevel}{}

\textbf{Usage:}
\begin{verbatim}  outSignal = IPEMAdaptLevel (inSignal,indB)

\end{verbatim}
\textbf{Description:}
\begin{verbatim}  This function adapts the RMS power level of the (multi-channel) signal
  to the specified dB level. Channels are represented by rows.

\end{verbatim}
\textbf{Input arguments:}
\begin{verbatim}  inSignal = the input signal (each row represents a channel)
  indB = wanted level (0 is maximum level without clipping, -20 is reasonable)

\end{verbatim}
\textbf{Output:}
\begin{verbatim}  outSignal = the adapted signal

\end{verbatim}
\textbf{Remarks:}
\begin{verbatim}  The reference value of 0 dB is the level of a square wave with amplitude 1
  Thus, a sine wave with amplitude 1 yields -3.01 dB.

\end{verbatim}
\textbf{Example:}
\begin{verbatim}  Signal = IPEMAdaptLevel(Signal,-6);

\end{verbatim}
\textbf{Authors:}
\begin{verbatim}  Marc Leman
  Koen Tanghe - 20000626
\end{verbatim}


\newpage
\section*{\hyperlink{Concepts:IPEMAnimateSlices}{IPEMAnimateSlices}\sectionmark{IPEMAnimateSlices}}
\addcontentsline{toc}{subsection}{IPEMAnimateSlices}
\hypertarget{FuncRef:IPEMAnimateSlices}{}

\textbf{Usage:}
\begin{verbatim}  IPEMAnimateSlices(inSignal,inSampleFreq,XAxisLabel,YAxisLabel,...
                    inXData,inXRange,inYRange,inTimeRange,...
                    inTimeStep,inAnimationInterval,inScales,inGraphicsHandle)

\end{verbatim}
\textbf{Description:}
\begin{verbatim}  Displays animated plots of a 2D matrix where each column represents a slice
  of data at a certain moment in time.
  Could be used as an alternative to a surf plot...

\end{verbatim}
\textbf{Input arguments:}
\begin{verbatim}  inSignal = 2D vector in which each column contains a slice of data
  inSampleFreq = sample frequency of inSignal (in Hz)
  inXAxisLabel = label to display on the X axis of the plot
                 if empty or not specified '' is used by default
  inYAxisLabel = label to display on the Y axis of the plot
                 if empty or not specified '' is used by default
  inXData = vector containing the values that correspond with each row of the
            2D vector
            if empty or not specified, 1:size(inSignal,1) is used by default
  inXRange = X range for which values need to be displayed in each plot
             specified as a vector [LowXValue HighXValue]
             if empty or not specified, [min(inXData) max(XData)] is used
             by default
  inYRange = Y range for which values will be visible in the plot, specified
             as a vector [MinYValue MaxYValue]
             if empty or not specified, [min(min(inSignal)) max(max(inSignal))]
             is used by default
  inTimeRange = time range that should be animated, specified as a vector
                [StartTime EndTime] (in s)
                if empty or not specified, the entire signal is animated
                by default (so: [0 (size(inSignal,2)-1)/inSampleFreq])
  inTimeStep = time step to use for stepping through the animated data (in s)
               if empty or not specified, 1/inSampleFreq is used by default
  inAnimationInterval = time to wait between two successive plots in
                        the animation (in s)
                        if -1, real-time is used
                        if empty or not specified, 0 is used by default
  inScales = 2 element cell vector specifying the type of scale to use for the
             X and Y axis specified as {XType YType}, where XType and YType can
             be either 'linear' or 'log'
             if empty or not specified, {'linear' 'linear'} is used by default
  inGraphicsHandle = handle of a subplot or a figure in which the animation
                     should be shown
                     if empty or not specified, a new subplot in a new figure
                     will be used by default

\end{verbatim}
\textbf{Remarks:}
\begin{verbatim}  Keys that can be pressed in the figure while the plot is animated:
  '+' = forward
  '-' = backward
  ' ' = pause
  'x' = exit

\end{verbatim}
\textbf{Example:}
\begin{verbatim}  [s,fs] = IPEMReadSoundFile('schum1.wav');
  [S,T,F] = IPEMCalcSpectrogram(s,fs,0.040,0.010);
  IPEMAnimateSlices(abs(S),1/T(2),'Frequency (in Hz)','Amplitude',F);

\end{verbatim}
\textbf{Authors:}
\begin{verbatim}  Koen Tanghe - 20021021
\end{verbatim}


\newpage
\section*{\hyperlink{Concepts:IPEMBatchExtractSoundFragment}{IPEMBatchExtractSoundFragment}\sectionmark{IPEMBatchExtractSoundFragment}}
\addcontentsline{toc}{subsection}{IPEMBatchExtractSoundFragment}
\hypertarget{FuncRef:IPEMBatchExtractSoundFragment}{}

\textbf{Usage:}
\begin{verbatim}  IPEMBatchExtractSoundFragment(inFragment,inInputDirectory,inFilePattern,...
                                inOutputDirectory,inNewNameFormat,...
                                inCreateDir,inFadeInOutTime,...
                                inPreferNegativeTimes)

\end{verbatim}
\textbf{Description:}
\begin{verbatim}  Extracts a fragment from all sound files in a specific directory matching 
  the specified wildcard pattern, and saves each fragment to the specified 
  output directory appending a suffix.

\end{verbatim}
\textbf{Input arguments:}
\begin{verbatim}  inFragment = Fragment that should be extracted.
               Specified as [StartTime EndTime] where both elements are in s.
               If StartTime < 0, it is referenced from the end of file.
               If EndTime <= 0, it is referenced from the end of the file.
  inInputDirectory = Directory where the sound files are located.
  inFilePattern = Wildcard pattern for the files that should be processed.
                  If empty or not specified, '*.wav' is used by default.
  inOutputDirectory = Directory where the extracted sound fragments should be 
                      written. If empty or not specified, inInputDirectory is 
                      used by default.
  inNewNameFormat = Format string that should be used for the file names of
                    the new sound fragments. This must be a format string 
                    where the first type specifier is a %s (will be replaced 
                    by the original base file name), and the second and third 
                    are a %f (will be replaced by the start resp. end time
                    in seconds). 
                    If empty or not specified, '%s_%gs_%gs" is used by default.
  inCreateDir = If 1, the output directory is created if it doesn't exist yet.
                If empty or not specified, 1 is used by default.
  inFadeInOutTime = Time over which the beginning and end should be faded in
                    repectively out (in s).
                    If empty or not specified, 0.010 s is used by default.
  inPreferNegativeTimes = If 0, negative start or end times will be shown as 
                          the actually used times in the output file names.
                          Otherwise, negative times will be kept.
                          If empty or not specified, 1 is used by default.

\end{verbatim}
\textbf{Example:}
\begin{verbatim}  IPEMBatchExtractSoundFragment([60 90],'D:\Temp\','*.wav','D:\Temp2\');

\end{verbatim}
\textbf{Authors:}
\begin{verbatim}  Koen Tanghe - 20040702
\end{verbatim}


\newpage
\section*{\hyperlink{Concepts:IPEMBellShape}{IPEMBellShape}\sectionmark{IPEMBellShape}}
\addcontentsline{toc}{subsection}{IPEMBellShape}
\hypertarget{FuncRef:IPEMBellShape}{}

\textbf{Usage:}
\begin{verbatim}  outY = IPEMBellShape (inX,inCenter,inWidth,inPeak)

\end{verbatim}
\textbf{Description:}
\begin{verbatim}  This function generates a bell shaped curve.

\end{verbatim}
\textbf{Input arguments:}
\begin{verbatim}  inX = input data points
  inCenter = center of the curve
  inWidth = width of the curve:
            the value at (inCenter + inWidth) is 10% of inPeak 
  inPeak = maximum value (at the center of the curve)

\end{verbatim}
\textbf{Output:}
\begin{verbatim}  outY = curve values for the input

\end{verbatim}
\textbf{Example:}
\begin{verbatim}  Y = IPEMBellShape(0:1000,500,100,1);

\end{verbatim}
\textbf{Authors:}
\begin{verbatim}  Koen Tanghe - 20000208
\end{verbatim}


\newpage
\section*{\hyperlink{Concepts:IPEMBlockDC}{IPEMBlockDC}\sectionmark{IPEMBlockDC}}
\addcontentsline{toc}{subsection}{IPEMBlockDC}
\hypertarget{FuncRef:IPEMBlockDC}{}

\textbf{Usage:}
\begin{verbatim}  [outSignal,outSignalFreq] = ...
    IPEMBlockDC(inSignal,inSignalFreq,inCutoffFreq,inPlotFlag)

\end{verbatim}
\textbf{Description:}
\begin{verbatim}  Blocks DC (very low frequencies).
Based on the formula:
	y(n) = g * ( x(n) - x(n-1) ) + R * y(n-1)
where
	R = 1 - (2*PI*Fc/Fs)
      g = (1 + R)/2 (for gain correction)
	Fc = cutoff frequency in Hz (-3 dB point)
	Fs = sample frequency in Hz

\end{verbatim}
\textbf{Input arguments:}
\begin{verbatim}  inSignal = one dimensional input signal
  inSignalFreq = sample frequency of inSignal (in Hz)
  inCutoffFreq = -3 dB cutoff frequency for HP filter (in Hz)
  inPlotFlag = if non-zero, plots are generated
               if not specified or empty, 0 is used by default

\end{verbatim}
\textbf{Output:}
\begin{verbatim}  outSignal = Processed signal.
  outSignalFreq = Sample frequency of output signal (same as input) (in Hz).

\end{verbatim}
\textbf{Example:}
\begin{verbatim}  [s2,fs2] = IPEMBlockDC(s,fs,50,1);

\end{verbatim}
\textbf{Authors:}
\begin{verbatim}  Koen Tanghe - 20050120
\end{verbatim}


\newpage
\section*{\hyperlink{Concepts:IPEMCalcANI}{IPEMCalcANI}\sectionmark{IPEMCalcANI}}
\addcontentsline{toc}{subsection}{IPEMCalcANI}
\hypertarget{FuncRef:IPEMCalcANI}{}

\textbf{Usage:}
\begin{verbatim}  [outANI,outANIFreq,outANIFilterFreqs] = ...
    IPEMCalcANI (inSignal,inSampleFreq,inAuditoryModelPath,...
                 inPlotFlag,inDownsamplingFactor,...
                 inNumOfChannels,inFirstCBU,inCBUStep)

\end{verbatim}
\textbf{Description:}
\begin{verbatim}  This function calculates the auditory nerve image for the given signal.

\end{verbatim}
\textbf{Input arguments:}
\begin{verbatim}  inSignal = the sound signal to be processed
  inSampleFreq = the sample frequency of the input signal (in Hz)
  inAuditoryModelPath = path to the working directory for the auditory model
                        if empty or not specified, IPEMRootDir('code')\Temp
                        is used by default
  inPlotFlag = if non-zero, plots the ANI
               if empty or not specified, 0 is used by default
  inDownsamplingFactor = the integer factor by which the outcome of the
                         auditory model is downsampled
                         (use 1 for no downsampling)
                         if empty or not specified, 4 is used by default
  inNumOfChannels = number of channels to use
                    if empty or not specified, 40 is used by default
  inFirstCBU = frequency of first channel (in critical band units)
               if empty or not specified, 2.0 is used by default
  inCBUStep = frequency difference between channels (in cbu)
              if empty or not specified, 0.5 is used by default

\end{verbatim}
\textbf{Output:}
\begin{verbatim}  outANI = a matrix of size [N M] representing the auditory nerve image,
           where N is the number of channels (currently 40) and
                 M is the number of samples
  outANIFreq = sample freq of ANI (in Hz)
  outANIFilterFreqs = center frequencies used by the auditory model (in Hz)

\end{verbatim}
\textbf{Remarks:}
\begin{verbatim}  The outcome of the auditory model normally has a sample frequency of
  11025 Hz, but for most calculations this can be downsampled to a lower
  value (11025/4 is the default).

\end{verbatim}
\textbf{Example:}
\begin{verbatim}  [ANI,ANIFreq,ANIFilterFreqs] = IPEMCalcANI(Signal,SampleFreq,[],1);

\end{verbatim}
\textbf{Authors:}
\begin{verbatim}  Koen Tanghe - 20010129
\end{verbatim}


\newpage
\section*{\hyperlink{Concepts:IPEMCalcANIFromFile}{IPEMCalcANIFromFile}\sectionmark{IPEMCalcANIFromFile}}
\addcontentsline{toc}{subsection}{IPEMCalcANIFromFile}
\hypertarget{FuncRef:IPEMCalcANIFromFile}{}

\textbf{Usage:}
\begin{verbatim}  [outANI,outANIFreq,outANIFilterFreqs] = ...
    IPEMCalcANIFromFile (inFileName,inFilePath,inAuditoryModelPath,...
                         inPlotFlag,inDownsamplingFactor,...
                         inNumOfChannels,inFirstCBU,inCBUStep)

\end{verbatim}
\textbf{Description:}
\begin{verbatim}  This function calculates the auditory nerve image for the given sound file.

\end{verbatim}
\textbf{Input arguments:}
\begin{verbatim}  inFileName = the sound file to be processed
  inFilePath = the path to the sound file
               if empty or not specified, IPEMRootDir('input')\Sounds
               is used by default
  inAuditoryModelPath = path to the working directory for the auditory model
                        if empty or not specified, IPEMRootDir('code')\Temp
                        is used by default)
  inPlotFlag = if non-zero or empty, plots the ANI
               if empty or not specified, 0 is used by default
  inDownsamplingFactor = the integer factor by which the outcome of the
                         auditory model is downsampled
                         (use 1 for no downsampling)
                         if empty or not specified, 4 is used by default
  inNumOfChannels = number of channels to use
                    if empty or not specified, 40 is used by default
  inFirstCBU = frequency of first channel (in critical band units)
               if empty or not specified, 2.0 is used by default
  inCBUStep = frequency difference between channels (in cbu)
              if empty or not specified, 0.5 is used by default

\end{verbatim}
\textbf{Output:}
\begin{verbatim}  outANI = a matrix of size [N M] representing the auditory nerve images,
           where N is the number of channels (currently 40) and
                 M is the number of samples
  outANIFreq = sample freq of ANI
  outANIFilterFreqs = center frequencies used by the auditory model

\end{verbatim}
\textbf{Remarks:}
\begin{verbatim}  The outcome of the auditory model normally has a sample frequency of
  11025 Hz, but for most calculations this can be downsampled to a lower
  value (11025/4 is the default).

\end{verbatim}
\textbf{Example:}
\begin{verbatim}  [ANI,ANIFreq,ANIFilterFreqs] = IPEMCalcANIFromFile ('music.wav',[],[],1);

\end{verbatim}
\textbf{Authors:}
\begin{verbatim}  Koen Tanghe - 20010129
\end{verbatim}


\newpage
\section*{\hyperlink{Concepts:IPEMCalcCentroid}{IPEMCalcCentroid}\sectionmark{IPEMCalcCentroid}}
\addcontentsline{toc}{subsection}{IPEMCalcCentroid}
\hypertarget{FuncRef:IPEMCalcCentroid}{}

\textbf{Usage:}
\begin{verbatim}  outCentroid = IPEMCalcCentroid (inWeights,inWeightFreq,...
                                  inFrameWidth,inFrameInterval,inDistances,...
                                  inPlotFlag)

\end{verbatim}
\textbf{Description:}
\begin{verbatim}  Calculates the centroid for the given (time-varying) weights and their
  (constant) distances.
  This can be used to calculate the (time-varying) centroid of the spectrum
  of a musical signal.

\end{verbatim}
\textbf{Input arguments:}
\begin{verbatim}  inWeights = a matrix (of size [N M]) in which each column represents weight
              data at a given moment in time
  inWeightFreq = frequency at which the weight data is sampled (in Hz)
  inFrameWidth = width of 1 frame (in s)
  inFrameInterval = interval between successive frames (in s)
  inDistances = a vector of size [N 1], representing distances for each weight
                if empty or not specified, (1:N)' is used
  inPlotFlag = if non-zero, plots the centroid
               if empty or not specified, 0 is used by default

\end{verbatim}
\textbf{Output:}
\begin{verbatim}  outCentroid = the (time-varying) centroid of the weights
  outCentroidFreq = sample frequency for centroid

\end{verbatim}
\textbf{Example:}
\begin{verbatim}  [Centroid,CentroidFreq] = IPEMCalcCentroid(ANI,ANIFreq,0.05,0.01,...
                                             ANIFilterFreqs);

\end{verbatim}
\textbf{Authors:}
\begin{verbatim}  Koen Tanghe - 20010221
\end{verbatim}


\newpage
\section*{\hyperlink{Concepts:IPEMCalcCentroidWidth}{IPEMCalcCentroidWidth}\sectionmark{IPEMCalcCentroidWidth}}
\addcontentsline{toc}{subsection}{IPEMCalcCentroidWidth}
\hypertarget{FuncRef:IPEMCalcCentroidWidth}{}

\textbf{Usage:}
\begin{verbatim}  [outWidth,outWidthFreq] =
    IPEMCalcCentroidWidth(inWeights,inWeightFreq,
                          inFrameWidth,inFrameInterval,
                          inDistances,inCentroid,
                          inPlotFlag)

\end{verbatim}
\textbf{Description:}
\begin{verbatim}  Calculates the weighted difference between the spectral components and the
  (already) calculated centroid of the given multi-channel signal.

\end{verbatim}
\textbf{Input arguments:}
\begin{verbatim}  inWeights = a matrix (of size [N M]) in which each column represents weight
              data at a given moment in time
  inWeightFreq = frequency at which the weight data is sampled (in Hz)
  inFrameWidth = width of 1 frame (in s)
  inFrameInterval = interval between successive frames (in s)
  inDistances = a vector of size [N 1], representing distances for each weight
                if empty or not specified, (1:N)' is used
  inCentroid = this provides the centroid for the data that was
               calculated with IPEMCalcCentroid, using the same parameters
               if empty or not specified, the centroid itself is calculated
               in this function
  inPlotFlag = if non-zero, plots three curves:
               the centroid itself (central line), plus both the centroid + the
               centroid width and the centroid - the centroid width
               if empty or not specified, 0 is used by default

\end{verbatim}
\textbf{Output:}
\begin{verbatim}  outWidth = the calculated width
  outWidthFreq = sample frequency of the width

\end{verbatim}
\textbf{Example:}
\begin{verbatim}  [Width,WidthFreq] = IPEMCalcCentroidWidth(ANI,ANIFreq,0.05,0.01,...
                                            ANIFilterFreqs);

\end{verbatim}
\textbf{Authors:}
\begin{verbatim}  Koen Tanghe - 20010221
\end{verbatim}


\newpage
\section*{\hyperlink{Concepts:IPEMCalcFFT}{IPEMCalcFFT}\sectionmark{IPEMCalcFFT}}
\addcontentsline{toc}{subsection}{IPEMCalcFFT}
\hypertarget{FuncRef:IPEMCalcFFT}{}

\textbf{Usage:}
\begin{verbatim}  [outAmpl,outPhase,outFreqs] = IPEMCalcFFT (inSignal, inSampleFreq, ...
                                             inFFTSize, inPlotFlag, ...
                                             inUseLogScale, inShowPhase)

\end{verbatim}
\textbf{Description:}
\begin{verbatim}  This function calculates (and shows) the amplitude and phase of a real
  signal.

\end{verbatim}
\textbf{Input arguments:}
\begin{verbatim}  inSignal = the signal to be analyzed
  inSampleFreq = the frequency at which the signal was sampled (in Hz)
  inFFTSize = size of the FFT 
              if empty or not specified, the power of two nearest to the full
              length of the signal is used by default
  inPlotFlag = if non-zero, shows FFT plot
               if empty or not specified, 1 is used by default
  inUseLogScale = if non-zero, a logarithmic scale is used for the frequency
                  if empty or not specified, 1 is used by default
  inShowPhase = if non-zero, the phase is shown as well
                if empty or not specified, 0 is used by default

\end{verbatim}
\textbf{Output:}
\begin{verbatim}  outAmpl = amplitude of fft components
  outPhase = phase of fft components
  outFreqs = frequencies used for fft calculation
  
  The indices in these output matrices (of length L) correspond to the
  following frequencies:
    index 1  corresponds to  the DC component
    index i  corresponds to  frequency (i-1)*inSampleFreq/inFFTSize
    index L  corresponds to  the Nyquist frequency (if inFFTSize was even)

\end{verbatim}
\textbf{Remarks:}
\begin{verbatim}  If inFFTSize is smaller than the length of inSignal, inSignal is truncated
  (a warning is issued when this happens).
  If inFFTSize is bigger than the length of inSignal, inSignal is padded with
  zeroes.

\end{verbatim}
\textbf{Example:}
\begin{verbatim}  [Ampl,Phase,Freqs] = IPEMCalcFFT (Signal,SampleFreq,1024,1,1,0);

\end{verbatim}
\textbf{Authors:}
\begin{verbatim}  Koen Tanghe - 20010627
\end{verbatim}


\newpage
\section*{\hyperlink{Concepts:IPEMCalcFlux}{IPEMCalcFlux}\sectionmark{IPEMCalcFlux}}
\addcontentsline{toc}{subsection}{IPEMCalcFlux}
\hypertarget{FuncRef:IPEMCalcFlux}{}

\textbf{Usage:}
\begin{verbatim}  outFlux = IPEMCalcFlux (inSignal,inSampleFreq,inPlotFlag)

\end{verbatim}
\textbf{Description:}
\begin{verbatim}  Calculates the flux of the signal, where the flux is defined as
  the norm of the difference (vector) between the current and the previous
  values of the (multichannel) signal.

\end{verbatim}
\textbf{Input arguments:}
\begin{verbatim}  inSignal = input signal (spectral data, for example)
  inSampleFreq = sample frequency of inSignal (in Hz)
  inPlotFlag = if non-zero, plots are generated
               if not specified, 1 is used by default

\end{verbatim}
\textbf{Output:}
\begin{verbatim}  outFlux = calculated flux

\end{verbatim}
\textbf{Example:}
\begin{verbatim}  Flux = IPEMCalcFlux (Signal,SampleFreq);

\end{verbatim}
\textbf{Authors:}
\begin{verbatim}  Koen Tanghe - 20010906
\end{verbatim}


\newpage
\section*{\hyperlink{Concepts:IPEMCalcMeanAndVariance}{IPEMCalcMeanAndVariance}\sectionmark{IPEMCalcMeanAndVariance}}
\addcontentsline{toc}{subsection}{IPEMCalcMeanAndVariance}
\hypertarget{FuncRef:IPEMCalcMeanAndVariance}{}

\textbf{Usage:}
\begin{verbatim}  [outMean,outVariance,outFreq] = 
    IPEMCalcMeanAndVariance(inSignal,inSampleFreq,
                            inFrameWidth,inFrameInterval,
                            inPlotFlag,inPlotTitle)

\end{verbatim}
\textbf{Description:}
\begin{verbatim}  Calculates 'running' mean and variance of multi-channel signal:
  for each channel, the mean and variance is calculated within succesive
  frames.

\end{verbatim}
\textbf{Input arguments:}
\begin{verbatim}  inSignal = signal to be analyzed
  inSampleFreq = sample frequency of the input signal (in Hz)
  inFrameWidth = width of 1 frame (in s)
  inFrameInterval = interval between successive frames (in s)
  inPlotFlag = if non-zero, plots are generated
               if not specified or empty, 1 is used by default
  inPlotTitle = title for the plot
                if not specified or empty, no title is shown

\end{verbatim}
\textbf{Output:}
\begin{verbatim}  outMean = 'running' mean of inSignal
  outVariance = 'running' variance of inSignal
  outFreq = sample frequency for both outMean and outVariance

\end{verbatim}
\textbf{Example:}
\begin{verbatim}  [Mean,Variance,Freq] = IPEMCalcMeanAndVariance(RMS,RMSFreq,0.05,0.01,1,...
                                                 'RMS');

\end{verbatim}
\textbf{Authors:}
\begin{verbatim}  Koen Tanghe - 20001012
\end{verbatim}


\newpage
\section*{\hyperlink{Concepts:IPEMCalcNoteFrequency}{IPEMCalcNoteFrequency}\sectionmark{IPEMCalcNoteFrequency}}
\addcontentsline{toc}{subsection}{IPEMCalcNoteFrequency}
\hypertarget{FuncRef:IPEMCalcNoteFrequency}{}

\textbf{Usage:}
\begin{verbatim}  outFrequency = 
    IPEMCalcNoteFrequency(inNoteNr,inOctaveNr,
                          inRefNoteNr,inRefOctaveNr,
                          inRefFreq,inNotesPerOctave,
                          inOctaveRatio)

\end{verbatim}
\textbf{Description:}
\begin{verbatim}  Calculates the frequency of a note.
  Supports non-standard tone scales that can be calculated like this:

      Frequency = RefFreq * OctaveRatio^Exponent

      where:

      Exponent = (OctaveNr-RefOctaveNr) + (NoteNr-RefNoteNr)/NotesPerOctave 

\end{verbatim}
\textbf{Input arguments:}
\begin{verbatim}  inNoteNr = the rank number of the note (for example: 1 for C, 2 for C#, ...)
             For the standard octave division (12 notes, ref. A4), this can
             also be a note string like 'A#' (in which case inOctaveNr must
             still be specified), or 'A#4' (in which case inOctaveNr can be
             omitted or left empty). Both sharps (#) and flats (b) can be used.
  inOctaveNr = the octave number of the note
               if empty or not specified, inNoteNr is assumed to contain the
               octave specification 
  inRefNoteNr = the rank number of the reference note
                if empty or not specified, 10 is used by default
  inRefOctaveNr = the octave number of the reference note
                  if empty or not specified, 4 is used by default
  inRefFreq = the frequency of the reference note (in Hz)
              if empty or not specified, 440 Hz is used
  inNotesPerOctave = the number of notes in one octave
                     if empty or not specified, 12 is used by default
  inOctaveRatio = the frequency ratio between two octaves
                  if empty or not specified, 2 is used

\end{verbatim}
\textbf{Output:}
\begin{verbatim}  outFrequency = the frequency for the note (in Hz)

\end{verbatim}
\textbf{Example:}
\begin{verbatim}  Frequency = IPEMCalcNoteFrequency('A#5');

\end{verbatim}
\textbf{Authors:}
\begin{verbatim}  Koen Tanghe - 20000509
\end{verbatim}


\newpage
\section*{\hyperlink{Concepts:IPEMCalcOnsets}{IPEMCalcOnsets}\sectionmark{IPEMCalcOnsets}}
\addcontentsline{toc}{subsection}{IPEMCalcOnsets}
\hypertarget{FuncRef:IPEMCalcOnsets}{}

\textbf{Usage:}
\begin{verbatim}  [outOnsetSignal,outOnsetFreq] = 
    IPEMCalcOnsets(inSignal,inSampleFreq,inPlotFlag)

\end{verbatim}
\textbf{Description:}
\begin{verbatim}  This function calculates the onsets for the given signal using the auditory
  model and an integrate-and-fire neural net layer.

\end{verbatim}
\textbf{Input arguments:}
\begin{verbatim}  inSignal = signal to be processed
  inSampleFreq = sample frequency of input signal (in Hz)
  inPlotFlag = if non-zero, a plot is generated at the end
               if empty or not specified, 0 is used by default

\end{verbatim}
\textbf{Output:}
\begin{verbatim}  outOnsetSignal = signal having a non-zero value for an onset, and zero
                   otherwise (the higher the non-zero value, the more our
                   system is convinced that the onset is really an onset)
  outOnsetFreq = sample frequency of outOnsetSignal (in Hz)

\end{verbatim}
\textbf{Example:}
\begin{verbatim}  [OnsetSignal,OnsetFreq] = IPEMCalcOnsets(s,fs);

\end{verbatim}
\textbf{Authors:}
\begin{verbatim}  Koen Tanghe - 20030327
\end{verbatim}


\newpage
\section*{\hyperlink{Concepts:IPEMCalcOnsetsFromANI}{IPEMCalcOnsetsFromANI}\sectionmark{IPEMCalcOnsetsFromANI}}
\addcontentsline{toc}{subsection}{IPEMCalcOnsetsFromANI}
\hypertarget{FuncRef:IPEMCalcOnsetsFromANI}{}

\textbf{Usage:}
\begin{verbatim}  [outOnsetSignal,outOnsetFreq] =
    IPEMCalcOnsetsFromANI(inANI,inANIFreq,inPlotFlag)

\end{verbatim}
\textbf{Description:}
\begin{verbatim}  This function calculates the onsets for a given signal represented by its 
  auditory nerve image, using an integrate-and-fire neural net layer.

\end{verbatim}
\textbf{Input arguments:}
\begin{verbatim}  inANI = auditory nerve image to be processed
  inANIFreq = sample frequency of auditory nerve image (in Hz)
  inPlotFlag = if non-zero, a plot is generated at the end
               if empty or not specified, 1 is used by default

\end{verbatim}
\textbf{Output:}
\begin{verbatim}  outOnsetSignal = signal having a non-zero value for an onset, and zero
                   otherwise (the higher the non-zero value, the more our
                   system is convinced that the onset is really an onset)

\end{verbatim}
\textbf{Example:}
\begin{verbatim}  [OnsetSignal,OnsetFreq] = IPEMCalcOnsetsFromANI(ANI,ANIFreq);

\end{verbatim}
\textbf{Authors:}
\begin{verbatim}  Koen Tanghe - 20010122
\end{verbatim}


\newpage
\section*{\hyperlink{Concepts:IPEMCalcPeakLevel}{IPEMCalcPeakLevel}\sectionmark{IPEMCalcPeakLevel}}
\addcontentsline{toc}{subsection}{IPEMCalcPeakLevel}
\hypertarget{FuncRef:IPEMCalcPeakLevel}{}

\textbf{Usage:}
\begin{verbatim}  [outPeakSignal,outPeakFreq] = ...
    IPEMCalcPeakLevel(inSignal,inSampleFreq,inFrameWidth,
                      inFrameInterval,inUseAbs,inPlotFlag)

\end{verbatim}
\textbf{Description:}
\begin{verbatim}  This function calculates the running maximum of the given signal:
  a maximum is generated every inFrameInterval seconds over a period of  
  inFrameWidth seconds.

\end{verbatim}
\textbf{Input arguments:}
\begin{verbatim}  inSignal = the input signal (if this is a matrix, peak values are calculated
             for each channel (ie. row) in the signal)
  inSampleFreq = the sample frequency of the input signal (in Hz)
  inFrameWidth = the period over which the maximum is calculated (in s)
  inFrameInterval = the period between two successive frames (in s)
  inUseAbs = if non-zero, abs(inSignal) is used instead of the original input
             if not specified or empty, 1 is used by default
  inPlotFlag = if non-zero, plots are generated
               if not specified or empty, 0 is used by default

\end{verbatim}
\textbf{Output:}
\begin{verbatim}  outPeakSignal = the running maximum value of the signal
  outPeakFreq = the sample frequency of the maximum signal

\end{verbatim}
\textbf{Example:}
\begin{verbatim}  [s,fs] = IPEMReadSoundFile;
  [Max,MaxFreq] = IPEMCalcPeakLevel(s,fs,0.020,0.010,1,1);

\end{verbatim}
\textbf{Authors:}
\begin{verbatim}  Koen Tanghe - 20040323
\end{verbatim}


\newpage
\section*{\hyperlink{Concepts:IPEMCalcRMS}{IPEMCalcRMS}\sectionmark{IPEMCalcRMS}}
\addcontentsline{toc}{subsection}{IPEMCalcRMS}
\hypertarget{FuncRef:IPEMCalcRMS}{}

\textbf{Usage:}
\begin{verbatim}  [outRMSSignal,outRMSFreq] = IPEMCalcRMS (inSignal,inSampleFreq,inFrameWidth,
                                           inFrameInterval,inPlotFlag)

\end{verbatim}
\textbf{Description:}
\begin{verbatim}  This function calculates the running RMS value of the given signal:
  an RMS value is generated every inFrameInterval seconds over a period of  
  inFrameWidth seconds.

\end{verbatim}
\textbf{Input arguments:}
\begin{verbatim}  inSignal = the input signal (if this is a matrix, RMS values are calculated
             for each channel (ie. row) in the signal)
  inSampleFreq = the sample frequency of the input signal (in Hz)
  inFrameWidth = the period over which the RMS is calculated (in s)
  inFrameInterval = the period between two successive frames (in s)
  inPlotFlag = if non-zero, plots are generated
               if not specified or empty, 0 is used by default

\end{verbatim}
\textbf{Output:}
\begin{verbatim}  outRMSSignal = the running RMS value of the signal
  outRMSFreq = the sample frequency of the RMS signal

\end{verbatim}
\textbf{Example:}
\begin{verbatim}  [RMSSignal,RMSFreq] = IPEMCalcRMS(ANI,ANIFreq,0.050,0.010);

\end{verbatim}
\textbf{Authors:}
\begin{verbatim}  Koen Tanghe - 20010221
\end{verbatim}


\newpage
\section*{\hyperlink{Concepts:IPEMCalcRoughnessOfToneComplex}{IPEMCalcRoughnessOfToneComplex}\sectionmark{IPEMCalcRoughnessOfToneComplex}}
\addcontentsline{toc}{subsection}{IPEMCalcRoughnessOfToneComplex}
\hypertarget{FuncRef:IPEMCalcRoughnessOfToneComplex}{}

\textbf{Usage:}
\begin{verbatim}  [outSignal,outSignalFreq,outRoughness,outRoughnessFreq] = ...
    IPEMCalcRoughnessOfToneComplex(inBaseFreq,inOctaveRatio,
                                   inNumOfHarmonics,inDuration,inPlotFlag)

\end{verbatim}
\textbf{Description:}
\begin{verbatim}  Calculates the roughness for a superposition of two complex tones
  having inNumOfHarmonics harmonics:
  - a constant one with fundamental frequency inBaseFreq
  - one with linearly increasing fundamental frequency
    (from inBaseFreq to inBaseFreq*inOctaveRatio)
  Any octave ratio can be specified.

\end{verbatim}
\textbf{Input arguments:}
\begin{verbatim}  inBaseFreq = fundamental frequency for the tone complex (in Hz)
  inOctaveRatio = frequency ratio for the octave
  inNumOfHarmonics = total number of harmonics for each tone complex
  inDuration = duration of the signal (in s)
  inPlotFlag = if non-zero, plots are generated
               if empty or not specified, 1 is used by default

\end{verbatim}
\textbf{Output:}
\begin{verbatim}  outSignal = signal that was analyzed
  outSignalFreq = sample frequency for outSignal (in Hz)
  outRoughness = roughness calculated for the signal
  outRoughnessFreq = sample frequency for outRoughness (in Hz)

\end{verbatim}
\textbf{Example:}
\begin{verbatim}  [s,fs,r,rfreq] = IPEMCalcRoughnessOfToneComplex(440,2,5,5);

\end{verbatim}
\textbf{Authors:}
\begin{verbatim}  Koen Tanghe - 20050120
\end{verbatim}


\newpage
\section*{\hyperlink{Concepts:IPEMCalcRoughnessOverSubparts}{IPEMCalcRoughnessOverSubparts}\sectionmark{IPEMCalcRoughnessOverSubparts}}
\addcontentsline{toc}{subsection}{IPEMCalcRoughnessOverSubparts}
\hypertarget{FuncRef:IPEMCalcRoughnessOverSubparts}{}

\textbf{Usage:}
\begin{verbatim}  [outRoughness,outDiffCents] = 
    IPEMCalcRoughnessOverSubparts(inOriginalFileName,inOriginalFilePath,
                                  inCentsToRaise,inCentsPerStep,
                                  inMixedFileName,inMixedFilePath)

\end{verbatim}
\textbf{Description:}
\begin{verbatim}  Calculates the roughness over the subparts of a sound file that was made by
  CoolEdit2000 using a script file generated by IPEMGeneratePitchShiftScript.
  An array is returned containing a value for the roughness for each
  subpart of the sound file.

\end{verbatim}
\textbf{Input arguments:}
\begin{verbatim}  inOriginalFileName = name of the original sound file that was processed
  inOriginalFilePath = path to the location of the original sound file
                       if empty or not specified, IPEMRootDir('input')\Sounds'
                       is used by default
  inCentsToRaise = maximum number of cents the pitch was raised
                   if empty or not specified, 1200 is used by default
  inCentsPerStep = number of cents the pitch was raised per step
                   if empty or not specified, 100 is used by default
  inMixedFileName = name of the file containing the mix of the original 
                    and the pitch shifted sounds
                    if empty or not specified, the same name as the original
                    file is used, but now ending on '_mixed.wav' instead of
                    just '.wav'
  inMixedFilePath = path to the mixed file
                    if empty or not specified, the same location as the
                    original sound file is used by default

\end{verbatim}
\textbf{Output:}
\begin{verbatim}  outRoughness = the roughness values
  outDiffCents = the corresponding difference in pitch (in cents)

\end{verbatim}
\textbf{Example:}
\begin{verbatim}  [Roughness,DiffCents] = IPEMCalcRoughnessOverSubparts('bottle.wav');

\end{verbatim}
\textbf{Authors:}
\begin{verbatim}  Koen Tanghe - 20011008
\end{verbatim}


\newpage
\section*{\hyperlink{Concepts:IPEMCalcSpectrogram}{IPEMCalcSpectrogram}\sectionmark{IPEMCalcSpectrogram}}
\addcontentsline{toc}{subsection}{IPEMCalcSpectrogram}
\hypertarget{FuncRef:IPEMCalcSpectrogram}{}

\textbf{Usage:}
\begin{verbatim}  [outSpectrogram,outTimes,outFrequencies] = ...
    IPEMCalcSpectrogram(inSignal,inSampleFreq,inFrameSize,inFrameInterval,
                        inWindowType,inFFTSize,inPlotFlag,inPlotTreshold)

\end{verbatim}
\textbf{Description:}
\begin{verbatim}  Utility function for calculating/plotting the spectrogram of a signal.

\end{verbatim}
\textbf{Input arguments:}
\begin{verbatim}  inSignal = signal to analyze
  inSampleFreq = sample frequency of the incoming signal (in Hz)
  inFrameSize = size of one analysis frame (in s)
                if empty or not specified, 0.040 is used by default
  inFrameInterval = interval between successive frames (in s)
                    if empty or not specified, 0.010 is used by default
  inWindowType = if this is a string: type of window to be used
                 supported types are: 'bartlett','blackman','boxcar','hamming',
                                      'hann','hanning','triang'
                 if this is a 1-column matrix: any user defined window of size
                 [Round(inFrameSize*inSampleFreq) 1]
                 if empty or not specified, 'hanning' is used by default
  inFFTSize = size of FFT to be used
              if -1 is specified, a value corresponding to inFrameSize is used
              if empty or not specified, the power of two >= inFrameSize
              in samples is used
  inPlotFlag = if non-zero, a plot is generated (showing dB levels)
               if empty or not specified, 1 is used by default
  inPlotTreshold = values below this level are ignored in the plot (in dB)
                   if empty or not specified, -Inf is used by default

\end{verbatim}
\textbf{Output:}
\begin{verbatim}  outSpectrogram = spectrogram data (amplitude values, so not in dB)
  outTimes = instants at which an analysis was calculated (in s)
  outFrequencies = frequencies used in decomposition (in Hz)

\end{verbatim}
\textbf{Example:}
\begin{verbatim}  [S,T,F] = IPEMCalcSpectrogram(s,fs,0.040,0.010);

\end{verbatim}
\textbf{Authors:}
\begin{verbatim}  Koen Tanghe - 20001130
\end{verbatim}


\newpage
\section*{\hyperlink{Concepts:IPEMCalcZeroCrossingRate}{IPEMCalcZeroCrossingRate}\sectionmark{IPEMCalcZeroCrossingRate}}
\addcontentsline{toc}{subsection}{IPEMCalcZeroCrossingRate}
\hypertarget{FuncRef:IPEMCalcZeroCrossingRate}{}

\textbf{Usage:}
\begin{verbatim}  [outZCR,outZCRFreq] = 
    IPEMCalcZeroCrossingRate(inSignal,inSignalFreq,
                             inFrameWidth,inFrameInterval,
                             inZeroTolerance,inPlotFlag)

\end{verbatim}
\textbf{Description:}
\begin{verbatim}  Calculates the number of zero-crossings per second for successive frames.

\end{verbatim}
\textbf{Input arguments:}
\begin{verbatim}  inSignal = signal to be analyzed
  inSignalFreq = sample frequency of inSignal
  inFrameWidth = width of 1 frame (in s)
  inFrameInterval = interval between successive frames (in s)
  inZeroTolerance = defines a region around zero where zero-crossings are not
                    counted (useful for noisy parts of the signal)
                    if empty or not specified, 0 is used by default
  inPlotFlag = if non-zero, plots the ZCR
               if not specified or empty, 1 is used by default

\end{verbatim}
\textbf{Output:}
\begin{verbatim}  outZCR = number of zero-crossings per second for each frame
  outZCRFreq = sample frequency of outZCR

\end{verbatim}
\textbf{Example:}
\begin{verbatim}  [ZCR,ZCRFreq] = IPEMCalcZeroCrossingRate(s,fs,0.05,0.01,0.001);

\end{verbatim}
\textbf{Authors:}
\begin{verbatim}  Koen Tanghe - 20001012
\end{verbatim}


\newpage
\section*{\hyperlink{Concepts:IPEMCheckVersion}{IPEMCheckVersion}\sectionmark{IPEMCheckVersion}}
\addcontentsline{toc}{subsection}{IPEMCheckVersion}
\hypertarget{FuncRef:IPEMCheckVersion}{}

\textbf{Usage:}
\begin{verbatim}  function [outCompatibility,outCurrentVersion] = ...
    IPEMCheckVersion(inComponent,inReferenceVersion)

\end{verbatim}
\textbf{Description:}
\begin{verbatim}  Checks compatibility using version numbers

\end{verbatim}
\textbf{Input arguments:}
\begin{verbatim}  inComponent = string identifying the component to check
  inReferenceVersion = version number string to compare with

\end{verbatim}
\textbf{Output:}
\begin{verbatim}  outCompatibility = compatibility result:
                     -1 if the current version is lower than the reference
                      0 if the current version is the same as the reference
                     +1 if the current version is higher than the reference
                     empty if the component was not present at all
  outCurrentVersion = current version number string of requested component
                      empty if not present

\end{verbatim}
\textbf{Remarks:}
\begin{verbatim}  For comparison, str2num is used on the version number strings.

\end{verbatim}
\textbf{Example:}
\begin{verbatim}  [Comp,CurrVer] = IPEMCheckVersion('signal','5.0');

\end{verbatim}
\textbf{Authors:}
\begin{verbatim}  Koen Tanghe - 20011204
\end{verbatim}


\newpage
\section*{\hyperlink{Concepts:IPEMClip}{IPEMClip}\sectionmark{IPEMClip}}
\addcontentsline{toc}{subsection}{IPEMClip}
\hypertarget{FuncRef:IPEMClip}{}

\textbf{Usage:}
\begin{verbatim}  outSignal = IPEMClip(inSignal,inLowLimit,inHighLimit,
                       inClipLowTo,inClipHighTo)

\end{verbatim}
\textbf{Description:}
\begin{verbatim}  This function clips the incoming (multi-channel) signal at the given limits.
  Specify empty values for either one of the limits if you don't want clipping
  at that side of the signal range.
  Signal channels are represented by rows.

\end{verbatim}
\textbf{Input arguments:}
\begin{verbatim}  inSignal = the (multi-channel) signal to be clipped
  inLowLimit = specifies the low level clipping value: values lower than this
               are replaced by either the limit itself or inClipLowTo
               if empty, no clipping occurs
  inHighLimit = specifies the high level clipping value: values higher than
                this are replaced by either the limit itself or inClipHighTo
                if empty or not specified, no clipping occurs
  inClipLowTo = if non-empty, this is a replacement value for too low values
                if empty or not specified, inLowLimit is used
  inClipHighTo = if non-empty, this is a replacement value for too high values
                 if empty or not specified, inHighLimit is used

\end{verbatim}
\textbf{Output:}
\begin{verbatim}  outSignal = the clipped signal

\end{verbatim}
\textbf{Example:}
\begin{verbatim}  Signal = IPEMClip(Signal,0.05,1,0,[]);

\end{verbatim}
\textbf{Authors:}
\begin{verbatim}  Koen Tanghe - 20000419
\end{verbatim}


\newpage
\section*{\hyperlink{Concepts:IPEMCombFilter}{IPEMCombFilter}\sectionmark{IPEMCombFilter}}
\addcontentsline{toc}{subsection}{IPEMCombFilter}
\hypertarget{FuncRef:IPEMCombFilter}{}

\textbf{Usage:}
\begin{verbatim}  outSignal = IPEMCombFilter(inSignal,inSampleFreq,inDelay,
                             inAttenuation,inSync)

\end{verbatim}
\textbf{Description:}
\begin{verbatim}  Filters the incoming signal with a comb filter according to the
  following scheme:

       OUT = IN * (1 - inAttenuation * z^(-m) )
       where m = round(inDelay*inSampleFreq)

  In the frequency domain, this produces a comb-like frequency response,
  in which either the peaks (if inSync == 1) or the valleys (if inSync == 0)
  of the comb are at frequencies n.F0 (F0 = 1/inDelay).

\end{verbatim}
\textbf{Input arguments:}
\begin{verbatim}  inSignal = input signal (each row represents a channel)
  inSampleFreq = input signal sample frequency (in Hz)
  inDelay = delay of feedforward branch (in s)
  inAttenuation = attenuation ratio (0 to 1)
  inSync = if 1, the comb's peaks are at frequencies n/delay
           if 0, the comb's valleys are at frequencies n/delay

\end{verbatim}
\textbf{Output:}
\begin{verbatim}  outSignal = the filtered signal

\end{verbatim}
\textbf{Example:}
\begin{verbatim}  scomb = IPEMCombFilter(s,fs,1/500,0.98,1);

\end{verbatim}
\textbf{Authors:}
\begin{verbatim}  Koen Tanghe - 20000208
\end{verbatim}


\newpage
\section*{\hyperlink{Concepts:IPEMContextualityIndex}{IPEMContextualityIndex}\sectionmark{IPEMContextualityIndex}}
\addcontentsline{toc}{subsection}{IPEMContextualityIndex}
\hypertarget{FuncRef:IPEMContextualityIndex}{}

\textbf{Usage:}
\begin{verbatim}  [outChords,outToneCenters,...
   outContextuality1,outContextuality2,outContextuality3] = ...
    IPEMContextualityIndex(inPeriodicityPitch,inSampleFreq,inPeriods,...
                           inSnapShot,inHalfDecayChords,...
                           inHalfDecayToneCenters,inEnlargement,inPlotFlag)

\end{verbatim}
\textbf{Description:}
\begin{verbatim}  This function calculates the contextuality index. Two methods are used:
  - inspection: compares fixed chord images with running chord images
                and running tone center images
  - comparison: compares running chord images with running tone center images

\end{verbatim}
\textbf{Input arguments:}
\begin{verbatim}  inPeriodicityPitch = periodicity pitch image
  inSampleFreq = sample frequency of the input signal (in Hz)
  inPeriods = periods of periodicity analysis (in s)
  inSnapShot = time where the snapshot should be taken (in s)
               if negative, the time is taken at abs(inSnapShot) from the end
               of the sample
               if empty or not specified, the time of the last sample is used
               by default
  inHalfDecayChords = half decay time for leaky integration into chord image
                      if empty or not specified, 0.1 is used by default
  inHalfDecayToneCenters = half decay time for leaky integration into tone
                           center image
                           if empty or not specified, 1.5 is used by default
  inEnlargement = time by which the input signal is enlarged (in s) 
                  if -1, 2*inHalfDecayToneCenters is used
                  if empty or not specified, 0 is used by default
  inPlotFlag = if non-zero, plots are generated
               if empty or not specified, 0 is used by default

\end{verbatim}
\textbf{Output:}
\begin{verbatim}  outChords = local integration of inPeriodicityPitch into chord image
  outToneCenters = global integration of inPeriodicityPitch into tone center
                   image
  outContextuality1 = correspondence between chord taken at snapshot position
                      and running chord image
  outContextuality2 = correspondence between chord taken at snapshot position
                      and running tone center image
  outContextuality3 = correspondence between running chord image and running
                      tone center image

\end{verbatim}
\textbf{Remarks:}
\begin{verbatim}  Sample frequency of output signals is the same as inSampleFreq.

\end{verbatim}
\textbf{Example:}
\begin{verbatim}  [Chords,ToneCenters,Contextuality1,Contextuality2,Contextuality3] = ...
    IPEMContextualityIndex(PP,PPFreq,PPPeriods);

\end{verbatim}
\textbf{Authors:}
\begin{verbatim}  Marc Leman - 19991221 (originally made)
  Koen Tanghe - 20010129 (minor code changes + documentation)
\end{verbatim}


\newpage
\section*{\hyperlink{Concepts:IPEMConvertToAMNoise}{IPEMConvertToAMNoise}\sectionmark{IPEMConvertToAMNoise}}
\addcontentsline{toc}{subsection}{IPEMConvertToAMNoise}
\hypertarget{FuncRef:IPEMConvertToAMNoise}{}

\textbf{Usage:}
\begin{verbatim}  outAMNoise = ...
    IPEMConvertToAMNoise(inSignal,inSampleFreq,inWantedSampleFreq,...
                         inPlaySound,inNoiseBand)

\end{verbatim}
\textbf{Description:}
\begin{verbatim}  Converts a signal to amplitude modulated noise (the given signal is used
  as modulator) and plays back the sound (optionally).
  Can be used for a quick listening to envelope-like signals.

\end{verbatim}
\textbf{Input arguments:}
\begin{verbatim}  inSignal = signal to convert to AM noise
  inSampleFreq = sample frequency of incoming signal (in Hz)
  inWantedSampleFreq = wanted sample frequency for the resulting AM noise
                       (in Hz)
                       if empty or not specified, 22050 is used by default
  inPlaySound = if empty, non-zero or not specified, the AM noise will be
                played back
                otherwise, no sound is produced
  inNoiseBand = two element row vector specifiying the wanted the noise band
                (in Hz)
                if empty or not specified, broad band noise is used by default

\end{verbatim}
\textbf{Output:}
\begin{verbatim}  outAMNoise = the resulting amplitude modulated noise

\end{verbatim}
\textbf{Example:}
\begin{verbatim}  AMNoise = IPEMConvertToAMNoise(RMS,RMSFreq,22050,1);

\end{verbatim}
\textbf{Authors:}
\begin{verbatim}  Koen Tanghe - 20010221
\end{verbatim}


\newpage
\section*{\hyperlink{Concepts:IPEMConvertToClickSound}{IPEMConvertToClickSound}\sectionmark{IPEMConvertToClickSound}}
\addcontentsline{toc}{subsection}{IPEMConvertToClickSound}
\hypertarget{FuncRef:IPEMConvertToClickSound}{}

\textbf{Usage:}
\begin{verbatim}  outClickSignal = IPEMConvertToClickSound(inClickTimes,inWantedSampleFreq,...
                                           inTimeRange,inAmplitudes)

\end{verbatim}
\textbf{Description:}
\begin{verbatim}  Converts a sequence of times into a click sound signal.

\end{verbatim}
\textbf{Input arguments:}
\begin{verbatim}  inClickTimes = Times where a click should be generated (in s).
  inWantedSampleFreq = Sample frequency of the output signal (in Hz).
  inTimeRange = Time range for which clicks should be generated (only times
                from inClickTimes that fall into this range will be used).
                Specified as [min max] (in s).
                If empty or not specified, the full range will be used.
  inAmplitudes = Amplitudes corresponding to the click times.
                 Should be in [0...1] range.
                 If this is a single scalar, this value will be used for all
                 click times.
                 If empty or not specified, 1 is used by default.

\end{verbatim}
\textbf{Output:}
\begin{verbatim}  outClickSignal = The click sound signal.

\end{verbatim}
\textbf{Remarks:}
\begin{verbatim}  The clicks are very simple 0 to 1 transitions (Dirac pulses).
  The end time of the time range is exclusive.

\end{verbatim}
\textbf{Example:}
\begin{verbatim}  s = IPEMConvertToClickSound(0:0.5:5,44100,[1 4],0.8);

\end{verbatim}
\textbf{Authors:}
\begin{verbatim}  Koen Tanghe - 20040423
\end{verbatim}


\newpage
\section*{\hyperlink{Concepts:IPEMConvertToMIDINoteNr}{IPEMConvertToMIDINoteNr}\sectionmark{IPEMConvertToMIDINoteNr}}
\addcontentsline{toc}{subsection}{IPEMConvertToMIDINoteNr}
\hypertarget{FuncRef:IPEMConvertToMIDINoteNr}{}

\textbf{Usage:}
\begin{verbatim}  outMIDINoteNr = IPEMConvertToMIDINoteNr(inNoteString)

\end{verbatim}
\textbf{Description:}
\begin{verbatim}  Converts a cell array of note specification strings to an array of
  MIDI note numbers.

\end{verbatim}
\textbf{Input arguments:}
\begin{verbatim}  inNoteString = cell array of note specification string

\end{verbatim}
\textbf{Output:}
\begin{verbatim}  outMIDINoteNr = MIDI note number array

\end{verbatim}
\textbf{Remarks:}
\begin{verbatim}  Both sharps ('#') or flats ('b') can be used.
  You can specify a sign in front of the octave part, if you want to...
  Things like 'Cb' and 'E#' will be interpreted as, respectively, 'B' and 'F'.

  If you want to compare note strings, you'll have to make sure that you
  always stick to the same notation method (sharps OR flats).
  If you don't, you'll have to use the note NUMBERS to test for equality,
  and not the note STRINGS !

\end{verbatim}
\textbf{Example:}
\begin{verbatim}  IPEMConvertToMIDINoteNr('A4')

\end{verbatim}
\textbf{Authors:}
\begin{verbatim}  Koen Tanghe - 20000315
\end{verbatim}


\newpage
\section*{\hyperlink{Concepts:IPEMConvertToNoteString}{IPEMConvertToNoteString}\sectionmark{IPEMConvertToNoteString}}
\addcontentsline{toc}{subsection}{IPEMConvertToNoteString}
\hypertarget{FuncRef:IPEMConvertToNoteString}{}

\textbf{Usage:}
\begin{verbatim}  outNoteString = IPEMConvertToNoteString (inMIDINoteNr,inUseFlats)

\end{verbatim}
\textbf{Description:}
\begin{verbatim}  Converts an array of MIDI note numbers to a cell array of note strings.

\end{verbatim}
\textbf{Input arguments:}
\begin{verbatim}  inMIDINoteNr = array containing MIDI note numbers
  inUseFlats = if non-zero, flats will be used instead of sharps
               if empty or not specified, 0 is used by default (sharps)

\end{verbatim}
\textbf{Output:}
\begin{verbatim}  outNoteString = cell array with note specification strings

\end{verbatim}
\textbf{Remarks:}
\begin{verbatim}  Where inMIDINoteNr contains an illegal MIDI note number (NaN), the note
  string corresponding with that value will be left empty...

\end{verbatim}
\textbf{Example:}
\begin{verbatim}  IPEMConvertToNoteString([69 70 ; 58 57])

\end{verbatim}
\textbf{Authors:}
\begin{verbatim}  Koen Tanghe - 20000315
\end{verbatim}


\newpage
\section*{\hyperlink{Concepts:IPEMCountZeroCrossings}{IPEMCountZeroCrossings}\sectionmark{IPEMCountZeroCrossings}}
\addcontentsline{toc}{subsection}{IPEMCountZeroCrossings}
\hypertarget{FuncRef:IPEMCountZeroCrossings}{}

\textbf{Usage:}
\begin{verbatim}  outZeroCrossings = IPEMCountZeroCrossings(inSignal,inZeroTolerance)

\end{verbatim}
\textbf{Description:}
\begin{verbatim}  Counts the number of zero crossings for each row of inSignal.

\end{verbatim}
\textbf{Input arguments:}
\begin{verbatim}  inSignal = signal to be analyzed for zero-crossings
  inZeroTolerance = defines a region around zero where zero-crossings are not
                    counted (useful for noisy parts of the signal)
                    if empty or not specified, 0 is used by default

\end{verbatim}
\textbf{Output:}
\begin{verbatim}  outZeroCrossings = number of zero crossings for each row

\end{verbatim}
\textbf{Example:}
\begin{verbatim}  ZeroCrossing = IPEMCountZeroCrossings(MusicExcerpt,0.01);

\end{verbatim}
\textbf{Authors:}
\begin{verbatim}  Koen Tanghe - 20000509
\end{verbatim}


\newpage
\section*{\hyperlink{Concepts:IPEMCreateMask}{IPEMCreateMask}\sectionmark{IPEMCreateMask}}
\addcontentsline{toc}{subsection}{IPEMCreateMask}
\hypertarget{FuncRef:IPEMCreateMask}{}

\textbf{Usage:}
\begin{verbatim}  outMask = IPEMCreateMask(inTime,inStartTimes,inAmplitudes,inDecayPeriod,...
                           inFractionAtDecayPeriod)

\end{verbatim}
\textbf{Description:}
\begin{verbatim}  This function creates an exponentially decaying mask from the given start
  times and amplitudes.

\end{verbatim}
\textbf{Input arguments:}
\begin{verbatim}  inTime = the time span for which a mask is needed (in samples !)
  inStartTimes = the start times of the masking events (in samples !)
  inAmplitudes = the amplitudes of the masking events
  inDecayPeriod = the time (in samples !) it takes before a single mask reaches
                  inFractionAtDecayPeriod of its initial amplitude
  inFractionAtDecayPeriod = fraction of original amplitude at inDecayPeriod
                            if empty or not specified, 0.5 is used by default
                            (i.e. half decay time)
 
\end{verbatim}
\textbf{Output:}
\begin{verbatim}  outMask = the requested mask

\end{verbatim}
\textbf{Remarks:}
\begin{verbatim}  The relation between a decay time (DT) + fraction at decay time (DF)
  specification and a half decay time (HDT) specification is as follows:
    HDT = DT*log(0.5)/log(DF)

\end{verbatim}
\textbf{Example:}
\begin{verbatim}  Mask = IPEMCreateMask(1:1000,[20 250 400 860 900],[1 0.5 1 0.5 1],100);
    or
  fs = 22050; t = 0:1/fs:5-1/fs;
  Mask = IPEMCreateMask(t,[1 2 3],[1 0.5 1],0.2);

\end{verbatim}
\textbf{Authors:}
\begin{verbatim}  Koen Tanghe - 20040106
\end{verbatim}


\newpage
\section*{\hyperlink{Concepts:IPEMDoOnsets}{IPEMDoOnsets}\sectionmark{IPEMDoOnsets}}
\addcontentsline{toc}{subsection}{IPEMDoOnsets}
\hypertarget{FuncRef:IPEMDoOnsets}{}

\textbf{Usage:}
\begin{verbatim}  [Ts,Tsmp] = IPEMDoOnsets(inFileName,inFilePath,inPlotFlag);

\end{verbatim}
\textbf{Description:}
\begin{verbatim}  Convenience function for detecting onsets in a sound signal.
  The onset times are also written to a text file.

\end{verbatim}
\textbf{Input arguments:}
\begin{verbatim}  inFileName = name of the sound file to process (with extension!)
  inFilePath = path to the directory where the sound file is located
               if empty or not specified, IPEMRootDir('input')\Sounds is used
               by default
  inPlotFlag = if non-zero, plots are generated (no plots if zero)
               if empty or not specified, 0 is used by default

\end{verbatim}
\textbf{Output:}
\begin{verbatim}  Ts = onset times in seconds
  Tsmp = onset times in samples

  Additionally, a text file containing info about the detected onsets is
  stored in the same directory as the sound file. It has the same name as the
  sound file without extension, but followed by '_onsets.txt'.

\end{verbatim}
\textbf{Example:}
\begin{verbatim}  [Ts,Tsmp] = IPEMDoOnsets('music.wav');

\end{verbatim}
\textbf{Authors:}
\begin{verbatim}  Koen Tanghe - 20011107
\end{verbatim}


\newpage
\section*{\hyperlink{Concepts:IPEMEnsureDirectory}{IPEMEnsureDirectory}\sectionmark{IPEMEnsureDirectory}}
\addcontentsline{toc}{subsection}{IPEMEnsureDirectory}
\hypertarget{FuncRef:IPEMEnsureDirectory}{}

\textbf{Usage:}
\begin{verbatim}  outResult = IPEMEnsureDirectory (inDirectoryPath,inCreate)

\end{verbatim}
\textbf{Description:}
\begin{verbatim}  Checks whether the given directory exists (and optionally creates it, if not)

\end{verbatim}
\textbf{Input arguments:}
\begin{verbatim}  inDirectoryPath = full directory path to be checked (created)
  inCreate = if non-zero, the directory will be created if it doesn't exist
             if empty or not specified, 1 is used by default

\end{verbatim}
\textbf{Output:}
\begin{verbatim}  outResult = 1 if the directory already existed before calling this function

\end{verbatim}
\textbf{Remarks:}
\begin{verbatim}  Use this function with care: it's good policy to only create a directory
  1. if this is really necessary (let user specify directory if possible)
  2. within the default output directory (that way, the user's disk doesn't
     get scattered with new directories)

\end{verbatim}
\textbf{Example:}
\begin{verbatim}  IPEMEnsureDirectory(fullfile(IPEMRootDir('output'),'Test','Example1'),1);

  The above line makes sure that a directory Test\Example1 exists within the
  default output directory.

\end{verbatim}
\textbf{Authors:}
\begin{verbatim}  Koen Tanghe - 20040504
\end{verbatim}


\newpage
\section*{\hyperlink{Concepts:IPEMEnvelopeFollower}{IPEMEnvelopeFollower}\sectionmark{IPEMEnvelopeFollower}}
\addcontentsline{toc}{subsection}{IPEMEnvelopeFollower}
\hypertarget{FuncRef:IPEMEnvelopeFollower}{}

\textbf{Usage:}
\begin{verbatim}  [outEnvelope,outEnvelopeFreq] = ...
     IPEMEnvelopeFollower(inSignal,inSignalFreq,inAttackTime,inReleaseTime,...
                          inPlotFlag);

\end{verbatim}
\textbf{Description:}
\begin{verbatim}  Does envelope following on a signal.

\end{verbatim}
\textbf{Input arguments:}
\begin{verbatim}  inSignal = one-dimensional input signal
  inSignalFreq = sample frequency for inSignal (in Hz)
  inAttackTime = time it takes to reach 0.5 for a 0 to 1 step signal (in s)
  inReleaseTime = time it takes to reach 0.5 for a 1 to 0 step signal (in s)
                  if not specified or empty, inAttackTime is used by default
  inPlotFlag = if not specified or empty, 0 is used by default

\end{verbatim}
\textbf{Output:}
\begin{verbatim}  outEnvelope = envelope signal
  outEnvelopeFreq = sample frequency for outEnvelope (same as inSignalFreq)

\end{verbatim}
\textbf{Remarks:}
\begin{verbatim}  Uses one-pole low-pass filters with different coefficients for attack and
  release on the absolute value of the signal. The attack and release times
  are specified as "half-value times".

\end{verbatim}
\textbf{Example:}
\begin{verbatim}  fs = 44100;
  s = zeros(1,fs*5);
  s(fs:fs*3) = 1;
  IPEMEnvelopeFollower(s,fs,0.010,0.020,1);

\end{verbatim}
\textbf{Authors:}
\begin{verbatim}  Koen Tanghe - 20030403
\end{verbatim}


\newpage
\section*{\hyperlink{Concepts:IPEMExportFigures}{IPEMExportFigures}\sectionmark{IPEMExportFigures}}
\addcontentsline{toc}{subsection}{IPEMExportFigures}
\hypertarget{FuncRef:IPEMExportFigures}{}

\textbf{Usage:}
\begin{verbatim}  IPEMExportFigures(inInputPath,inOutputPath,inFormat,inUseIPEMLayout)

\end{verbatim}
\textbf{Description:}
\begin{verbatim}  Exports figures to specified graphics format.

\end{verbatim}
\textbf{Input arguments:}
\begin{verbatim}  inInputPath = path to scan for .fig files
                if empty or not specified, the currently opened figures are
                exported, with the names being just 'Figure_No_1', etc...
                (in this case, inOutputPath must be given)
  inOutputPath = path to be used for saving the exported figures
                 if empty or not specified, inInputPath is used by default
  inFormat = graphics format to be used
             this can be one of the following:
               'lowpng' = low resolution png ('-dpng -r75')
               'highpng' = high resolution png ('-dpng -r300')
               'screenpng' = screen resolution png ('-dpng -r0')
               'eps' = black & white eps ('-deps -r600')
               'epsc' = colored eps ('-depsc -r600')
             or an entire specification conforming to Matlab's 'print' options
             (to be specified as a cell array of strings, see 'help print')
             if empty or not specified, 'lowpng' is used by default
  inUseIPEMLayout = if non-zero, the default IPEM layout settings are used
                    if empty or not specified, each figure is used "as is"

\end{verbatim}
\textbf{Output:}
\begin{verbatim}  Files having the same name as the figures, but with an extension depending
  on inFormat.

\end{verbatim}
\textbf{Remarks:}
\begin{verbatim}  This thing is NOT COMPILABLE because in Matlab R11.1 the print command does
  not work when called from a compiled M-file (does work in R12 however).

\end{verbatim}
\textbf{Example:}
\begin{verbatim}  IPEMExportFigures('E:\Koen\Docs\Research\TeacupPaper\Originals');

\end{verbatim}
\textbf{Authors:}
\begin{verbatim}  Koen Tanghe - 20010528
\end{verbatim}


\newpage
\section*{\hyperlink{Concepts:IPEMExtractSegments}{IPEMExtractSegments}\sectionmark{IPEMExtractSegments}}
\addcontentsline{toc}{subsection}{IPEMExtractSegments}
\hypertarget{FuncRef:IPEMExtractSegments}{}

\textbf{Usage:}
\begin{verbatim}  outSignalSegments = IPEMExtractSegments(inSignal,inSampleFreq,inTimeSegments)

\end{verbatim}
\textbf{Description:}
\begin{verbatim}  Extracts segments from the signal according to the given time segments.

\end{verbatim}
\textbf{Input arguments:}
\begin{verbatim}  inSignal = input signal (1 row vector)
  inSampleFreq = sample frequency of inSignal (in Hz)
  inTimeSegments = time segments for cutting up the input signal (in s)
                   (each row contains the start and end of a time segment)
                   if this is a positive scalar, the signal is divided into
                   inTimeSegments parts of equal size + an additional segment
                   containing the remaining part of the signal (which will
                   contain less than inTimeSegments samples)
                   if this is a negative scalar, the signal is divided into
                   -inTimeSegments parts where some segments will contain
                   1 sample more than other segments

\end{verbatim}
\textbf{Output:}
\begin{verbatim}  outSignalSegments = 1 column cell (!) vector of signal segments
                      (each row contains a segment)

\end{verbatim}
\textbf{Example:}
\begin{verbatim}  Parts = IPEMExtractSegments(s,fs,OnsetSegments);
  FirstSegment = Parts{1}; % etc...

\end{verbatim}
\textbf{Authors:}
\begin{verbatim}  Koen Tanghe - 20000510
\end{verbatim}


\newpage
\section*{\hyperlink{Concepts:IPEMFadeLinear}{IPEMFadeLinear}\sectionmark{IPEMFadeLinear}}
\addcontentsline{toc}{subsection}{IPEMFadeLinear}
\hypertarget{FuncRef:IPEMFadeLinear}{}

\textbf{Usage:}
\begin{verbatim}  outSignal = IPEMFadeLinear (inSignal,inSampleFreq,inFadeTime)

\end{verbatim}
\textbf{Description:}
\begin{verbatim}  Does a linear fade in and out of the signal over inFadeTime seconds.
  Can be used for smoothing beginnings and endings of a synthesized signal.

\end{verbatim}
\textbf{Input arguments:}
\begin{verbatim}  inSignal = the input signal
  inSampleFreq = sample frequency of the signal (in Hz)
  inFadeTime = the time to fade in and out (in s)
               if this is a scalar, both fade in and out time are the same
               if this is a two element row vector, inFadeTime(1) is the fade
               in time and inFadeTime(2) is the fade out time
               (fade times should be smaller than the signal duration)

\end{verbatim}
\textbf{Output:}
\begin{verbatim}  outSignal = the faded signal

\end{verbatim}
\textbf{Remarks:}
\begin{verbatim}  If the sum of fade in and fade out time is >= the sound length,
  then the fades will accumulate at the overlapping section, thus introducing
  a quadratic fade at that section.

\end{verbatim}
\textbf{Example:}
\begin{verbatim}  Signal = IPEMFadeLinear(Signal,SampleFreq,0.01);

\end{verbatim}
\textbf{Authors:}
\begin{verbatim}  Koen Tanghe - 20000913
\end{verbatim}


\newpage
\section*{\hyperlink{Concepts:IPEMFindAllPeaks}{IPEMFindAllPeaks}\sectionmark{IPEMFindAllPeaks}}
\addcontentsline{toc}{subsection}{IPEMFindAllPeaks}
\hypertarget{FuncRef:IPEMFindAllPeaks}{}

\textbf{Usage:}
\begin{verbatim}  thePeakIndices = IPEMFindAllPeaks (inSignal,inFlatPreference,inPlotFlag)

\end{verbatim}
\textbf{Description:}
\begin{verbatim}  This function finds the indices of all peaks in the given signal vector.
  A peak is taken to occur whenever a rising part in the signal is followed
  by a falling part.

\end{verbatim}
\textbf{Input arguments:}
\begin{verbatim}  inSignal = the signal that is scanned for peaks (a row vector)
  inFlatPreference = preference for choosing exact position in case of
                     flat peaks:
                     if 'left', the leftmost point of a flat peak is chosen
                     if 'center', the center of a flat peak is chosen
                     if 'right', the rightmost point of a flat peak is chosen
                     if empty or not specified, 'center' is used by default
  inPlotFlag = if non-zero, plots are generated
               if empty or not specified, 0 is used by default

\end{verbatim}
\textbf{Output:}
\begin{verbatim}  outPeakIndices = a vector with the indices of the peaks (could be empty!) 

\end{verbatim}
\textbf{Example:}
\begin{verbatim}  PeakIndices = IPEMFindAllPeaks(Signal,'center');

\end{verbatim}
\textbf{Authors:}
\begin{verbatim}  Koen Tanghe - 20000509
\end{verbatim}


\newpage
\section*{\hyperlink{Concepts:IPEMFindNearestMinima}{IPEMFindNearestMinima}\sectionmark{IPEMFindNearestMinima}}
\addcontentsline{toc}{subsection}{IPEMFindNearestMinima}
\hypertarget{FuncRef:IPEMFindNearestMinima}{}

\textbf{Usage:}
\begin{verbatim}  [outLeftIndex, outRightIndex] = IPEMFindNearestMinima(inSignal,inPeakIndex)

\end{verbatim}
\textbf{Description:}
\begin{verbatim}  Finds the nearest minima to the left and the right of the specified peak.
  Peaks can be found with IPEMFindAllPeaks.

\end{verbatim}
\textbf{Input arguments:}
\begin{verbatim}  inSignal = the signal to analyze
  inPeakIndex = index in inSignal of a peak

\end{verbatim}
\textbf{Output:}
\begin{verbatim}  outLeftIndex = the index of the nearest miminum to the left of the peak
                 if no real minimum was found, 1 is returned  
  outRightIndex = the index of the nearest miminum to the right of the peak
                  if no real minimum was found, length(inSignal) is returned

\end{verbatim}
\textbf{Example:}
\begin{verbatim}  [LeftIndex,RightIndex] = IPEMFindNearestMinima(Signal,PeakIndex);

\end{verbatim}
\textbf{Authors:}
\begin{verbatim}  Koen Tanghe - 20000208
\end{verbatim}


\newpage
\section*{\hyperlink{Concepts:IPEMFindNoteFromFrequency}{IPEMFindNoteFromFrequency}\sectionmark{IPEMFindNoteFromFrequency}}
\addcontentsline{toc}{subsection}{IPEMFindNoteFromFrequency}
\hypertarget{FuncRef:IPEMFindNoteFromFrequency}{}

\textbf{Usage:}
\begin{verbatim}  [outNoteNr,outOctaveNr,outExactFrequency,outNoteString] =
     IPEMFindNoteFromFrequency(inFrequency,inRefNoteNr,inRefOctaveNr,
                               inRefFreq,inNotesPerOctave,inOctaveRatio)

\end{verbatim}
\textbf{Description:}
\begin{verbatim}  Finds nearest note corresponding to the given frequency.
  Supports non-standard tone scales that can be calculated like this:

      Frequency = RefFreq * OctaveRatio^Exponent

      where:

      Exponent = (OctaveNr-RefOctaveNr) + (NoteNr-RefNoteNr)/NotesPerOctave 

\end{verbatim}
\textbf{Input arguments:}
\begin{verbatim}  inFrequency = frequency for which a corresponding note must be found
  inRefNoteNr = the rank number of the reference note
                if empty or not specified, 10 is used by default
  inRefOctaveNr = the octave number of the reference note
                  if empty or not specified, 4 is used by default
  inRefFreq = the frequency of the reference note (in Hz)
              if empty or not specified, 440 Hz is used
  inNotesPerOctave = the number of notes in one octave
                     if empty or not specified, 12 is used by default
  inOctaveRatio = the frequency ratio between two octaves
                  if empty or not specified, 2 is used

\end{verbatim}
\textbf{Output:}
\begin{verbatim}  outNoteNr = rank number of the note (1 to inNotesPerOctave)
  outOctaveNr = octave number of the note (integer)
  outExactFrequency = exact frequency of the found note (in Hz)
  outNoteString = string representing the found note
                  (only for standard tone scale, otherwise empty)

\end{verbatim}
\textbf{Remarks:}
\begin{verbatim}  Getting the note name for standard tone scale:
    MIDINoteNr = (outNoteNr-1) + 12*(outOctaveNr+1);
    NoteString = IPEMConvertToNoteString(MIDINoteNr);

\end{verbatim}
\textbf{Example:}
\begin{verbatim}  [NoteNr,OctaveNr,Freq,Name] = ...
    IPEMFindNoteFromFrequency([441.01 153 ; 222 94]);

\end{verbatim}
\textbf{Authors:}
\begin{verbatim}  Koen Tanghe - 20000629
\end{verbatim}


\newpage
\section*{\hyperlink{Concepts:IPEMGenerateBandPassedNoise}{IPEMGenerateBandPassedNoise}\sectionmark{IPEMGenerateBandPassedNoise}}
\addcontentsline{toc}{subsection}{IPEMGenerateBandPassedNoise}
\hypertarget{FuncRef:IPEMGenerateBandPassedNoise}{}

\textbf{Usage:}
\begin{verbatim}  outSignal = ...
    IPEMGenerateBandPassedNoise(inPassBands,inDuration,inSampleFreq,...
                                indBLevel,inFFTWidth)

\end{verbatim}
\textbf{Description:}
\begin{verbatim}  Generates a band passed noise signal using inverse FFT.

\end{verbatim}
\textbf{Input arguments:}
\begin{verbatim}  inPassBands = 2 column vector containing specification of the pass bands:
                each row specifies a low and high frequency (in Hz)
  inDuration = duration of the sound (in s)
  inSampleFreq = wanted sample frequency (in Hz)
                 if empty or not specified, 22050 is used by default
  indBLevel = dB level for the signal (in dB)
              if empty or not specified, -6 is used by default
  inFFTWidth = FFT width to use for the inverse FFT
               if -1, the next power of 2 >= number of samples is used
               if empty or not specified, the number of samples is used
               by default

\end{verbatim}
\textbf{Output:}
\begin{verbatim}  outSignal = generated band passed noise signal

\end{verbatim}
\textbf{Remarks:}
\begin{verbatim}  When using an inFFTWidth that is smaller than the number of requested
  samples, the resulting sound will consist of the appropriate number of
  repetitions of the inverse fft of size inFFTWidth, thus introducing an
  artificial period of the size of 1 inverse fft. For non dense noise, this
  can become noticable.

\end{verbatim}
\textbf{Example:}
\begin{verbatim}  s = IPEMGenerateBandPassedNoise([1000 1200],1);

\end{verbatim}
\textbf{Authors:}
\begin{verbatim}  Koen Tanghe - 20000926
\end{verbatim}


\newpage
\section*{\hyperlink{Concepts:IPEMGenerateFrameBasedSegments}{IPEMGenerateFrameBasedSegments}\sectionmark{IPEMGenerateFrameBasedSegments}}
\addcontentsline{toc}{subsection}{IPEMGenerateFrameBasedSegments}
\hypertarget{FuncRef:IPEMGenerateFrameBasedSegments}{}

\textbf{Usage:}
\begin{verbatim}  [outSegments,outActualWidth,outActualInterval] = ...
    IPEMGenerateFrameBasedSegments(inSignalOrDuration,inSampleFreq,...
                                   inFrameWidth,inFrameInterval,...
                                   inIncludeIncompleteFrames)

\end{verbatim}
\textbf{Description:}
\begin{verbatim}  Generates (overlapping) time segments using equally sized frames.

\end{verbatim}
\textbf{Input arguments:}
\begin{verbatim}  inSignalOrDuration = if this is a vector, it is the signal for which to
                       calculate the frame-based time segments
                       if this is a scalar, it is the duration of a signal
                       (in s) (allows to generate segments even if you don't
                       have a signal)
  inSampleFreq = sample frequency of inSignal (in Hz)
  inFrameWidth = wanted width of 1 frame (in s)
  inFrameInterval = wanted interval between frames (in s)
  inIncludeIncompleteFrames = if non-zero, incomplete frames at the end are
                              included as well (different lengths!)
                              if empty or not specified, 0 is used by default

\end{verbatim}
\textbf{Output:}
\begin{verbatim}  outSegments = 2-column array of segments:
                each row contains the start and end moment of a segment (in s)
  outActualWidth = width used (to get an integer number of samples)
  outActualInterval = interval used (to get an integer number of samples)

\end{verbatim}
\textbf{Remarks:}
\begin{verbatim}  Because of the discrete nature of sampled signals, the frame width and
  interval requested for by the caller might not be representable in the
  resolution specified by the sample frequency.
  In that case, the nearest width and interval that are integer multiples of
  the sample period will be used. This is what the outActualWidth and
  outActualInterval arguments are for.

\end{verbatim}
\textbf{Example:}
\begin{verbatim}  [Segments,Width,Interval] = IPEMGenerateFrameBasedSegments(1,100,0.1,0.025)

  Width will be 0.1 s (as requested)
  Interval will become 0.03 s, since 0.025 s does not exist in a 1/100 s
  resolution 

\end{verbatim}
\textbf{Authors:}
\begin{verbatim}  Koen Tanghe - 20000418
\end{verbatim}


\newpage
\section*{\hyperlink{Concepts:IPEMGeneratePitchShiftScript}{IPEMGeneratePitchShiftScript}\sectionmark{IPEMGeneratePitchShiftScript}}
\addcontentsline{toc}{subsection}{IPEMGeneratePitchShiftScript}
\hypertarget{FuncRef:IPEMGeneratePitchShiftScript}{}

\textbf{Usage:}
\begin{verbatim}  IPEMGeneratePitchShiftScript(inFileName,inFilePath,inCentsToRaise,...
                               inCentsPerStep,inScriptFilePath,inOnlyShift)

\end{verbatim}
\textbf{Description:}
\begin{verbatim}  Generates a CoolEdit2000 script for creating a sound file containing
  appended pitch-shifted versions of the original sound file.
  Generated pitch-shifted sounds can be mixed with original sound.

  Way to go:
    1. run this function on your sound file
    2. start CoolEdit2000 and open the sound file
    3. go to 'Options\Scripts & Batch Processing' and open the script file
    4. deselect 'Pause at Dialogs', select the script 'shift' and press 'Run'
    5. wait...
    6. save your sound file (either using suffix '_shifted' or '_mixed')

\end{verbatim}
\textbf{Input arguments:}
\begin{verbatim}  inFileName = name of the sound file to generate a script for
  inFilePath = path to the location of the sound file
               if empty or not specified, IPEMRootDir('input')\Sounds is used
               by default
  inCentsToRaise = maximum number of cents for the pitch to raise
                   if empty or not specified, 1200 is used by default
  inCentsPerStep = number of cents the pitch will raise per step
                   if empty or not specified, 100 is used by default
  inScriptFilePath = path to the location where the script file will be saved
                     if empty or not specified, the same location as the sound
                     file is used by default
  inOnlyShift = if 1, the script will only generate the shifted versions
                of the sound signal (from low to high)
                otherwise, it will immediately mix these shifted versions with
                the appropriate number of copies of the original sound signal
                if empty or not specified, 0 is used by default
                (mixing enabled)

\end{verbatim}
\textbf{Output:}
\begin{verbatim}  A CoolEdit2000 script with the same name as the sound file, but with the
  extension .scp

\end{verbatim}
\textbf{Remarks:}
\begin{verbatim}  CoolEdit2000 scripts contain selection ranges and sample rates (which are
  different for different sound files), so you can only use the script for the
  sound file you created it for (or for sound files with exactly the same
  number of samples and sampling rate)...

  You can download a trial version of CoolEdit2000 from:
      http://www.syntrillium.com
  Make sure you select function groups 1 (Save...) and 3 (Stretching...)
  in order to be able to run the script.

\end{verbatim}
\textbf{Example:}
\begin{verbatim}  IPEMGeneratePitchShiftScript('bottle.wav');

\end{verbatim}
\textbf{Authors:}
\begin{verbatim}  Koen Tanghe - 20000418
\end{verbatim}


\newpage
\section*{\hyperlink{Concepts:IPEMGetAllCombinations}{IPEMGetAllCombinations}\sectionmark{IPEMGetAllCombinations}}
\addcontentsline{toc}{subsection}{IPEMGetAllCombinations}
\hypertarget{FuncRef:IPEMGetAllCombinations}{}

\textbf{Usage:}
\begin{verbatim}  outCombinations = IPEMGetAllCombinations(inValueArrays)

\end{verbatim}
\textbf{Description:}
\begin{verbatim}  Finds all combinations of values by picking a value from a specified set of
  values for each of the multiple dimensions.

\end{verbatim}
\textbf{Input arguments:}
\begin{verbatim}  inValueArrays = Cell array containing for each dimension an array of the 
                  values from which one should be chosen for a combination.

\end{verbatim}
\textbf{Output:}
\begin{verbatim}  outCombinations = Array containing all possible combinations in rows.

\end{verbatim}
\textbf{Example:}
\begin{verbatim}  Comb = IPEMGetAllCombinations({[1 2 3],[10 20],[111 222 333]})

\end{verbatim}
\textbf{Authors:}
\begin{verbatim}  Koen Tanghe - 20040510
\end{verbatim}


\newpage
\section*{\hyperlink{Concepts:IPEMGetCrestFactor}{IPEMGetCrestFactor}\sectionmark{IPEMGetCrestFactor}}
\addcontentsline{toc}{subsection}{IPEMGetCrestFactor}
\hypertarget{FuncRef:IPEMGetCrestFactor}{}

\textbf{Usage:}
\begin{verbatim}  outCrestFactor = IPEMGetCrestFactor(inSignal)

\end{verbatim}
\textbf{Description:}
\begin{verbatim}  Gets the crest factor of the signal, which is defined as:
       CF = max(abs(signal))/rms(signal))

\end{verbatim}
\textbf{Input arguments:}
\begin{verbatim}  inSignal = input signal

\end{verbatim}
\textbf{Output:}
\begin{verbatim}  outCrestFactor = the crest factor over the entire signal

\end{verbatim}
\textbf{Example:}
\begin{verbatim}  [s,fs] = IPEMReadSoundFile;
  CF = IPEMGetCrestFactor(s);

\end{verbatim}
\textbf{Authors:}
\begin{verbatim}  Koen Tanghe - 20030415
\end{verbatim}


\newpage
\section*{\hyperlink{Concepts:IPEMGetFileList}{IPEMGetFileList}\sectionmark{IPEMGetFileList}}
\addcontentsline{toc}{subsection}{IPEMGetFileList}
\hypertarget{FuncRef:IPEMGetFileList}{}

\textbf{Usage:}
\begin{verbatim}  [outFullFiles,outFileNames,outFilePaths] = ...
    IPEMGetFileList(inDirectory,inFilePattern,inRecurse)

\end{verbatim}
\textbf{Description:}
\begin{verbatim}  Gets a list of files located in a given directory and whose names adhere
  to a specific pattern (option for recursion).

\end{verbatim}
\textbf{Input arguments:}
\begin{verbatim}  inDirectory = Path to the directory to search in.
  inFilePattern = Pattern to which files should adhere in order to be listed.
                  If empty or not specified, '*' is used by default.
  inRecurse = If non-zero, also subdirectories are scanned.
              If empty or not specified, 0 is used by default.

\end{verbatim}
\textbf{Output:}
\begin{verbatim}  outFullFiles = Cell array containing the full file specifications.
  outFileNames = Cell array containing only the file names.
  outFilePaths = Cell array containing only the paths to the files.

\end{verbatim}
\textbf{Example:}
\begin{verbatim}  [F,FN,FP] = IPEMGetFileList('D:\Koen\','*.txt',1);

\end{verbatim}
\textbf{Authors:}
\begin{verbatim}  Koen Tanghe - 20040507
\end{verbatim}


\newpage
\section*{\hyperlink{Concepts:IPEMGetKurtosis}{IPEMGetKurtosis}\sectionmark{IPEMGetKurtosis}}
\addcontentsline{toc}{subsection}{IPEMGetKurtosis}
\hypertarget{FuncRef:IPEMGetKurtosis}{}

\textbf{Usage:}
\begin{verbatim}  outKurtosis = IPEMGetKurtosis(inData)

\end{verbatim}
\textbf{Description:}
\begin{verbatim}  Gets the kurtosis of the data, which is defined here as:
    Kurtosis(X) = sum((X-mu)^4) / (N*sigma^4) - 3
  where
    mu = average of X
    sigma = standard deviation (normalized using N) of X
    N = number of data points in X

\end{verbatim}
\textbf{Input arguments:}
\begin{verbatim}  inData = data vector

\end{verbatim}
\textbf{Output:}
\begin{verbatim}  outKurtosis = the kurtosis of the data

\end{verbatim}
\textbf{Example:}
\begin{verbatim}  X = [1 2 3 1 5 6 4 3];
  Kurtosis = IPEMGetKurtosis(X);

\end{verbatim}
\textbf{Authors:}
\begin{verbatim}  Koen Tanghe - 20030415
\end{verbatim}


\newpage
\section*{\hyperlink{Concepts:IPEMGetLevel}{IPEMGetLevel}\sectionmark{IPEMGetLevel}}
\addcontentsline{toc}{subsection}{IPEMGetLevel}
\hypertarget{FuncRef:IPEMGetLevel}{}

\textbf{Usage:}
\begin{verbatim}  outRMSLevel = IPEMGetLevel(inSignal,inUseDecibels)

\end{verbatim}
\textbf{Description:}
\begin{verbatim}  Calculates the average RMS power level of a signal in dB (or not).

\end{verbatim}
\textbf{Input arguments:}
\begin{verbatim}  inSignal = signal to be analyzed
  inUseDecibels = if non-zero, dB units are used instead of plain RMS
                  if empty or not specified, 1 is used by default

\end{verbatim}
\textbf{Output:}
\begin{verbatim}  outRMSLevel = average RMS level (in dB if inUseDecibels is non-zero)

\end{verbatim}
\textbf{Remarks:}
\begin{verbatim}  The reference value of 0 dB is the level of a square wave with amplitude 1.
  Thus, a sine wave with amplitude 1 yields -3.01 dB.

\end{verbatim}
\textbf{Example:}
\begin{verbatim}  RMSLevel = IPEMGetLevel(Signal);

\end{verbatim}
\textbf{Authors:}
\begin{verbatim}  Koen Tanghe - 20040323
\end{verbatim}


\newpage
\section*{\hyperlink{Concepts:IPEMGetRolloff}{IPEMGetRolloff}\sectionmark{IPEMGetRolloff}}
\addcontentsline{toc}{subsection}{IPEMGetRolloff}
\hypertarget{FuncRef:IPEMGetRolloff}{}

\textbf{Usage:}
\begin{verbatim}  outRolloffIndex = IPEMGetRolloff(inData,inRolloffFraction)

\end{verbatim}
\textbf{Description:}
\begin{verbatim}  Gets the index R in inData for which:
    sum(inData,1,R) = inRolloffFraction*sum(inData,1,length(inData))

\end{verbatim}
\textbf{Input arguments:}
\begin{verbatim}  inData = Data to be analyzed (each row is analyzed separately).
  inRolloffFraction = Column vector containing a rolloff fraction for each row
                      in inData. If a scalar is given, the same value is used
                      for all rows.

\end{verbatim}
\textbf{Output:}
\begin{verbatim}  outRolloffIndex = column vector containing the rolloff indices for each row
                    in inData

\end{verbatim}
\textbf{Example:}
\begin{verbatim}  v = rand(1,20);
  ri = IPEMGetRolloff(v,0.85);
  v(ri)

\end{verbatim}
\textbf{Authors:}
\begin{verbatim}  Koen Tanghe - 20030523
\end{verbatim}


\newpage
\section*{\hyperlink{Concepts:IPEMGetRoughnessFFTReference}{IPEMGetRoughnessFFTReference}\sectionmark{IPEMGetRoughnessFFTReference}}
\addcontentsline{toc}{subsection}{IPEMGetRoughnessFFTReference}
\hypertarget{FuncRef:IPEMGetRoughnessFFTReference}{}

\textbf{Usage:}
\begin{verbatim}  outReferenceValue = IPEMGetRoughnessFFTReference(indBLevel,inLength)

\end{verbatim}
\textbf{Description:}
\begin{verbatim}  Returns the reference value for IPEMRoughnessFFT at the specified
  dB level.

\end{verbatim}
\textbf{Input arguments:}
\begin{verbatim}  indBLevel = dB level for which the reference value is requested
  inLength = length (in s) used for calculating roughness with FFT method
             if empty or not specified, 1 is used by default

\end{verbatim}
\textbf{Output:}
\begin{verbatim}  outReferenceValue = the requested reference value

\end{verbatim}
\textbf{Remarks:}
\begin{verbatim}  The reference is taken to be an AM sine tone at the specified dB level
  with the following properties:
    - carrier frequency = 1000 Hz
    - modulation frequency = 70 Hz
    - modulation index = 1
    - duration = 1 s
  The reference value is calculated over the reference sound as a whole
  (not as the mean of the calculated frames over the sound)

\end{verbatim}
\textbf{Example:}
\begin{verbatim}  Ref = IPEMGetRoughnessFFTReference(-20,0.5)

\end{verbatim}
\textbf{Authors:}
\begin{verbatim}  Koen Tanghe - 20010816
\end{verbatim}


\newpage
\section*{\hyperlink{Concepts:IPEMGetSkew}{IPEMGetSkew}\sectionmark{IPEMGetSkew}}
\addcontentsline{toc}{subsection}{IPEMGetSkew}
\hypertarget{FuncRef:IPEMGetSkew}{}

\textbf{Usage:}
\begin{verbatim}  outSkew = IPEMGetSkew(inData)

\end{verbatim}
\textbf{Description:}
\begin{verbatim}  Gets the skew of the data, which is defined here as:
    Skew(X) = sum((X-mu)^3) / (N*sigma^3)
  where
    mu = average of X
    sigma = standard deviation (normalized using N) of X
    N = number of data points in X

\end{verbatim}
\textbf{Input arguments:}
\begin{verbatim}  inData = data vector

\end{verbatim}
\textbf{Output:}
\begin{verbatim}  outSkew = the skew of the data

\end{verbatim}
\textbf{Example:}
\begin{verbatim}  X = [1 2 3 1 5 6 4 3];
  Skew = IPEMGetSkew(X);

\end{verbatim}
\textbf{Authors:}
\begin{verbatim}  Koen Tanghe - 20030415
\end{verbatim}


\newpage
\section*{\hyperlink{Concepts:IPEMHandleInputArguments}{IPEMHandleInputArguments}\sectionmark{IPEMHandleInputArguments}}
\addcontentsline{toc}{subsection}{IPEMHandleInputArguments}
\hypertarget{FuncRef:IPEMHandleInputArguments}{}

\textbf{Usage:}
\begin{verbatim}  [...] = IPEMHandleInputArguments(inPassedArgs,inNumFirstOptionalArg,
                                   inDefaults)

\end{verbatim}
\textbf{Description:}
\begin{verbatim}  This function is a simple but general input handling routine.
  Just pass what you got from your caller, say what variables you expect and
  which of them are optional, and specify the defaults (which will be used to
  replace arguments that were left empty).
  After calling this function, you'll have the valid arguments you need to
  proceed with.

\end{verbatim}
\textbf{Input arguments:}
\begin{verbatim}  inPassedArguments = cell array with the arguments as passed to the caller
  inNumFirstOptionalArg = number of the first optional argument
                          if empty, length(inDefaults)+1 is used by default
                          (this means: no optional arguments)
  inDefaults = cell array with default values for the arguments that were
               left empty

\end{verbatim}
\textbf{Output:}
\begin{verbatim}  The specified variables are set in the caller's name space or an appropriate
  error message is returned if something was wrong with the arguments.

\end{verbatim}
\textbf{Example:}
\begin{verbatim}
  This is a function using IPEMHandleInputArguments:

      function Test (varargin)
      % Usage: 
      %   Test(inArg1,inArg2,inArg3,inArg4)
      % ----------------------------------------------------------------------

      % Handle input arguments
      [inArg1,inArg2,inArg3,inArg4] = ...
          IPEMHandleInputArguments (varargin, 3, {'def1' 'def2' 0.1 0.2});

      % Rest of code for Test
      .....

  If you now call Test the following way:

      Test('abc',[],5);

  the internal variables of Test after the call to IPEMHandleInputArguments
  will be:

      inArg1 = 'abc'
      inArg2 = 'def2'
      inArg3 = 5
      inArg4 = 0.2

\end{verbatim}
\textbf{Authors:}
\begin{verbatim}  Koen Tanghe - 19991109
\end{verbatim}


\newpage
\section*{\hyperlink{Concepts:IPEMHarmonicTone}{IPEMHarmonicTone}\sectionmark{IPEMHarmonicTone}}
\addcontentsline{toc}{subsection}{IPEMHarmonicTone}
\hypertarget{FuncRef:IPEMHarmonicTone}{}

\textbf{Usage:}
\begin{verbatim}  outSignal = IPEMHarmonicTone (inFundamentalFreq,inDuration,inSampleFreq,...
                                inPhaseFlag,indBLevel,inNumOfHarmonics)

\end{verbatim}
\textbf{Description:}
\begin{verbatim}  This function generates a tone with a number of harmonics.
  The amplitudes of the harmonics vary as 1/N (N = number of harmonic).

\end{verbatim}
\textbf{Input arguments:}
\begin{verbatim}  inFundamentalFreq = the main frequency in Herz
  inDuration = the duration (in s)
               if empty or not specified, 1 is used by default
  inSampleFreq = the desired sample frequency for the output signal (in Hz)
                 if empty or not specified, 22050 is used by default
  inPhaseFlag = for choosing whether random phase has to be used or not
                (1 to use random phase, 0 otherwise)
                if empty or not specified, 1 is used by default
  indBLevel = dB level of generated tone (in dB)
              if empty or not specified, no level adjustment is performed
  inNumOfHarmonics = number of harmonics (including fundamental frequency)
                     if empty or not specified, 10 is used by default

\end{verbatim}
\textbf{Output:}
\begin{verbatim}  outSignal = the signal for the tone

\end{verbatim}
\textbf{Example:}
\begin{verbatim}  Signal = IPEMHarmonicTone(440,1,22050,1,-20,5);

\end{verbatim}
\textbf{Authors:}
\begin{verbatim}  Koen Tanghe - 20010116
\end{verbatim}


\newpage
\section*{\hyperlink{Concepts:IPEMHarmonicToneComplex}{IPEMHarmonicToneComplex}\sectionmark{IPEMHarmonicToneComplex}}
\addcontentsline{toc}{subsection}{IPEMHarmonicToneComplex}
\hypertarget{FuncRef:IPEMHarmonicToneComplex}{}

\textbf{Usage:}
\begin{verbatim}  outSignal = IPEMHarmonicToneComplex(inToneVector,inDuration,inSampleFreq,...
                                      inPhaseFlag,indBLevel,inNumOfHarmonics)

\end{verbatim}
\textbf{Description:}
\begin{verbatim}  This function generates a tone complex built up of harmonic tones.

\end{verbatim}
\textbf{Input arguments:}
\begin{verbatim}  inToneVector = a 12 elements vector representing the amplitude for each tone
                 C, C#, D, ... in the tone complex (0 = no tone, 1 = full
                 amplitude)
  inDuration = the duration (in s)
               if empty or not specified, 1 is used by default
  inSampleFreq = the desired sample frequency for the output signal (in Hz)
                 if empty or not specified, 22050 is used by default
  inPhaseFlag = for choosing whether random phase has to be used or not
                (1 to use random phase, 0 otherwise)
                if empty or not specified, 1 is used by default
  indBLevel = dB level of generated tone complex (in dB)
              if empty or not specified, no level adjustment is performed
  inNumOfHarmonics = number of harmonics for each tone (including the
                     fundamental frequency)
                     if empty or not specified, 10 is used by default

\end{verbatim}
\textbf{Output:}
\begin{verbatim}  outSignal = the signal for the tone complex

\end{verbatim}
\textbf{Remarks:}
\begin{verbatim}  This is a rather simple routine because the frequencies are fixed:

      notes = C4,    C#4,   D4,    D#4,   E4,    F4,    F#4,   G4,    G#4,
      freqs = 261.6, 277.2, 293.7, 311.1, 329.6, 349.2, 370.0, 392.0, 415.3,

      notes = A,     A#4,   B
      freqs = 440.0, 466.2, 493.9

\end{verbatim}
\textbf{Example:}
\begin{verbatim}  Signal = IPEMHarmonicToneComplex([1 0 0 0 1 0 0 1 0 0 0 0],1,22050,1,-20,5);

\end{verbatim}
\textbf{Authors:}
\begin{verbatim}  Marc Leman - 19990528
  Koen Tanghe - 20010116
\end{verbatim}


\newpage
\section*{\hyperlink{Concepts:IPEMLeakyIntegration}{IPEMLeakyIntegration}\sectionmark{IPEMLeakyIntegration}}
\addcontentsline{toc}{subsection}{IPEMLeakyIntegration}
\hypertarget{FuncRef:IPEMLeakyIntegration}{}

\textbf{Usage:}
\begin{verbatim}  outLeakyIntegration = ...
    IPEMLeakyIntegration(inSignal,inSampleFreq,inHalfDecayTime,...
                         inEnlargement,inPlotFlag)

\end{verbatim}
\textbf{Description:}
\begin{verbatim}  Calculates leaky integration with specified half decay time.

\end{verbatim}
\textbf{Input arguments:}
\begin{verbatim}  inSignal = (multi-dimensional) input signal, each row representing a channel
             leaky integration is performed for each channel
  inSampleFreq = sample frequency of inSignal (in Hz)
  inHalfDecayTime = time (in s) at which an impulse would be reduced to half
                    its value
                    if empty or not specified, 0.1 is used by default
  inEnlargement = time (in s) to enlarge the input signal
                  if -1, 2*inHalfDecayTime is used
                  if empty or not specified, 0 is used by default
  inPlotFlag = if non-zero, plots are generated
               if empty or not specified, 0 is used by default

\end{verbatim}
\textbf{Output:}
\begin{verbatim}  outLeakyIntegration = the leaky integration of the input signal

\end{verbatim}
\textbf{Remarks:}
\begin{verbatim}  Sample frequency of outLeakyIntegration is the same as inSampleFreq.

\end{verbatim}
\textbf{Example: }
\begin{verbatim}  LeakyIntegration = ...
    IPEMLeakyIntegration(PeriodicyPitchMatrix,PeriodicityPitchFreq,0.1,0.1);

\end{verbatim}
\textbf{Authors:}
\begin{verbatim}  Marc Leman - 19991224
  Koen Tanghe - 20010129
\end{verbatim}


\newpage
\section*{\hyperlink{Concepts:IPEMLoadANI}{IPEMLoadANI}\sectionmark{IPEMLoadANI}}
\addcontentsline{toc}{subsection}{IPEMLoadANI}
\hypertarget{FuncRef:IPEMLoadANI}{}

\textbf{Usage:}
\begin{verbatim}  [outANI,outANIFreq,outANIFilterFreqs] = IPEMLoadANI(inName,inPath,...
      inANIName,inANIFreqName,inANIFilterFreqsName)

\end{verbatim}
\textbf{Description:}
\begin{verbatim}  Loads an auditory nerve image and its corresponding sample frequency and
  filter frequencies from a .mat file on disk.

\end{verbatim}
\textbf{Input arguments:}
\begin{verbatim}  inName = the name of the .mat file containing the nerve image
           if empty or not specified, 'ANI.mat' is used by default
  inPath = path to the .mat file
           if empty or not specified, IPEMRootDir('code')\Temp is used
           by default
  inANIName = name of the auditory nerve image array
              if empty or not specified, 'ANI' is used by default
  inANIFreqName = name of the sample frequency of the auditory nerve image
                  if empty or not specified, 'ANIFreq' is used by default
  inANIFilterFreqsName = name of the filter frequencies array
                         if empty or not specified, 'ANIFilterFreqs' is used
                         by default

\end{verbatim}
\textbf{Output:}
\begin{verbatim}  outANI = the auditory nerve image
  outANIFreq = the sample frequency of the ANI
  outANIFilterFreqs = center frequencies used for calaculting the ANI

\end{verbatim}
\textbf{Example:}
\begin{verbatim}  [ANISchum1,ANIFreqSchum1,ANIFilterFreqsSchum1] = ...
    IPEMLoadANI('ANIs','c:\','Schum1','Schum1Freq','Schum1FilterFreqs');

\end{verbatim}
\textbf{Authors:}
\begin{verbatim}  Koen Tanghe - 20000221
\end{verbatim}


\newpage
\section*{\hyperlink{Concepts:IPEMMECAnalysis}{IPEMMECAnalysis}\sectionmark{IPEMMECAnalysis}}
\addcontentsline{toc}{subsection}{IPEMMECAnalysis}
\hypertarget{FuncRef:IPEMMECAnalysis}{}

\textbf{Usage:}
\begin{verbatim}  [outValues,outValuesFreq,outPeriods] = ...
    IPEMMECAnalysis(inSignal,inSampleFreq,inMinPeriod,inMaxPeriod,...
                    inStepSize,inHalfDecayTime,inPlotFlag)

\end{verbatim}
\textbf{Description:}
\begin{verbatim}  Performs a periodicity analysis of a (multi-channel) signal using the
  Minimal Energy Change algorithm. This part calculates the difference
  values. Use IPEMMECFindBestPeriods to find the best matching periods.

\end{verbatim}
\textbf{Input arguments:}
\begin{verbatim}  inSignal = (multi-channel) signal to be analyzed
  inSampleFreq = sample frequency of inSignal (in Hz)
  inMinPeriod = minimum period to look for (in s)
                if empty, 1/inSampleFreq is used by default
  inMaxPeriod = maximum period to look for (in s)
  inStepSize = interval between successive evaluations (in s)
               if empty or not specified, 1/inSampleFreq is used by default
               (this corresponds to a step size of 1 sample)
  inHalfDecayTime = half decay time for leaky integration (in s)
                    if empty or not specified, no integration is performed
                    (this corresponds to inHalfDecayTime = 0)
  inPlotFlag = if non-zero, plot will be generated
               if empty or not specified, 0 is used by default

\end{verbatim}
\textbf{Output:}
\begin{verbatim}  outValues = cell array containing the calculated values for each channel
              and for each period at each evaluated moment
              (outValues{i}(j,k) represents the value calculated for the j-th
              period at the time corresponding with index k for channel i)
  outValuesFreq = sample frequency of outValues (in Hz)
  outPeriods = column vector containing the periods for which values were
               calculated (in s)

\end{verbatim}
\textbf{Example:}
\begin{verbatim}  [Values,ValuesFreq,Periods] = IPEMMECAnalysis(RMS,RMSFreq,0.5,5,[],1.5);

\end{verbatim}
\textbf{Authors:}
\begin{verbatim}  Koen Tanghe - 20010225
\end{verbatim}


\newpage
\section*{\hyperlink{Concepts:IPEMMECExtractPatterns}{IPEMMECExtractPatterns}\sectionmark{IPEMMECExtractPatterns}}
\addcontentsline{toc}{subsection}{IPEMMECExtractPatterns}
\hypertarget{FuncRef:IPEMMECExtractPatterns}{}

\textbf{Usage:}
\begin{verbatim}  [outPatterns,outPatternLenghts,outPatternFreq] = ...
    IPEMMECExtractPatterns(inAnalyzedSignal,inSampleFreq,inPeriods,...
                           inBestPeriodIndices,inAnalysisFreq,...
                           inRescalePatterns,inPlotFlag);

\end{verbatim}
\textbf{Description:}
\begin{verbatim}  Extracts the best pattern from the original signal using the results
  of an IPEMMECAnalysis run.

\end{verbatim}
\textbf{Input arguments:}
\begin{verbatim}  inAnalyzedSignal = (multi-channel) signal that was used for IPEMMECAnalysis
  inSampleFreq = sample frequency of inAnalyzedSignal (in Hz)
  inPeriods = column vector with the periods that were evaluated (in s)
  inBestPeriodIndices = 2D matrix containing the indices in inPeriods of the
                        best periods (each row represents a channel)
                        (if inBestPeriodIndices is single-channel while
                        inAnalyzedSignal is multi-channel, inBestPeriodIndices
                        is used for each channel)
  inAnalysisFreq = sample frequency of the IPEMMECAnalysis output (in Hz)
  inRescalePatterns = if non-zero, the patterns are rescaled between 0 and 1
                      if empty or not specified, 0 is used by default
  inPlotFlag = if non-zero, plots are generated
               if empty or not specified, 0 is used by default

\end{verbatim}
\textbf{Output:}
\begin{verbatim}  outPatterns = cell array containing the extracted patterns for each channel
                at each moment in time
                (outPatterns{i}(j,k) represents the j-th sample of the pattern
                extracted at the time corresponding with index k for channel i)
  outPatternLengths = 2D array containing the length of the patterns in
                      outPatterns (in samples) (each row represents a channel)
  outPatternFreq = sample frequency of outPatterns (same as inAnalysisFreq)
                   (each pattern itself is of course sampled at inSampleFreq)

\end{verbatim}
\textbf{Remarks:}
\begin{verbatim}  The input arguments inPeriods, inBestPeriodIndices and inAnalysisFreq are
  direct results of IPEMMECAnalysis and IPEMMECFindBestPeriods.

\end{verbatim}
\textbf{Example:}
\begin{verbatim}  [P,PLengths,PFreq] = ...
    IPEMMECExtractPatterns(RMS,RMSFreq,Periods,BestPeriodIndices,...
                           AnalysisFreq,0,1);

\end{verbatim}
\textbf{Authors:}
\begin{verbatim}  Koen Tanghe - 20010925
\end{verbatim}


\newpage
\section*{\hyperlink{Concepts:IPEMMECFindBestPeriods}{IPEMMECFindBestPeriods}\sectionmark{IPEMMECFindBestPeriods}}
\addcontentsline{toc}{subsection}{IPEMMECFindBestPeriods}
\hypertarget{FuncRef:IPEMMECFindBestPeriods}{}

\textbf{Usage:}
\begin{verbatim}  [outBestPeriodIndices,outSampleFreq] = ...
    IPEMMECAnalysis(inValues,inValuesFreq,inPeriods,inPlotFlag,...
                    inMergeType,inDecisionType)

\end{verbatim}
\textbf{Description:}
\begin{verbatim}  Finds the best matching periods corresponding to the given values obtainded
  from IPEMMECAnalysis.

\end{verbatim}
\textbf{Input arguments:}
\begin{verbatim}  inValues = cell array containing difference values from IPEMMECAnalysis
  inValuesFreq = sample frequency for inValues
  inPeriods = periods corresponding to the rows of the elements in inValues
              (in s)
  inPlotFlag = if non-zero, plot will be generated
               if empty or not specified, 0 is used by default
  inMergeType = type of merging in case of multi-channel input
                supported types are:
                'sum' = decision is made on sum of given values
                'separate' = decision is made for each channel separately
                if empty or not specified, 'sum' is used by default
  inDecisionType = type of decision used to find the best matching periods
                   supported types are:
                   'min' = period corresponding to smallest difference value
                   if empty or not specified, 'min' is used by default

\end{verbatim}
\textbf{Output:}
\begin{verbatim}  outBestPeriodIndices = indices corresponding to the best matching periods
                         (so Periods(outBestPeriodIndices(i,j),1) is the best
                         matching period for channel i at moment j, where
                         Periods is the row matrix from IPEMMECAnalysis)
  outSampleFreq = sample frequency of outBestPeriodIndices

\end{verbatim}
\textbf{Remarks:}
\begin{verbatim}  For a single channel inValues, the merge type has no influence of course.

\end{verbatim}
\textbf{Example:}
\begin{verbatim}  [Best,BestFreq] = IPEMMECAnalysis(Values,ValuesFreq,Periods,1);

\end{verbatim}
\textbf{Authors:}
\begin{verbatim}  Koen Tanghe - 20010225
\end{verbatim}


\newpage
\section*{\hyperlink{Concepts:IPEMMECReSynthUI}{IPEMMECReSynthUI}\sectionmark{IPEMMECReSynthUI}}
\addcontentsline{toc}{subsection}{IPEMMECReSynthUI}
\hypertarget{FuncRef:IPEMMECReSynthUI}{}

\textbf{Usage:}
\begin{verbatim}  IPEMMECReSynthUI(inAction,inData)

\end{verbatim}
\textbf{Description:}
\begin{verbatim}  User Interface callback function for interactively handling the resynthesis
  of MEC analysis results.

\end{verbatim}
\textbf{Input arguments:}
\begin{verbatim}  inAction = string representing the type of action to take
             (should be 'Start' when launched)
  inData = struct containing the data needed by the interface
           contains the following fields:
             Sound = original sound
             SampleFreq = sample frequency of Sound
             Periods = periods that were analyzed
             BestPeriodIndices = indices for detected best periods
             AnalysisFreq = sample frequency of BestPeriodIndices
             P = extracted patterns (see remark below)
             PLengths = lenghts of the patterns in P
             PFreq = sample frequency of P (same as AnalysisFreq)
             OriginalFreq = sample frequency of contents of P
             NoiseBands = noise bands to use for resynthesis
             ANIFilterFreqs = filter frequencies
                              (empty in case of non-ANI data)

\end{verbatim}
\textbf{Remarks:}
\begin{verbatim}  1. For more info on the data stored in inData, see input and output
     arguments of the functions IPEMMECAnalysis, IPEMMECExtractPatterns,
     IPEMMECSynthesis.
  2. For the 'P' field, you have to specify P between curly braces:
     DataStruct = struct('Sound',s,'SampleFreq',fs,....,'P',{P},...);
     This is because of Matlab's way of handling cell arrays in struct
     definitions.

\end{verbatim}
\textbf{Example:}
\begin{verbatim}  UIData = struct('Sound',s,'SampleFreq',fs,'Periods',Periods,...
                  'BestPeriodIndices',BestPeriodIndices,'AnalysisFreq',Freq,...
                  'P',{P},'PLengths',PLengths,'PFreq',PFreq,'OriginalFreq',...
                  RMSFreq,'NoiseBands',NoiseBands,'ANIFilterFreqs',[]);
  IPEMMECReSynthUI('Start',UIData);

\end{verbatim}
\textbf{Authors:}
\begin{verbatim}  Koen Tanghe - 20010925
\end{verbatim}


\newpage
\section*{\hyperlink{Concepts:IPEMMECSaveResults}{IPEMMECSaveResults}\sectionmark{IPEMMECSaveResults}}
\addcontentsline{toc}{subsection}{IPEMMECSaveResults}
\hypertarget{FuncRef:IPEMMECSaveResults}{}

\textbf{Usage:}
\begin{verbatim}  IPEMMECSaveResults(inFileName,inFilePath,inSound,inSampleFreq,...
                     inPeriods,inBestPeriodIndices,inAnalysisFreq,...
                     inPatterns,inPatternLengths,inPatternFreq,
                     inOriginalSignalFreq,inNoiseBands,inANIFilterFreqs,...
                     inAppend,inSuffix)

\end{verbatim}
\textbf{Description:}
\begin{verbatim}  Utility function that saves the results of IPEMMECExtractPatterns, so that
  resynthesis can still be done at a later date, after reloading this data.

\end{verbatim}
\textbf{Input arguments:}
\begin{verbatim}  inFileName = name of the .mat file to save the results to
  inFilePath = full path to the location where the file must be saved
               if empty, IPEMRootDir('output') is used by default
  inSound = original sound
  inSampleFreq = sample frequency of inSound (in Hz)
  inPeriods = (see IPEMMECExtractPatterns)
  inBestPeriodIndices = (idem)
  inAnalysisFreq = (idem)
  inPatterns = (see IPEMMECSynthesis)
  inPatternLengths = (idem)
  inPatternFreq = (idem)
  inOriginalSignalFreq = (idem)
  inNoiseBands = (idem)
  inANIFilterFreqs = ANI filter frequencies (empty if non-ANI data)
  inAppend = if non-zero, the data is appended to the data already in the file
             if zero, empty or not specified, all old data is cleared
  inSuffix = suffix that should be added to the general names
             (this facilitates saving multiple result sets)
             if empty or not specified, no suffix is used by default

\end{verbatim}
\textbf{Remarks:}
\begin{verbatim}  1. For generality, the given input arguments are saved under the following
     names:
     'Sound','SampleFreq',Periods','BestPeriodIndices','AnalysisFreq','P',
     'PLengths','PFreq','OrginalFreq','NoiseBands' and 'ANIFilterFreqs'.
     A suffix can be added to these names if inSuffix is given.
  2. You can later reload the data and perform another resynthesis using
     IPEMMECSynthesis like shown in the example below

\end{verbatim}
\textbf{Example:}
\begin{verbatim}  IPEMMECSaveResults('MEC_ligeti.mat','e:\temp',s,fs,Periods,...
                     BestPeriodIndices,AnalysisFreq,P,PLengths,PFreq,...
                     RMSFreq,NoiseBands,ANIFilterFreqs);
  ... 
  DataStruct = load('e:\temp\MEC_ligeti.mat');
  IPEMMECReSynthUI('Start',DataStruct);

\end{verbatim}
\textbf{Authors:}
\begin{verbatim}  Koen Tanghe - 20010228
\end{verbatim}


\newpage
\section*{\hyperlink{Concepts:IPEMMECSynthesis}{IPEMMECSynthesis}\sectionmark{IPEMMECSynthesis}}
\addcontentsline{toc}{subsection}{IPEMMECSynthesis}
\hypertarget{FuncRef:IPEMMECSynthesis}{}

\textbf{Usage:}
\begin{verbatim}  [outResynthesizedSound,outDetectedPeriods,outIndividualSounds] = ...
    IPEMMECSynthesis(inPatterns,inPatternLengths,inPatternFreq,...
                     inOriginalSignalFreq,inSelectionTime,inDuration,...
                     inOutputFreq,inNoiseBands,indBLevel,inEnhanceContrast,...
                     inReSynthWidth,inPlotFlag);

\end{verbatim}
\textbf{Description:}
\begin{verbatim}  Generates an AM modulated noise signal constructed from a repetition of the
  pattern found at the specified time moment.

\end{verbatim}
\textbf{Input arguments:}
\begin{verbatim}  inPatterns = the patterns extracted from the original signal
  inPatternLengths = the lengths of the patterns in inPatterns (in samples) 
  inPatternFreq = sample frequency for inPatterns (in Hz)
  inOriginalSignalFreq = sample frequency of the contents of the patterns
                         (in Hz)
  inSelectionTime = moment in time specifying the pattern to be used (in s)
  inDuration = wanted duration of the synthesized sound (in s)
               if empty or not specified, 3 times the maximum pattern length
               is used by default
  inOutputFreq = wanted sample frequency (in Hz)
                 if empty or not specified, 22050 is used by default
  inNoiseBands = 2 column matrix with wanted noise bands, one row per channel
                 (in Hz)
                 if empty or not specified, broad band noise is used by default
  indBLevel = wanted dB level for the resynthesized sound (in dB)
              if empty or not specified, -20 dB is used by default
  inEnhanceContrast = if non-zero, the pattern used for modulation of a noise
                      band is first reshaped according to:
                      ((P/max(abs(P))).^3)*max(abs(P))
                      if empty or not specified, 0 is used by default
  inReSynthWidth = if non-zero, specifies the width of one piece of
                   resynthesized noise in samples
                   (see IPEMGenerateBandPassedNoise for more details)
                   if empty or not specified, the entire wanted duration
                   is used by default
  inPlotFlag = if non-zero, plot are generated
               if empty or not specified, 0 is used by default

\end{verbatim}
\textbf{Output:}
\begin{verbatim}  outResynthesizedSound = the resynthesized sound
  outDetectedPeriods = duration of the periodic patterns that were used in
                       each channel (in s)
  outIndividualSounds = resynthesized sounds per channel

\end{verbatim}
\textbf{Example:}
\begin{verbatim}  [ss,periods,sschan] = IPEMMECSynthesis(P,PL,PFreq,RMSFreq,3,10,22050);

\end{verbatim}
\textbf{Authors:}
\begin{verbatim}  Marc Leman - 20000907 (original prototype version)
  Koen Tanghe - 20010926 (code generalization + documentation)
\end{verbatim}


\newpage
\section*{\hyperlink{Concepts:IPEMNormalize}{IPEMNormalize}\sectionmark{IPEMNormalize}}
\addcontentsline{toc}{subsection}{IPEMNormalize}
\hypertarget{FuncRef:IPEMNormalize}{}

\textbf{Usage:}
\begin{verbatim}  outSignal = IPEMNormalize (inSignal, inNormalizeSeparately)

\end{verbatim}
\textbf{Description:}
\begin{verbatim}  This function normalizes the given (multi-channel) signal, so that all
  values are in the range [-1,1]. Channels are represented by rows.

\end{verbatim}
\textbf{Input arguments:}
\begin{verbatim}  inSignal = the signal to be normalized (each row represents a channel)
  inNormalizeSeparately = if non-zero, each channel is normalized separately
                          if zero, empty or not specified, all channels are
                          multiplied with the same normalisation factor,
                          so that channel level ratio's are conserved

\end{verbatim}
\textbf{Output:}
\begin{verbatim}  outSignal = the normalized signal

\end{verbatim}
\textbf{Example:}
\begin{verbatim}  Signal = IPEMNormalize(Signal);

\end{verbatim}
\textbf{Authors:}
\begin{verbatim}  Koen Tanghe - 20000919
\end{verbatim}


\newpage
\section*{\hyperlink{Concepts:IPEMOnsetPattern}{IPEMOnsetPattern}\sectionmark{IPEMOnsetPattern}}
\addcontentsline{toc}{subsection}{IPEMOnsetPattern}
\hypertarget{FuncRef:IPEMOnsetPattern}{}

\textbf{Usage:}
\begin{verbatim}  [outOnsetPattern,outOnsetPatternFreq] = ...
    IPEMOnsetPattern(inSignal,inSampleFreq)

\end{verbatim}
\textbf{Description:}
\begin{verbatim}  Integrate-and-fire neural net for onset-detection.
  Based on an article by Leslie S. Smith (1996)
  Neuron dynamics are given by:
      dA/dt = I(t) - diss*A       with:  A = neurons' accumulated value
                                         I = input
                                         diss = dissipation 
\end{verbatim}
\textbf{Input arguments:}
\begin{verbatim}  inSignal = a matrix of size [N M], where N = the number of channels
             and M = the number of samples
  inSampleFreq = a scalar representing the frequency at which the signal was
                 sampled (in Hz)

\end{verbatim}
\textbf{Output:}
\begin{verbatim}  outOnsetPattern = a matrix of size [N M] where an element is 1 if an onset
                    was detected
  outOnsetPatternFreq = sample frequency of result (same as inSampleFreq)

\end{verbatim}
\textbf{Example:}
\begin{verbatim}  [OnsetPattern,OnsetPatternFreq] = IPEMOnsetPattern(Signal,SampleFreq);

\end{verbatim}
\textbf{Authors:}
\begin{verbatim}  Koen Tanghe - 20000510
\end{verbatim}


\newpage
\section*{\hyperlink{Concepts:IPEMOnsetPatternFilter}{IPEMOnsetPatternFilter}\sectionmark{IPEMOnsetPatternFilter}}
\addcontentsline{toc}{subsection}{IPEMOnsetPatternFilter}
\hypertarget{FuncRef:IPEMOnsetPatternFilter}{}

\textbf{Usage:}
\begin{verbatim}  [outOnsetSignal,outOnsetSignalFreq] = ...
    IPEMOnsetPatternFilter (inOnsetPattern, inSampleFreq)

\end{verbatim}
\textbf{Description:}
\begin{verbatim}  This function takes an (multi-channel) OnsetPattern generated by 
  IPEMOnsetPattern, and outputs a (one-dimensional) signal in which a non-zero
  value means an onset occurred (the value is a measurement of the likeliness
  for the onset).

  Currently, the following (simple) strategy is followed:
  an onset is detected at a certain moment, if:
  1. a certain fraction of channels have an onset within a specific period of
     time
  2. the moment falls at least a minimum period of time behind the last
     detected onset

\end{verbatim}
\textbf{Input arguments:}
\begin{verbatim}  inOnsetPattern = a matrix of size [N M],
                   where N = the number of channels and
                         M = the number of samples
  inSampleFreq = sample frequency of the input signal (in Hz)

\end{verbatim}
\textbf{Output:}
\begin{verbatim}  outOnsetSignal = a 1D signal having a non-zero value at detected onset-times
                   (the clearer the onset, the higher the value)
  outOnsetSignalFreq = sample frequency of the result (same as inSampleFreq)

\end{verbatim}
\textbf{Example:}
\begin{verbatim}  [OnsetSignal,OnsetSignalFreq] = ...
    IPEMOnsetPatternFilter(OnsetPattern,SampleFreq);

\end{verbatim}
\textbf{Authors:}
\begin{verbatim}  Koen Tanghe - 20031209
\end{verbatim}


\newpage
\section*{\hyperlink{Concepts:IPEMOnsetPeakDetection}{IPEMOnsetPeakDetection}\sectionmark{IPEMOnsetPeakDetection}}
\addcontentsline{toc}{subsection}{IPEMOnsetPeakDetection}
\hypertarget{FuncRef:IPEMOnsetPeakDetection}{}

\textbf{Usage:}
\begin{verbatim}  [outOnsetResults,outOnsetResultsFreq] = ...
    IPEMOnsetPeakDetection (inSignal,inSampleFreq,inPlotFlag)

\end{verbatim}
\textbf{Description:}
\begin{verbatim}  Returns a pattern having a non-zero value on moments of possible onset peaks

\end{verbatim}
\textbf{Input arguments:}
\begin{verbatim}  inSignal = a matrix of size [N,M] where N is the number of channels and
             M is the length in samples
  inSampleFreq = sample frequency of the incoming signal (in Hz)
  inPlotFlag = if non-zero, plots are generated
               if empty or not specified, 0 is used by default

\end{verbatim}
\textbf{Output:}
\begin{verbatim}  outOnsetResults = a matrix of size [N,M] with a value between 0 and 1
                    (non-zero where an onset occurs, and proportional to the 
                    importance of the onset)
  outOnsetResultsFreq = sample frequency of the result (same as inSampleFreq)

\end{verbatim}
\textbf{Example:}
\begin{verbatim}  [OnsetResults,OnsetResultsFreq] = IPEMOnsetPeakDetection(Signal,SampleFreq);

\end{verbatim}
\textbf{Authors:}
\begin{verbatim}  Koen Tanghe - 20010122
\end{verbatim}


\newpage
\section*{\hyperlink{Concepts:IPEMOnsetPeakDetection1Channel}{IPEMOnsetPeakDetection1Channel}\sectionmark{IPEMOnsetPeakDetection1Channel}}
\addcontentsline{toc}{subsection}{IPEMOnsetPeakDetection1Channel}
\hypertarget{FuncRef:IPEMOnsetPeakDetection1Channel}{}

\textbf{Usage:}
\begin{verbatim}  [outPeakIndices, outImportances] = ...
    IPEMOnsetPeakDetection1Channel(inSignal,inSampleFreq)

\end{verbatim}
\textbf{Description:}
\begin{verbatim}  Returns the indices of the "important" peaks in the input signal,
  together with an indication of the "importance" of the peak.

\end{verbatim}
\textbf{Input arguments:}
\begin{verbatim}  inSignal = the signal to be scanned for peaks
  inSampleFreq = the sample frequency of the signal (in Hz)

\end{verbatim}
\textbf{Output:}
\begin{verbatim}  outPeakIndices = indices within the input signal where the peaks are found
  outImportance = value between 0 and 1 showing the "importance" of a peak

\end{verbatim}
\textbf{Example:}
\begin{verbatim}  [PeakIndices,Importances] = ...
    IPEMOnsetPeakDetection1Channel(Signal,SampleFreq);

\end{verbatim}
\textbf{Authors:}
\begin{verbatim}  Koen Tanghe - 20000510
\end{verbatim}


\newpage
\section*{\hyperlink{Concepts:IPEMPeriodicityPitch}{IPEMPeriodicityPitch}\sectionmark{IPEMPeriodicityPitch}}
\addcontentsline{toc}{subsection}{IPEMPeriodicityPitch}
\hypertarget{FuncRef:IPEMPeriodicityPitch}{}

\textbf{Usage:}
\begin{verbatim}  [outSignal,outSampleFreq,outPeriods,outFANI] = IPEMPeriodicityPitch ...
                      (inMatrix,inSampleFreq,inLowFrequency,...
                       inFrameWidth,inFrameStepSize,inPlotFlag)

\end{verbatim}
\textbf{Description:}
\begin{verbatim}  This function calculates the periodicity pitch of a matrix containing 
  neural firing patterns in different auditory channels. 
  It is based on the idea that periodicity pitch is calculated as the summed
  autocorrelation over bandpass filtered fluctuations in auditory channels. 
  Due to the fact that the output of the auditory model gives the envelopes
  of the neural firing probabilities (< 1250 Hz) it suffices to use a low-pass
  filter in order to obtain the pitch e.g. between 80 - 1250 Hz.
  Only the lowest frequency can be changed by inLowFrequency.

\end{verbatim}
\textbf{Input arguments:}
\begin{verbatim}  inMatrix = the input signal is a matrix of size [N M] where
               N is the number of auditory channels (<= 40) and 
               M is the number of samples
             the input signal is the output of the auditory model (ANI).
  inSampleFreq = sample frequency of the input signal (in Hz)
  inLowFrequency = cutoff frequency (in Hz) of a first order lowpass filter
                   applied before calculating the autocorrelation
                   if empty or not specified, 80 is used by default
  inFrameWidth = width of the frame used for the accumulation of the
                 autocorrelation (in s)
                 if empty or not specified, 0.064 is used by default
  inFrameStepSize = stepsize or timeinterval between two inFrameWidths (in s)
                  if empty or not specified, 0.010 is used by default
  inPlotFlag = if non-zero, plots are generated
               if empty or not specified, 0 is used by default

\end{verbatim}
\textbf{Output:}
\begin{verbatim}  outSignal = periodicity pitch: a matrix of size
              inFrameWidth * length(inMatrix) / outSampleFreq 
  outSampleFreq = sampling rate, equal to inSampleFreq/inFrameStepSize (in Hz)
  outPeriods = analyzed periods (in s)
  outFANI = bandpass filtered auditory nerve images (at the original sample
            frequency)

\end{verbatim}
\textbf{Remarks:}
\begin{verbatim}  As for any frame-based function, the first value of the output signal is the
  value calculated for the first complete frame in the input signal.
  This means the following:
  if you have an input signal of length 1 s at a sample frequency of 1000 Hz,
  a frame width of 0.050 s, and a frame step size of 0.010 s,
  then there will be ceil(((1 - 0.050)*1000 + 1)/(0.010*1000)) = 96 values in
  the output signal, where the first value corresponds to the first complete
  frame (the interval 0 to 0.050 s)

\end{verbatim}
\textbf{Example:}
\begin{verbatim}  [PP,PPFreq,PPPeriods,PPFANI] = IPEMPeriodicityPitch(ANI,ANIFreq);

\end{verbatim}
\textbf{Authors:}
\begin{verbatim}  Marc Leman - 20011204 (originally made)
  Koen Tanghe - 20050119
\end{verbatim}


\newpage
\section*{\hyperlink{Concepts:IPEMPlaySoundWithCursor}{IPEMPlaySoundWithCursor}\sectionmark{IPEMPlaySoundWithCursor}}
\addcontentsline{toc}{subsection}{IPEMPlaySoundWithCursor}
\hypertarget{FuncRef:IPEMPlaySoundWithCursor}{}

\textbf{Usage:}
\begin{verbatim}  IPEMPlaySoundWithCursor(inSignal,inSignalFreq,inAxisHandles,inTimeRange,
                          inTimeOffset,inSpeedFactor)

\end{verbatim}
\textbf{Description:}
\begin{verbatim}  Plays the given sound signal while showing a moving cursor in the given axes.

\end{verbatim}
\textbf{Input arguments:}
\begin{verbatim}  inSignal = Sound signal.
  inSignalFreq = Sample frequency of inSignal (in Hz).
  inAxisHandles = Handles of the axes for which a pointer should be shown.
                  If empty or not specified, all axes of the current figure
                  will be used by default (if no current figure, -1 is used).
                  If -1, the sound file is drawn in a new figure.
  inTimeRange = Range of playback in the form [starttime endtime] (in s).
                If emtpy or not specified, the entire signal will be used.
  inTimeOffset = Time offset of first sample (in s). This is useful when the
                 given signal is part of a larger signal.
                 If empty or not specified, 0 is used by default.
  inSpeedFactor = Factor by which the sound should be sped up (>1) or slowed
                  down (<1). If empty or not specified, 1 is used by default.

\end{verbatim}
\textbf{Remarks:}
\begin{verbatim}  The plots associated with the given handles are supposed to have an x-axis
  that has a time scale in seconds.

\end{verbatim}
\textbf{Example:}
\begin{verbatim}  IPEMCalcRMS(s,fs,0.040,0.020,1);
  IPEMPlaySoundWithCursor(s,fs);

\end{verbatim}
\textbf{Authors:}
\begin{verbatim}  Koen Tanghe - 20040330
\end{verbatim}


\newpage
\section*{\hyperlink{Concepts:IPEMPlotMultiChannel}{IPEMPlotMultiChannel}\sectionmark{IPEMPlotMultiChannel}}
\addcontentsline{toc}{subsection}{IPEMPlotMultiChannel}
\hypertarget{FuncRef:IPEMPlotMultiChannel}{}

\textbf{Usage:}
\begin{verbatim}  IPEMPlotMultiChannel(inData,inSampleFreq,inTitle,inXLabel,inYLabel,...
                       inFontSize,inChannelLabels,inChannelLabelStep,...
                       inMinY,inMaxY,inType,inTimeOffset,inYScaleFactor)

\end{verbatim}
\textbf{Description:}
\begin{verbatim}  Plots a multi-channel signal.

\end{verbatim}
\textbf{Input arguments:}
\begin{verbatim}  inData = the multi-channel signal (each row represents a channel)
  inSampleFreq = sample frequency of the data (in Hz)
                 if empty or not specified, 1 is used by default
  inTitle = title for the plot
            if empty or not specified, '' is used by default
  inXLabel = name for the X-values
             if empty or not specified, '' is used by default
  inYLabel = name for the Y-values
             if empty or not specified, '' is used by default
  inFontSize = size of the font to be used for title (size of axes labels will
               be set to inFontSize-2)
               if empty or not specified, no special fontsize is set by default
  inChannelLabels = labels for the rows of the multi-channel data
                    if empty or not specified, 1:Rows(inData) is used
                    by default
  inChannelLabelStep = step size for displaying the labels of the channels:
                       labels 1:inChannelLabelStep:NumOfRows will be used
                       as ticks
                       if -1, the initial automatic tick division is kept
                       if empty or not specified, 1 is used by default
  inMinY = minimum Y-value to be used for scaling the line plot of a channel
           if empty or not specified min(min(inData)) is used by default
  inMaxY = maximum Y-value to be used for scaling the line plot of a channel
           if empty or not specified max(max(inData)) is used by default
  inType = plot type to use:
             0 : uses line plots for each channel, starting from tick label
             1 : uses imagesc
             2 : uses line plots for each channel, centered around tick label
           if empty or not specified, 0 is used by default
  inTimeOffset = time offset to use for the first sample (in s): the first
                 sample will be assumed to be taken at time inTimeOffset
                 if empty or not specified, 0 is used by default
  inYScaleFactor = factor to be used on the Y values (can be used to scale the 
                   range of Y values to better show the signal dynamics)
                   if empty or not specified, 1 is used by default

\end{verbatim}
\textbf{Output:}
\begin{verbatim}  A graph in the current subplot of the current figure.

\end{verbatim}
\textbf{Example:}
\begin{verbatim}  IPEMPlotMultiChannel(ANI,ANIFreq,'Auditory nerve image','Time (in s)',...
       'Auditory channels (center freqs. in Hz)',14,ANIFilterFreqs,3);

\end{verbatim}
\textbf{Authors:}
\begin{verbatim}  Koen Tanghe - 20040322
\end{verbatim}


\newpage
\section*{\hyperlink{Concepts:IPEMReadSoundFile}{IPEMReadSoundFile}\sectionmark{IPEMReadSoundFile}}
\addcontentsline{toc}{subsection}{IPEMReadSoundFile}
\hypertarget{FuncRef:IPEMReadSoundFile}{}

\textbf{Usage:}
\begin{verbatim}  [outSignal,outSampleFreq,outSize,outFile] = ...
    IPEMReadSoundFile(inFileName,inFilePath,inSection,inPlotFlag)

\end{verbatim}
\textbf{Description:}
\begin{verbatim}  Reads (part of) a sound file into memory.

\end{verbatim}
\textbf{Input arguments:}
\begin{verbatim}  inFileName = name of the sound file (with extension)
               if empty or not specified, the standard "open file" dialog box
               can be used to select a sound file (if the user then presses
               'Cancel', all outputs will be empty)
  inFilePath = path to the sound file (if it does not exist, the current
               directory is used)
               use the empty string '' if inFilename already contains a full 
               path specification (no extra path will be prepended)
               if empty or not specified, IPEMRootDir('input')\Sounds is used
               by default (if it exists)
  inSection = section of the sound file to read, given as [start end] (in s)
              if 'size', no samples are read: outSignal will be empty, and
              only outSampleFreq and outSize will have relevant values
              if empty or not specified, the entire sound file is read
  inPlotFlag = if non-zero, a plot is generated
               if empty or not specified, 0 is used by default

\end{verbatim}
\textbf{Output:}
\begin{verbatim}  outSignal = the sound signal (each row represents a channel)
  outSampleFreq = the frequency at which the sound was sampled (in Hz)
  outSize = the size of the entire sound file returned as
            [NumSamples NumChannels] (optional)
            where NumSamples = number of samples in each channel
                  NumChannels = number of channels
  outFile = entire path specification of the sound file

\end{verbatim}
\textbf{Remarks:}
\begin{verbatim}  The 'end' in the section specification is exclusive:
  if you want to read the first second of a sound file sampled at 1000 Hz,
  you should specify [0 1] for inSection (and not [0 0.999]). This will read
  samples 1 to 1000, which is a full second (the sample at time 1 s is not
  read!).

\end{verbatim}
\textbf{Example:}
\begin{verbatim}  [s,fs] = IPEMReadSoundFile('schum1.wav');

\end{verbatim}
\textbf{Authors:}
\begin{verbatim}  Koen Tanghe - 20040409
\end{verbatim}


\newpage
\section*{\hyperlink{Concepts:IPEMReplaceSubStringInFileNames}{IPEMReplaceSubStringInFileNames}\sectionmark{IPEMReplaceSubStringInFileNames}}
\addcontentsline{toc}{subsection}{IPEMReplaceSubStringInFileNames}
\hypertarget{FuncRef:IPEMReplaceSubStringInFileNames}{}

\textbf{Usage:}
\begin{verbatim}  IPEMReplaceSubStringInFileNames(inDir,inOriginalString,inNewString,...
                                  inRecursive,inAffectExtensions)

\end{verbatim}
\textbf{Description:}
\begin{verbatim}  Replaces a substring in the filenames of the files contained in a specific
  directory (and its subdirectories if needed) by a new substring.

\end{verbatim}
\textbf{Input arguments:}
\begin{verbatim}  inDir = full directory path to the root directory for which the changes
          should be applied
  inOriginalString = substring to be replaced in the filenames
                     if empty or not specified, '_' is used by default
  inNewString = substring by which inOriginalString will be replaced
                if empty or not specified, '' is used by default
  inRecursive = if non-zero, subdirectories are also affected
                if empty or not specified, 0 is used by default
  inAffectExtensions = if non-zero, extensions of filenames are also affected
                       if empty or not specified, 0 is used by default

\end{verbatim}
\textbf{Example:}
\begin{verbatim}  % Remove all underscores in the filenames (not the extensions) of all files
  % under the given directory and its subdirectories
  IPEMReplaceSubStringInFileNames('D:\Koen\Temp\Test','_','',1,0);

\end{verbatim}
\textbf{Authors:}
\begin{verbatim}  Koen Tanghe - 20010522
\end{verbatim}


\newpage
\section*{\hyperlink{Concepts:IPEMRescale}{IPEMRescale}\sectionmark{IPEMRescale}}
\addcontentsline{toc}{subsection}{IPEMRescale}
\hypertarget{FuncRef:IPEMRescale}{}

\textbf{Usage:}
\begin{verbatim}  outSignal = IPEMRescale (inSignal,inLowLevel,inHighLevel,inScaleGlobally)

\end{verbatim}
\textbf{Description:}
\begin{verbatim}  This function rescales each channel of the incoming signal between the given
  lower and upper bounds. Signal channels are represented by rows.

\end{verbatim}
\textbf{Input arguments:}
\begin{verbatim}  inSignal = the signal that is to be rescaled (each row represents a channel)
  inLowLevel = the new value for the lowest value in the incoming signal
               if empty or not specified, 0 is used by default
  inHighLevel = the new value for the highest value in the incoming signal
                if empty or not specified, 1 is used by default
  inScaleGlobally = if non-zero, all channels are scaled in exactly the
                    same way
                    if zero, each channel is scaled separately
                    if empty or not specified, 0 is used by default

\end{verbatim}
\textbf{Output:}
\begin{verbatim}  outSignal = the rescaled signal

\end{verbatim}
\textbf{Remarks:}
\begin{verbatim}  1. For constant signals, inLowLevel is used as constant output level
  2. You can specify a value for inLowLevel that is higher than the
     inHighLevel, in which case the signal will be inverted and scaled between
     the two given levels.

\end{verbatim}
\textbf{Example:}
\begin{verbatim}  Signal = IPEMRescale(Signal,-1,1);

\end{verbatim}
\textbf{Authors:}
\begin{verbatim}  Koen Tanghe - 20000224
\end{verbatim}


\newpage
\section*{\hyperlink{Concepts:IPEMReshape}{IPEMReshape}\sectionmark{IPEMReshape}}
\addcontentsline{toc}{subsection}{IPEMReshape}
\hypertarget{FuncRef:IPEMReshape}{}

\textbf{Usage:}
\begin{verbatim}  outSignal = IPEMReshape(inData,inShaper,inMethod,inSymmetry,inPlotFlag)

\end{verbatim}
\textbf{Description:}
\begin{verbatim}  Reshapes the incoming data by substituting each original value by a 
  new value using the given shaper matrix.

\end{verbatim}
\textbf{Input arguments:}
\begin{verbatim}  inData = 1D data vector to be reshaped
  inShaper = shaper matrix: two column matrix of which the first column
             represents the original value and the second column represents
             its substituted value
  inMethod = string with the method to be used for interpolating substituded
             values (see description of methods in Matlab's interp1 function)
             if empty or not specified, 'linear' is used by default
  inSymmetry = if 1, inShaper is assumed to be one half of a curve symmetric
               around the point zero: Y(-X) = -Y(X)
               if 2, inShaper is assumed to be one half of a curve symmetric
               around the Y-axis: Y(-X) = Y(X)
               if empty or not specified, no symmetry is assumed by default
  inPlotFlag = if non-zero, a plot is generated
               if empty or not specified, 0 is used by default

\end{verbatim}
\textbf{Output:}
\begin{verbatim}  outSignal = reshaped data

\end{verbatim}
\textbf{Remarks:}
\begin{verbatim}  Only values between the min(inShaper(:,1)) and max(inShaper(:,1)) can be
  substituted, so always make sure to include these bounds and their
  subsitutes in inShaper. 
  In case you make use of inSymmetry, only max(inShaper(:,1)) will play a role
  in defining the region where values can be substituted.

\end{verbatim}
\textbf{Example:}
\begin{verbatim}  X = 0:0.1:1;
  Y = X.^3;
  s2 = IPEMReshape(s,[X' Y'],'linear',1,1);

\end{verbatim}
\textbf{Authors:}
\begin{verbatim}  Koen Tanghe - 20010110
\end{verbatim}


\newpage
\section*{\hyperlink{Concepts:IPEMRippleFilter}{IPEMRippleFilter}\sectionmark{IPEMRippleFilter}}
\addcontentsline{toc}{subsection}{IPEMRippleFilter}
\hypertarget{FuncRef:IPEMRippleFilter}{}

\textbf{Usage:}
\begin{verbatim}  outSignal = IPEMRippleFilter(inSignal,inSampleFreq,inDelay,...
                               inAttenuation,inSync)

\end{verbatim}
\textbf{Description:}
\begin{verbatim}  Filters the incoming signal with a ripple filter according to the following
  scheme:

       OUT = IN / (1 + inAttenuation * z^(-m) )
       where m = round(inDelay*inSampleFreq)

  In the frequency domain, this produces a rippled frequency response,
  in which either the peaks (if inSync == 1) or the valleys (if inSync == 0)
  of the ripples are at frequencies n.F0 (F0 = 1/inDelay).

\end{verbatim}
\textbf{Input arguments:}
\begin{verbatim}  inSignal = input signal (each row represents a channel)
  inSampleFreq = input signal sample frequency (in Hz)
  inDelay = delay of feedback branch (in s)
  inAttenuation = attenuation ratio (0 to 1)
  inSync = if 1, the ripple's peaks are at frequencies n/delay
           if 0, the ripple's valleys are at frequencies n/delay

\end{verbatim}
\textbf{Output:}
\begin{verbatim}  outSignal = the filtered signal

\end{verbatim}
\textbf{Example:}
\begin{verbatim}  Signal = IPEMRippleFilter(Signal,SampleFreq,1/100,0.95,1);

\end{verbatim}
\textbf{Authors:}
\begin{verbatim}  Koen Tanghe - 20000209
\end{verbatim}


\newpage
\section*{\hyperlink{Concepts:IPEMRootDir}{IPEMRootDir}\sectionmark{IPEMRootDir}}
\addcontentsline{toc}{subsection}{IPEMRootDir}
\hypertarget{FuncRef:IPEMRootDir}{}

\textbf{Usage:}
\begin{verbatim}  outRootDirPath = IPEMRootDir(inDirType)

\end{verbatim}
\textbf{Description:}
\begin{verbatim}  This function returns a default root directory for the requested type.
  These directories are used by some of the IPEM functions.

\end{verbatim}
\textbf{Input arguments:}
\begin{verbatim}  inType = string specifying the type of info for which you want the standard
           directory (the supported types are defined below)

\end{verbatim}
\textbf{Output:}
\begin{verbatim}  outRootDirPath = the requested directory path

\end{verbatim}
\textbf{Remarks:}
\begin{verbatim}
  *** Description of supported types ***
  
  'input'
  The directory that will be used as the root of the subdirectories
  containing data to be processed by the IPEM Toolbox code.
  (e.g. sound files could be stored in a subdirectory called 'Sounds',
  located in this 'input' directory)

  'output'
  Same as 'input', but for output of the IPEM Toolbox code.
  In some special cases, the directory of the processed file is used (but this
  will always be explicitely mentioned in the comments of such a function)

  'code'
  This is the directory where the IPEM Matlab code is located and can NOT be
  set by setpref.
  (e.g. 'E:\Code\Matlab\IPEMToolbox')


  *** Supports preferences ***

  You can specify the default 'input' and 'output' root directories by typing:

     setpref('IPEMToolbox','RootDir_Input','directory1')

  and/or

     setpref('IPEMToolbox','RootDir_Output','directory2') 

  where directory1 and directory2 should be replaced by the complete paths to
  existing directories on your system.
  If you don't set these directory preferences, the 'Temp' subdirectory in
  your 'IPEMToolbox' directory is used by default.

\end{verbatim}
\textbf{Authors:}
\begin{verbatim}  Koen Tanghe - 20010117
\end{verbatim}


\newpage
\section*{\hyperlink{Concepts:IPEMRotateMatrix}{IPEMRotateMatrix}\sectionmark{IPEMRotateMatrix}}
\addcontentsline{toc}{subsection}{IPEMRotateMatrix}
\hypertarget{FuncRef:IPEMRotateMatrix}{}

\textbf{Usage:}
\begin{verbatim}  outMatrix = IPEMRotateMatrix (inMatrix,inRows,inColumns)

\end{verbatim}
\textbf{Description:}
\begin{verbatim}  Rotates a matrix along its rows and/or columns.

\end{verbatim}
\textbf{Input arguments:}
\begin{verbatim}  inMatrix = matrix to process
  inRows = number of rows by which to rotate
           positive values mean rotation towards increasing row numbers
  inColumns = number of columns by which to rotate
              positive values mean rotation towards increasing column numbers

\end{verbatim}
\textbf{Output:}
\begin{verbatim}  outMatrix = the rotated matrix

\end{verbatim}
\textbf{Example:}
\begin{verbatim}  RotMat = IPEMRotateMatrix([1 2 3; 4 5 6; 7 8 9],-1,2);

  --> RotMat == [5 6 4; 8 9 7; 2 3 1]

\end{verbatim}
\textbf{Authors:}
\begin{verbatim}  Koen Tanghe - 20000209
\end{verbatim}


\newpage
\section*{\hyperlink{Concepts:IPEMRoughnessFFT}{IPEMRoughnessFFT}\sectionmark{IPEMRoughnessFFT}}
\addcontentsline{toc}{subsection}{IPEMRoughnessFFT}
\hypertarget{FuncRef:IPEMRoughnessFFT}{}

\textbf{Usage:}
\begin{verbatim}  [outRoughness,outSampleFreq,outFFTMatrix1,outFFTMatrix2] =
     IPEMRoughnessFFT(inANI,inANIFreq,inANIFilterFreqs,...
                      inFrameWidth,inFrameStepSize,inPlotFlag)

\end{verbatim}
\textbf{Description:}
\begin{verbatim}  Based on Leman (2000) (paper on calculation and visualization of
  Synchronization & Roughness)
  This function calculates three outputs:
  1. The total energy
  2. The energy in all channels
  3. The energy spectrum of the neuronal synchronization 

\end{verbatim}
\textbf{Input arguments:}
\begin{verbatim}  inANI = auditory nerve image to process, where each row represents a channel
  inANIFreq = sample frequency of the input signal (in samples per second)
  inANIFilterFreqs = filterbank frequencies used by the auditory model
  inFrameWidth = the width of the window for analysing the signal (in s)
                 if empty or not specified, 0.2 s is used by default
  inFrameStepSize = the stepsize or time interval between two
                    inFrameWidthInSampless (in s)
                    if empty or not specified, 0.02 s is used by default
  inPlotFlag = if non-zero plots are generated
               if empty or not specified, 0 is used by default

\end{verbatim}
\textbf{Output:}
\begin{verbatim}  outRoughness = roughness over signal
  outSampleFreq = sampling rate of outRoughness (in Hz)
  outFFTMatrix1 = visualisation of energy over channels
  outFFTMatrix2 = visualisation of energy spectrum for synchronization
                  (synchronisation index SI)

\end{verbatim}
\textbf{Remarks:}
\begin{verbatim}  For now, the roughness values are dependend on the used frame width.
  So, to make usefull comparisons, only results obtained using the same frame
  width should be compared (this should be fixed in the future...)

\end{verbatim}
\textbf{Example:}
\begin{verbatim}  [Roughness,RoughnessFreq] = ...
     IPEMRoughnessFFT(ANI,ANIFreq,ANIFilterFreqs,0.20,0.02,1);

\end{verbatim}
\textbf{Authors:}
\begin{verbatim}  Marc Leman - 20000910 (made)
  Koen Tanghe - 20010816 (bug fix + documentation corrections + code cleanup)
\end{verbatim}


\newpage
\section*{\hyperlink{Concepts:IPEMRoughnessOfSoundPairs}{IPEMRoughnessOfSoundPairs}\sectionmark{IPEMRoughnessOfSoundPairs}}
\addcontentsline{toc}{subsection}{IPEMRoughnessOfSoundPairs}
\hypertarget{FuncRef:IPEMRoughnessOfSoundPairs}{}

\textbf{Usage:}
\begin{verbatim}  [outMeanRoughness,outSortedList,outRoughness,outRoughnessFreq,...
   outUsedSection] = ...
     IPEMRoughnessOfSoundPairs(inFiles,inPaths,inSection,...
                               inSaveSoundCombinations,inOutputPath,...
                               inUseReference,inPlotFlag)

\end{verbatim}
\textbf{Description:}
\begin{verbatim}  Calculates roughness of 2-by-2 combinations of sounds.

\end{verbatim}
\textbf{Input arguments:}
\begin{verbatim}  inFiles = set of sound files to be analyzed (cell array of strings)
  inPaths = paths for the sound files
            either a cell array of strings containing a path for each sound
            file, or a single character array containing one path for all
            sound files
            if empty or not specified, IPEMRootDir('input')\Sounds is used
            by default
  inSection = section of the sound to be used (given as [start end] in s,
              where Inf for the end parameter means the minimum of all sound
              file lengths)
              if empty or not specified, the (minimum of the) full length of
              the sounds is used by default
  inSaveSoundCombinations = if non-zero, the sound combinations are saved to
                            inOutputPath
                            if empty or not specified, 1 is used by default
  inOutputPath = path where the sound combinations should be stored
                 if empty or not specified, IPEMRootDir('output') is used
                 by default
  inUseReference = if non-zero, a reference value is used depending on the dB
                   level (see IPEMGetRoughnessFFTReference)
                   if empty or not specified, 0 is used by default 
  inPlotFlag = if non-zero, plots outRoughness as a 2D map
               if empty or not specified, 1 is used by default

\end{verbatim}
\textbf{Output:}
\begin{verbatim}  outMeanRoughness = mean of outRoughness over the complete section
  outSortedList = sound combinations sorted according to roughness:
                     column 1 = roughness
                     columns 2 and 3 = number of the first resp. second sound
  outRoughness = 3D matrix containing the calculated roughness over the
                 combinations of sounds, where element outRoughness(i,j,:) 
                 are the roughness values calculated over the combination
                 of sound i with sound j
  outRoughnessFreq = sample frequency of outRoughness (Hz)
  outUsedSection = section of the sound that was eventually used
                   (same as inSection, except when inSection(2) = Inf)

\end{verbatim}
\textbf{Remarks:}
\begin{verbatim}  1. The sound signals (say s1 and s2) are mixed as follows:
       s1 = IPEMAdaptLevel(s1,-20); s2 = IPEMAdaptLevel(s2,-20);
       s = IPEMAdaptLevel(s1+s2,-20);
  2. All sound files should be of the same sample frequency...

\end{verbatim}
\textbf{Example:}
\begin{verbatim}  theFiles = {'sound1.wav','sound2.wav','sound3.wav'};
  thePaths = {'E:\Koen\Data','e:\Koen\Data','e:\Koen\Data\Sounds\New'};
  [MeanRoughness,SortedList] = ...
    IPEMRoughnessOfSoundPairs(theFiles,thePaths,[0.2 Inf]);

\end{verbatim}
\textbf{Authors:}
\begin{verbatim}  Koen Tanghe - 20011004
\end{verbatim}


\newpage
\section*{\hyperlink{Concepts:IPEMSaveANI}{IPEMSaveANI}\sectionmark{IPEMSaveANI}}
\addcontentsline{toc}{subsection}{IPEMSaveANI}
\hypertarget{FuncRef:IPEMSaveANI}{}

\textbf{Usage:}
\begin{verbatim}  IPEMSaveANI(inANI,inANIFreq,inANIFilterFreqs,inName,inPath,...
              inANIName,inANIFreqName,inANIFilterFreqsName,inAppend)

\end{verbatim}
\textbf{Description:}
\begin{verbatim}  Saves an auditory nerve image and its corresponding sample frequency and
  filter frequencies to a .mat file on disk.

\end{verbatim}
\textbf{Input arguments:}
\begin{verbatim}  inANI = auditory nerve image array
  inANIFreq = sample frequency of the auditory nerve image
  inANIFilterFreqs = filter frequencies array
  inName = the name of the .mat file for storing the nerve image
           if empty or not specified, 'ANI.mat' is used by default
  inPath = path to the .mat file
           if empty or not specified, IPEMRootDir('code')\Temp is used
           by default
  inANIName = name for the auditory nerve image array in the mat file
              if empty or not specified, 'ANI' is used by default
  inANIFreqName = name for the sample frequency
                  if empty or not specified, 'ANIFreq' is used by default
  inANIFilterFreqsName = name for the filter frequencies array
                         if empty or not specified, 'ANIFilterFreqs' is used
                         by default
  inAppend = if 1, the variables are appended to the mat file (if it already
             exists) otherwise, a new mat file is created
             if empty or not specified, 0 is used by default

\end{verbatim}
\textbf{Example:}
\begin{verbatim}  IPEMSaveANI(ANI,ANIFreq,ANIFilterFreqs,'ANIs','c:\','Schum1','FreqSchum1',...
              'FilterFreqsSchum1');

\end{verbatim}
\textbf{Authors:}
\begin{verbatim}  Koen Tanghe - 20000515
\end{verbatim}


\newpage
\section*{\hyperlink{Concepts:IPEMSaveVar}{IPEMSaveVar}\sectionmark{IPEMSaveVar}}
\addcontentsline{toc}{subsection}{IPEMSaveVar}
\hypertarget{FuncRef:IPEMSaveVar}{}

\textbf{Usage:}
\begin{verbatim}  IPEMSaveVar (inFile,inName,inValue)

\end{verbatim}
\textbf{Description:}
\begin{verbatim}  Saves a variable with name inValue and value inValue to a mat file.
  If the file did not already exist, a new file is created first.

\end{verbatim}
\textbf{Input arguments:}
\begin{verbatim}  inFile = full name (with path) of the mat file (the .mat extension will be
           added automatically if it's not there already)
  inName = name for the variable
  inValue = value for the variable

\end{verbatim}
\textbf{Remarks:}
\begin{verbatim}  This thing is NOT COMPILABLE because of call stack/workspace differences
  between M-file evaluation and mex-evaluation !

\end{verbatim}
\textbf{Example:}
\begin{verbatim}  IPEMSaveVar('E:\Koen\Data\music.mat','NerveImage',ANI);

\end{verbatim}
\textbf{Authors:}
\begin{verbatim}  Koen Tanghe - 20000209
\end{verbatim}


\newpage
\section*{\hyperlink{Concepts:IPEMSetFigureLayout}{IPEMSetFigureLayout}\sectionmark{IPEMSetFigureLayout}}
\addcontentsline{toc}{subsection}{IPEMSetFigureLayout}
\hypertarget{FuncRef:IPEMSetFigureLayout}{}

\textbf{Usage:}
\begin{verbatim}  IPEMSetFigureLayout(inFigureHandle)

\end{verbatim}
\textbf{Description:}
\begin{verbatim}  Sets the layout of the figure to the IPEM defaults.

\end{verbatim}
\textbf{Input arguments:}
\begin{verbatim}  inFigureHandle = handle of the figure to be adjusted
                   if 'all' is specified, all open figures will be adjusted
                   if empty or not specified, the current figure will be
                   adjusted

\end{verbatim}
\textbf{Example:}
\begin{verbatim}  FigH = figure;
  % ... some plots ...
  IPEMSetFigureLayout(FigH);

\end{verbatim}
\textbf{Authors:}
\begin{verbatim}  Koen Tanghe - 20010129
\end{verbatim}


\newpage
\section*{\hyperlink{Concepts:IPEMSetup}{IPEMSetup}\sectionmark{IPEMSetup}}
\addcontentsline{toc}{subsection}{IPEMSetup}
\hypertarget{FuncRef:IPEMSetup}{}

\textbf{Usage:}
\begin{verbatim}  IPEMSetup

\end{verbatim}
\textbf{Description:}
\begin{verbatim}  This function sets up the Matlab environment for working with the
  IPEM Toolbox.

\end{verbatim}
\textbf{Example:}
\begin{verbatim}  IPEMSetup;

\end{verbatim}
\textbf{Authors:}
\begin{verbatim}  Koen Tanghe - 20011204
\end{verbatim}


\newpage
\section*{\hyperlink{Concepts:IPEMSetupPreferences}{IPEMSetupPreferences}\sectionmark{IPEMSetupPreferences}}
\addcontentsline{toc}{subsection}{IPEMSetupPreferences}
\hypertarget{FuncRef:IPEMSetupPreferences}{}

\textbf{Usage:}
\begin{verbatim}  IPEMSetupPreferences(inReset)

\end{verbatim}
\textbf{Description:}
\begin{verbatim}  This function sets up preferences for the IPEM Toolbox.

  Use getpref('IPEMToolbox','preference') to get the value of a preference.
  Use setpref('IPEMToolbox','preference',value) to set the value of a
  preference.

\end{verbatim}
\textbf{Input arguments:}
\begin{verbatim}  inReset = if non-zero, resets all preferences to their initial values
            if zero, initializes non-existing preferences, but keeps existing
            ones
            if empty or not specified, 0 is used by default

\end{verbatim}
\textbf{Remarks:}
\begin{verbatim}  List of currently supported preferences:

  'RootDir_Input'        - standard input directory
                         - initial value: IPEMToolbox\Temp
    
  'RootDir_Output'       - standard output directory
                         - initial value: IPEMToolbox\Temp

\end{verbatim}
\textbf{Example:}
\begin{verbatim}  IPEMSetupPreferences(1);

\end{verbatim}
\textbf{Authors:}
\begin{verbatim}  Koen Tanghe - 20010117
\end{verbatim}


\newpage
\section*{\hyperlink{Concepts:IPEMShepardTone}{IPEMShepardTone}\sectionmark{IPEMShepardTone}}
\addcontentsline{toc}{subsection}{IPEMShepardTone}
\hypertarget{FuncRef:IPEMShepardTone}{}

\textbf{Usage:}
\begin{verbatim}  outSignal = IPEMShepardTone(inMainFreq,inDuration,inSampleFreq,...
                              inPhaseFlag,indBLevel)

\end{verbatim}
\textbf{Description:}
\begin{verbatim}  This function generates a Shepard tone.

\end{verbatim}
\textbf{Input arguments:}
\begin{verbatim}  inMainFreq = the main frequency (in Hz)
  inDuration = the duration (in s)
               if empty or not specified, 1 is used by default
  inSampleFreq = the desired sample frequency for the output signal (in Hz)
                 if empty or not specified, 22050 is used by default
  inPhaseFlag = for choosing whether random phase is to be used or not
                (1 to use random phase, 0 otherwise)
                if empty or not specified, 1 is used by default
  indBLevel = dB level of generated tone (in dB)
              if empty or not specified, no level adjustment is performed

\end{verbatim}
\textbf{Output:}
\begin{verbatim}  outSignal = the signal for the tone

\end{verbatim}
\textbf{Example:}
\begin{verbatim}  Signal = IPEMShepardTone(440,1,22050,1,-20);

\end{verbatim}
\textbf{Authors:}
\begin{verbatim}  Koen Tanghe - 20010116
\end{verbatim}


\newpage
\section*{\hyperlink{Concepts:IPEMShepardToneComplex}{IPEMShepardToneComplex}\sectionmark{IPEMShepardToneComplex}}
\addcontentsline{toc}{subsection}{IPEMShepardToneComplex}
\hypertarget{FuncRef:IPEMShepardToneComplex}{}

\textbf{Usage:}
\begin{verbatim}  outSignal = IPEMShepardToneComplex (inToneVector,inDuration,inSampleFreq,...
                                      inPhaseFlag,indBLevel)

\end{verbatim}
\textbf{Description:}
\begin{verbatim}  This function generates a tone complex built up of Shepard tones.

\end{verbatim}
\textbf{Input arguments:}
\begin{verbatim}  inToneVector = a 12 elements vector representing the amplitude for each tone
                 C, C#, D, ... in the tone complex (0 = no tone, 1 = full
                 amplitude)
  inDuration = the duration (in s)
               if empty or not specified, 1 is used by default
  inSampleFreq = the desired sample frequency for the output signal (in Hz)
                 if empty or not specified, 22050 is used by default
  inPhaseFlag = for choosing whether random phase has to be used or not
                (1 to use random phase, 0 otherwise)
                if empty or not specified, 1 is used by default
  indBLevel = dB level of generated tone complex (in dB)
              if empty or not specified, no level adjustment is performed

\end{verbatim}
\textbf{Output:}
\begin{verbatim}  outSignal = the signal for the tone complex

\end{verbatim}
\textbf{Example:}
\begin{verbatim}  Signal = IPEMShepardToneComplex([1 0 0 0 1 0 0 1 0 0 0 0],1,22050,1,-20);

\end{verbatim}
\textbf{Authors:}
\begin{verbatim}  Koen Tanghe - 20010116
\end{verbatim}


\newpage
\section*{\hyperlink{Concepts:IPEMShowSoundFile}{IPEMShowSoundFile}\sectionmark{IPEMShowSoundFile}}
\addcontentsline{toc}{subsection}{IPEMShowSoundFile}
\hypertarget{FuncRef:IPEMShowSoundFile}{}

\textbf{Usage:}
\begin{verbatim}  IPEMShowSoundFile(inFileName,inFilePath,inNewFigureFlag)

\end{verbatim}
\textbf{Description:}
\begin{verbatim}  This function shows the waveform of the specified sound file.
  Only .wav and .snd sound files are supported.

\end{verbatim}
\textbf{Input arguments:}
\begin{verbatim}  inFileName = the name of the sound file
  inFilePath = the path to the sound file
               if empty or not specified, IPEMRootDir('input')\Sounds is used
               by default
  inFigureFlag = if non-zero or not specified, a new figure is created
                 if 0, the graph will be shown in the current (sub)plot

\end{verbatim}
\textbf{Example:}
\begin{verbatim}  IPEMShowSoundFile('music.wav',[],0);

\end{verbatim}
\textbf{Authors:}
\begin{verbatim}  Koen Tanghe - 20000209
\end{verbatim}


\newpage
\section*{\hyperlink{Concepts:IPEMSineComplex}{IPEMSineComplex}\sectionmark{IPEMSineComplex}}
\addcontentsline{toc}{subsection}{IPEMSineComplex}
\hypertarget{FuncRef:IPEMSineComplex}{}

\textbf{Usage:}
\begin{verbatim}  [outSignal,outSubSignals] = ...
    IPEMSineComplex(inSampleFreq,inDuration,...
                    inFrequencies,inAmplitudes,inPhases,indBLevel)

\end{verbatim}
\textbf{Description:}
\begin{verbatim}  Generates a signal consisting of pure sinusoids with constant amplitude
  and phase.

\end{verbatim}
\textbf{Input arguments:}
\begin{verbatim}  inSampleFreq = wanted sample frequency (in Hz)
  inDuration = wanted duration (in s)
  inFrequencies = frequency components (in Hz)
  inAmplitudes = amplitudes for the frequency components
                 if empty or not specified, 1 is used for all components
  inPhases = phases for the frequency components (in radians)
             if you specify 'random', random phase between 0 and 2*pi is used
             if empty or not specified, 0 is used for all components
  indBLevel = normalization level (in dB) for the generated sound
              (the signals in outSubSignals are not normalized)
              if empty or not specified, no normalization is performed

\end{verbatim}
\textbf{Output:}
\begin{verbatim}  outSignal = the generated signal
  outSubSignals = multi-channel signal where each row contains the generated
                  sine wave for the corresponding frequency

\end{verbatim}
\textbf{Example:}
\begin{verbatim}  s = IPEMSineComplex(22050,1,[110 220 550 660],[1 0.5 0.8 0.4],'random',-6);

\end{verbatim}
\textbf{Authors:}
\begin{verbatim}  Koen Tanghe - 20020506
\end{verbatim}


\newpage
\section*{\hyperlink{Concepts:IPEMSnipSoundFile}{IPEMSnipSoundFile}\sectionmark{IPEMSnipSoundFile}}
\addcontentsline{toc}{subsection}{IPEMSnipSoundFile}
\hypertarget{FuncRef:IPEMSnipSoundFile}{}

\textbf{Usage:}
\begin{verbatim}  [outSignalSegments,outSampleFreq] = ...
    IPEMSnipSoundFile(inFileName,inFilePath,inTimeSegments,...
                      inWriteSegments,inOutputPath)

\end{verbatim}
\textbf{Description:}
\begin{verbatim}  Snips the sound file into segments.

\end{verbatim}
\textbf{Input arguments:}
\begin{verbatim}  inFileName = name of the sound file
  inFilePath = location of the sound file
               if empty or not specified, IPEMRootDir('input')\Sounds is
               used by default
  inTimeSegments = time segments for cutting up the input signal (in s)
                   (each row contains the start and end of a time segment)
                   if this is a positive scalar, the signal is divided into
                   inTimeSegments parts of equal size + an additional segment
                   containing the remaining part of the signal (which will
                   contain less than inTimeSegments samples)
                   if this is a negative scalar, the signal is divided into
                   -inTimeSegments parts where some segments will contain
                   1 sample more than other segments
  inWriteSegments = if non-zero, the segments are written to separate sound
                    files in inOutputPath (the name of the files is the
                    original name appended with the number of the segment)
                    if empty or not specified, 0 is used by default
  inOutputPath = ouput path used when sound files are written
                 if empty or not specified, the same directory as the sound
                 file is used by default

\end{verbatim}
\textbf{Output:}
\begin{verbatim}  outSignalSegments = the wanted signal segments
  outSampleFreq = sample frequency of the original sound file (and segments)
                  (in Hz)

\end{verbatim}
\textbf{Example:}
\begin{verbatim}  IPEMSnipSoundFile('dnb_drumloop_1patt.wav',[],-16,'E:\Koen\Data\Temp');

\end{verbatim}
\textbf{Authors:}
\begin{verbatim}  Koen Tanghe - 20000419
\end{verbatim}


\newpage
\section*{\hyperlink{Concepts:IPEMSnipSoundFileAtOnsets}{IPEMSnipSoundFileAtOnsets}\sectionmark{IPEMSnipSoundFileAtOnsets}}
\addcontentsline{toc}{subsection}{IPEMSnipSoundFileAtOnsets}
\hypertarget{FuncRef:IPEMSnipSoundFileAtOnsets}{}

\textbf{Usage:}
\begin{verbatim}  IPEMSnipSoundFileAtOnsets (inFileName,inFilePath,inOutputPath)

\end{verbatim}
\textbf{Description:}
\begin{verbatim}  Calculates the onsets for the specified sound file and cuts the sound file
  into new sound segments at these onset times.

\end{verbatim}
\textbf{Input arguments:}
\begin{verbatim}  inFileName = name of the sound file
  inFilePath = location of the sound file
               if empty or not specified, IPEMRootDir('input')\Sounds is
               used by default
  inOutputPath = path where the sound segments should be stored
                 if empty or not specified, the same directory as the sound
                 file is used by default

\end{verbatim}
\textbf{Example:}
\begin{verbatim}  IPEMSnipSoundFileAtOnsets('dnb_drumloop_1patt.wav',[],...
                            'E:\Koen\SnippedSamples');

\end{verbatim}
\textbf{Authors:}
\begin{verbatim}  Koen Tanghe - 20000209
\end{verbatim}


\newpage
\section*{\hyperlink{Concepts:IPEMStripFileSpecification}{IPEMStripFileSpecification}\sectionmark{IPEMStripFileSpecification}}
\addcontentsline{toc}{subsection}{IPEMStripFileSpecification}
\hypertarget{FuncRef:IPEMStripFileSpecification}{}

\textbf{Usage:}
\begin{verbatim}  outParts = IPEMStripFileSpecification (inFileSpecification)

\end{verbatim}
\textbf{Description:}
\begin{verbatim}  Strips a complete file specification into its subparts.
  Can also be used to strip directory paths in their subparts.

\end{verbatim}
\textbf{Input arguments:}
\begin{verbatim}  inFileSpecification = the complete file specification

\end{verbatim}
\textbf{Output:}
\begin{verbatim}  outParts = a struct containing the following information:
  outParts.FileName = string containing the name of the file
  outParts.Name = name of the file without extension
  outParts.Extension = extension of the file
  outParts.Path = complete path towards the file
  outParts.PathParts = cell array containing the subparts of the path
                       specification

\end{verbatim}
\textbf{Remarks:}
\begin{verbatim}  1. Only tested on Windows!
  2. If the specification doesn't end on an extension, it is assumed to be a
     directory path.

\end{verbatim}
\textbf{Example:}
\begin{verbatim}  theParts = IPEMStripFileSpecification('D:\Koen\Data\Sounds\schum1.wav');
  --->
  theParts.FileName = 'schum1.wav'
  theParts.Name = 'schum1'
  theParts.Extension = 'wav'
  theParts.Path = 'D:\Koen\Data\Sounds'
  theParts.PathParts = {'D:' 'Koen' 'Data' 'Sounds'}

\end{verbatim}
\textbf{Authors:}
\begin{verbatim}  Koen Tanghe - 20000607
\end{verbatim}


\newpage
\section*{\hyperlink{Concepts:IPEMToneScaleComparison}{IPEMToneScaleComparison}\sectionmark{IPEMToneScaleComparison}}
\addcontentsline{toc}{subsection}{IPEMToneScaleComparison}
\hypertarget{FuncRef:IPEMToneScaleComparison}{}

\textbf{Usage:}
\begin{verbatim}  [outError,outBestRatioDiffs] = IPEMToneScaleComparison;

\end{verbatim}
\textbf{Description:}
\begin{verbatim}  Compares tonescales with different subdivisions to the 'ideal case'

\end{verbatim}
\textbf{Output:}
\begin{verbatim}  outError = error compared to ideal case
  outBestRatioDiffs = difference of the best ratio's with the ideal ones

\end{verbatim}
\textbf{Remarks:}
\begin{verbatim}  Works for the 3 ideal ratio's 6/5, 5/4, 3/2

\end{verbatim}
\textbf{Authors:}
\begin{verbatim}  Koen Tanghe - 19990816
\end{verbatim}


\newpage
\section*{\hyperlink{Concepts:IPEMToolboxVersion}{IPEMToolboxVersion}\sectionmark{IPEMToolboxVersion}}
\addcontentsline{toc}{subsection}{IPEMToolboxVersion}
\hypertarget{FuncRef:IPEMToolboxVersion}{}

\textbf{Usage:}
\begin{verbatim}  IPEMToolboxVersion

\end{verbatim}
\textbf{Description:}
\begin{verbatim}  Returns the version of the IPEM Toolbox

\end{verbatim}
\textbf{Output:}
\begin{verbatim}  outVersionString = string with full version information
  outMajor = major version number
  outMinor = minor version number
  outType = version type string (either 'beta' or empty)
  outBuildDate = build date of this version using the format 'YYYYMMDD',
                 where YYYY = year, MM = month and DD = day

\end{verbatim}
\textbf{Authors:}
\begin{verbatim}  Koen Tanghe - 20050120
\end{verbatim}


\newpage
\section*{\hyperlink{Concepts:IPEMWriteSoundFile}{IPEMWriteSoundFile}\sectionmark{IPEMWriteSoundFile}}
\addcontentsline{toc}{subsection}{IPEMWriteSoundFile}
\hypertarget{FuncRef:IPEMWriteSoundFile}{}

\textbf{Usage:}
\begin{verbatim}  outResult = IPEMWriteSoundFile(inSignal,inSampleFreq,inFileName,...
                                 inFilePath,inSection)

\end{verbatim}
\textbf{Description:}
\begin{verbatim}  Writes (part of) a sound signal to file.

\end{verbatim}
\textbf{Input arguments:}
\begin{verbatim}  inSignal = sound signal (each row represents a channel)
  inSampleFreq = sample frequency of sound signal (in Hz)
  inFileName = name for the sound file (with extension)
               if empty or not specified, the standard "Write file" dialog box
               can be used to select a sound file to write to
  inFilePath = path for the sound file
               if empty or not specified, IPEMRootDir('input')\Sounds is used
               by default
  inSection = section of the sound signal to be written, given as [start end]
              (in s)
              if empty or not specified, the entire sound signal is written

\end{verbatim}
\textbf{Output:}
\begin{verbatim}  outResult = usually 1
              0 if the user pressed 'Cancel' when selecting a sound file

\end{verbatim}
\textbf{Remarks:}
\begin{verbatim}  The 'end' in the section specification is exclusive:
  if you want to write the first second of a sound signal sampled at 1000 Hz,
  you should specify [0 1] for inSection (and not [0 0.999]). This will read
  samples 1 to 1000, which is a full second (the sample at time 1 s is not
  written!).

  Always writes 16-bit.

\end{verbatim}
\textbf{Example:}
\begin{verbatim}  IPEMWriteSoundFile(s,fs,'test.wav');

\end{verbatim}
\textbf{Authors:}
\begin{verbatim}  Koen Tanghe - 20040323
\end{verbatim}


\newpage
\sectionmark{MEC pattern extraction application}
~\vfill
\begin{center}
   \section*{Demo functions}
   \subsection*{MEC pattern extraction application}
\end{center}
\vfill
\addcontentsline{toc}{section}{Demo functions - MEC pattern extraction application}

\newpage
\section*{\hyperlink{Concepts:IPEMDemoMECRhythmExtraction}{IPEMDemoMECRhythmExtraction}\sectionmark{IPEMDemoMECRhythmExtraction}}
\addcontentsline{toc}{subsection}{IPEMDemoMECRhythmExtraction}
\hypertarget{FuncRef:IPEMDemoMECRhythmExtraction}{}

\textbf{Usage:}
\begin{verbatim}  IPEMDemoMECRhythmExtraction

\end{verbatim}
\textbf{Description:}
\begin{verbatim}  Starts the demonstration of the MEC algorithm used for the extraction of
  rhythmic patterns in sound.

\end{verbatim}
\textbf{Authors:}
\begin{verbatim}  Koen Tanghe - 20020912
\end{verbatim}


\newpage
\section*{\hyperlink{Concepts:IPEMDemoStartMEC}{IPEMDemoStartMEC}\sectionmark{IPEMDemoStartMEC}}
\addcontentsline{toc}{subsection}{IPEMDemoStartMEC}
\hypertarget{FuncRef:IPEMDemoStartMEC}{}

\textbf{Usage:}
\begin{verbatim}  [outOriginalSound,outResynthesizedSound,outSampleFreq,outDetectedPeriods,...
   outIndividualSounds] = ...
    IPEMDemoStartMEC(inSoundFile,inSoundFilePath,inMinPeriod,inMaxPeriod,...
                     inSelectionTime,inStep,inHalfDecayTime,...
                     inResynthDuration,inRescalePatterns,...
                     inPatternIntegrationTime,inEnhanceContrast,...
                     inSaveMECResults,inUseMode);

\end{verbatim}
\textbf{Description:}
\begin{verbatim}  Starts an entire MEC analysis and resynthesis run for the given sound file.

\end{verbatim}
\textbf{Input arguments:}
\begin{verbatim}  inSoundFile = sound file to be processed
  inSoundFilePath = path to the sound file
                    if empty or not specified, IPEMRootDir('input')\Sounds
                    is used by default
  inMinPeriod = minimum period to look for (in s)
                if empty or not specified, 0.5 is used by default
  inMaxPeriod = maximum period to look for (in s)
                if empty or not specified, 5 is used by default
  inSelectionTime = selected moment in time to extract current pattern (in s)
                    if empty or not specified, 3 is used by default
  inStep = integer specifying number of samples between successive
           calculations
           if empty or not specified, 1 is used by default (every sample)
  inHalfDecayTime = half decay time for leaky integration of energy
                    differences (in s)
                    if empty or not specified, 1.5 is used by default
  inResynthDuration = duration (in s) of resynthesized sound
                      if empty or not specified, 10 is used by default
  inRescalePatterns = if non-zero, patterns are rescaled between 0 and 1
                      if empty or not specified, 0 is used by default
  inPatternIntegrationTime = half decay time for leaky integration of patterns
                             (in s)
                             if empty or not specified, 0 is used by default
  inEnhanceContrast = if non-zero, the "contrast" in the resynthesized sound
                      is enhanced (see IPEMMECSynthesis for more details on
                      how this is done)
                      if empty or not specified, 0 is used by default
  inSaveMECResults = if non-zero, intermediate results of the analysis that
                     are necessary for later resynthesis, are saved to a .mat
                     file
                     if zero, empty or not specified, results are not saved
  inAnalysisType = if 1, the RMS of the sound signal is used
                   if 2, the RMS of 10 ANI ch. (summed diff. values) is used
                   if 3, the RMS of 10 ANI ch. (separate diff. values) is used
                   if empty or not specified, 1 is used by default

\end{verbatim}
\textbf{Output:}
\begin{verbatim}  outOriginalSound = original sound
  outResynthesizedSound = resynthesized sound
  outSampleFreq = sample frequency for all returned sounds (Hz)
  outDetectedPeriods = detected periods in each channel (in s)
  outIndividualSounds = individual resynthesized sounds per channel

\end{verbatim}
\textbf{Remarks:}
\begin{verbatim}  If saving of intermediate results is requested, data is saved to a .mat file
  with the name 'MEC_xxx.mat', where xxx is the name of the sound file without
  extension. The location of this file will be IPEMRootDir('output').

\end{verbatim}
\textbf{Example:}
\begin{verbatim}  [s,ss,fs,periods,sschan] = ...
    IPEMDemoStartMEC('ligeti.wav',[],0.4,5,3,2,1.5,10,1,0.5,1,1,1);

\end{verbatim}
\textbf{Authors:}
\begin{verbatim}  Koen Tanghe - 20020222
\end{verbatim}


