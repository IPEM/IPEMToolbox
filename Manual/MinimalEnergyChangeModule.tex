% --------------------------------------------------------------------------------
\newpage
\section{Rhythm Module}
% --------------------------------------------------------------------------------

% Make general target
\hypertarget{Concepts:MinimalEnergyChangeModule}{}

% Make target for following functions:
\hypertarget{Concepts:IPEMMECAnalysis}{}
\hypertarget{Concepts:IPEMMECExtractPatterns}{}
\hypertarget{Concepts:IPEMMECReSynthUI}{}
\hypertarget{Concepts:IPEMMECSynthesis}{}
\hypertarget{Concepts:IPEMMECSaveResults}{}

\subsection*{Introductory description}
% --------------------------------------------------------------------------------

The Minimal Energy Change Module (MECM) calculates the
fundamental period of a signal. The input is a sound file and the
output is an estimation of the fundamental period at each time
step. The technique used is a generalization of the Average
Magnitude Difference Function, known as AMDF
\citeyear{article:LemanVerbeke:Ieper:2000}. The basic idea is
that:
\begin{itemize}
\item
repeating patterns have a more or less constant energy over the
period of the repeating pattern
\item
minimal changes of this energy point to the period of the
repetitive pattern.
\end{itemize}
In applying MEC to rhythm detection we perform the analysis on
the energy patterns in the auditory nerve images. This module
then has its place in the image transformation chart as shown in
Fig.~\ref{ImageTransformationModulesMECM}.
\begin{figure}[h]
  \centering
\includegraphics[width=14cm]{IeperFigures/ImageTransformationModulesMECM}
  \caption{Chart of image transformation modules, with OM highlighted}
  \label{ImageTransformationModulesMECM}
\end{figure}


\subsection*{Formal logical description}
% --------------------------------------------------------------------------------
The  Minimal Energy Change Module applied to the recognition of
repetition in rhythm involves:
\begin{equation}
    APM: s(t) \rightarrow e(t,c) \\
    RMS: e(t,c) \rightarrow r(t,c)
\end{equation}
where APM provides the auditory nerve images and where RMS
considers the energy in these images. The application of MEC
involves:
\begin{equation}
    MEC1:  r(t,c) \rightarrow m(\tau,t,c)
    \end{equation}
where MEC1 performs a periodicity analysis of these energies and
outputs the estimated periodicity patterns $m(\tau,t,c)$, where
$t$ is time as usual, $\tau$ is the period, and $c$ is the
auditory channel.
To obtain the summary MEC-analysis, one has to
sum over all auditory channels, which gives:
\begin{equation}
    m(\tau,t) = \sum_{c=1}^{C} m(\tau,t,c)~=~\tilde{m}(t)
\end{equation}
Hence, the MEC-function can be summarized as:
\begin{equation}
    MEC:  \tilde{r}(t) \rightarrow \tilde{m}(t)
\end{equation}
In this notation, one should take into account that the tilde
character, indicating a vector representation, applies to
different dimensions. The components of $\tilde{r}(t)$ are the
channels, while the components of $\tilde{m}(t)$ are the
different periods considered in the analysis. A derived value is
the so-called best period, which we obtain by taking the minimum
in each $\tilde{m}(t)$ for fixed $t$, hence:
\begin{equation}
    MEC:  \tilde{r}(t) \rightarrow \tilde{m}(t) \rightarrow p(t)
\end{equation}
\subsection*{Signal processing description}
% --------------------------------------------------------------------------------
The MEC-algorithm takes a signal $s(t)$ and its delayed copy
$s(t-\tau)$. It then calculates the difference between both
signals, takes the absolute value, and integrates the result:
\begin{equation}\label{MECM1}
  a(\tau,t)~=~
   \int_{0}^{t}~\left|~
   s(t'-\tau)-s(t') ~\right|~
  }~dt'~
\end{equation}
where $a(\tau,t)$ at fixed $t$ contains an analysis for all
$\tau$. $\tau$ is typically chosen in the region of interest. For
rhythm detection, this may be from 400 ms to 1 s. The obtained
analysis $a(\tau,t)$ at fixed $t$ is then minimized:
\begin{equation}\label{MECM2}
  p(t) = ~min_{\tau>0}~a(\tau,t)
\end{equation}
which gives a value $p$ for the best period at each time step $t$.

In practice, we can build in more stability by leaky-integrating
the $a(\tau,t)$ over $t$:
\begin{equation}\label{MECM1B}
  a(\tau,t)~=~
   \int_{0}^{t}~\left|~
   s(t'-\tau)-s(t') ~\right|~
  e^{\beta(t-t')}~dt'
\end{equation}
and apply Expression \ref{MECM2}.

When applied to the signal in different auditory channels, the
calculation will be applied to each channel, thus giving
$a(\tau,t,c)$). To get the best period, one may calculate $p(t)$
using $\sum_{c=1}^{C}a(\tau,t,c)$.

\subsection*{Implementation}
% --------------------------------------------------------------------------------
\hyperlink{FuncRef:IPEMMECAnalysis}{IPEMMECAnalysis} - Performs a
periodicity analysis of a multi-channel signal using the MEC model

\subsection*{Examples}
% --------------------------------------------------------------------------------
The first steps involve the preparation of the signal in which we want to
look for periodicity.
This is done by:\\\\
\IPEMCodeExtract{
 [ANI,ANIFreq,ANIFilterFreq] = IPEMCalcANIFromFile('SchumannKurioseGeschichte.wav');
 }\\
\IPEMCodeExtract{
 [RMSSignal,RMSFreq] = IPEMCalcRMS(ANI,ANIFreq,0.05,0.005);
}\\
\IPEMCodeExtract{
 [Periods,Best,AnalysisFreq,Values] = IPEMMECAnalysis(RMSSignal,RMSFreq,1/4,1,0.050,1.5);
}\\
The MEC Module has stored the time code for the periods in
\IPEMCodeExtract{Periods}, the best periods found in all channels
in \IPEMCodeExtract{Best}, and all values $m(\tau,t,c)$ are
stored in \IPEMCodeExtract{Values}. The dataformat of
\IPEMCodeExtract{Values} is described in
\hyperlink{FuncRef:IPEMMECAnalysis}{IPEMMECAnalysis}. Just to give
you an example of how to work with MATLAB, we sum over all
channels $c$ in the
following way:\\\\
 \begin{IPEMCodeEnvironment}
 SummaryPeriodicity = zeros(size(Values{1}(:,:))); \\
 for channel = 1:40\\
  SummaryPeriodicity = SummaryPeriodicity + Values{channel}(:,:);\\
 end\\
\end{IPEMCodeEnvironment}\\
The values in \IPEMCodeExtract{SummaryPeriodicity} contain the
periodicity analysis over all channels which can be visualized
using the MATLAB function:\\\\
 \begin{IPEMCodeEnvironment}
  imagesc((1:length(SummaryPeriodicity))/AnalysisFreq,Periods,SummaryPeriodicity);\\
  axis xy;colormap (1-gray)\\
\end{IPEMCodeEnvironment}\\
With\\\\
 \begin{IPEMCodeEnvironment}
 [M,I]=min(SummaryPeriodicity); hold on;\\
 plot((1:length(I))/AnalysisFreq,Periods(I));\\
\end{IPEMCodeEnvironment}\\
we plot the best periods of the summary periodicity on top of the
image. Notice that the periodicity is found after 2 s.
